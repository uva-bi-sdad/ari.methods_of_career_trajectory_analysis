% Options for packages loaded elsewhere
% Options for packages loaded elsewhere
\PassOptionsToPackage{unicode}{hyperref}
\PassOptionsToPackage{hyphens}{url}
\PassOptionsToPackage{dvipsnames,svgnames,x11names}{xcolor}
%
\documentclass[
  letterpaper,
  DIV=11,
  numbers=noendperiod]{scrartcl}
\usepackage{xcolor}
\usepackage{amsmath,amssymb}
\setcounter{secnumdepth}{-\maxdimen} % remove section numbering
\usepackage{iftex}
\ifPDFTeX
  \usepackage[T1]{fontenc}
  \usepackage[utf8]{inputenc}
  \usepackage{textcomp} % provide euro and other symbols
\else % if luatex or xetex
  \usepackage{unicode-math} % this also loads fontspec
  \defaultfontfeatures{Scale=MatchLowercase}
  \defaultfontfeatures[\rmfamily]{Ligatures=TeX,Scale=1}
\fi
\usepackage{lmodern}
\ifPDFTeX\else
  % xetex/luatex font selection
\fi
% Use upquote if available, for straight quotes in verbatim environments
\IfFileExists{upquote.sty}{\usepackage{upquote}}{}
\IfFileExists{microtype.sty}{% use microtype if available
  \usepackage[]{microtype}
  \UseMicrotypeSet[protrusion]{basicmath} % disable protrusion for tt fonts
}{}
\makeatletter
\@ifundefined{KOMAClassName}{% if non-KOMA class
  \IfFileExists{parskip.sty}{%
    \usepackage{parskip}
  }{% else
    \setlength{\parindent}{0pt}
    \setlength{\parskip}{6pt plus 2pt minus 1pt}}
}{% if KOMA class
  \KOMAoptions{parskip=half}}
\makeatother
% Make \paragraph and \subparagraph free-standing
\makeatletter
\ifx\paragraph\undefined\else
  \let\oldparagraph\paragraph
  \renewcommand{\paragraph}{
    \@ifstar
      \xxxParagraphStar
      \xxxParagraphNoStar
  }
  \newcommand{\xxxParagraphStar}[1]{\oldparagraph*{#1}\mbox{}}
  \newcommand{\xxxParagraphNoStar}[1]{\oldparagraph{#1}\mbox{}}
\fi
\ifx\subparagraph\undefined\else
  \let\oldsubparagraph\subparagraph
  \renewcommand{\subparagraph}{
    \@ifstar
      \xxxSubParagraphStar
      \xxxSubParagraphNoStar
  }
  \newcommand{\xxxSubParagraphStar}[1]{\oldsubparagraph*{#1}\mbox{}}
  \newcommand{\xxxSubParagraphNoStar}[1]{\oldsubparagraph{#1}\mbox{}}
\fi
\makeatother


\usepackage{longtable,booktabs,array}
\usepackage{calc} % for calculating minipage widths
% Correct order of tables after \paragraph or \subparagraph
\usepackage{etoolbox}
\makeatletter
\patchcmd\longtable{\par}{\if@noskipsec\mbox{}\fi\par}{}{}
\makeatother
% Allow footnotes in longtable head/foot
\IfFileExists{footnotehyper.sty}{\usepackage{footnotehyper}}{\usepackage{footnote}}
\makesavenoteenv{longtable}
\usepackage{graphicx}
\makeatletter
\newsavebox\pandoc@box
\newcommand*\pandocbounded[1]{% scales image to fit in text height/width
  \sbox\pandoc@box{#1}%
  \Gscale@div\@tempa{\textheight}{\dimexpr\ht\pandoc@box+\dp\pandoc@box\relax}%
  \Gscale@div\@tempb{\linewidth}{\wd\pandoc@box}%
  \ifdim\@tempb\p@<\@tempa\p@\let\@tempa\@tempb\fi% select the smaller of both
  \ifdim\@tempa\p@<\p@\scalebox{\@tempa}{\usebox\pandoc@box}%
  \else\usebox{\pandoc@box}%
  \fi%
}
% Set default figure placement to htbp
\def\fps@figure{htbp}
\makeatother





\setlength{\emergencystretch}{3em} % prevent overfull lines

\providecommand{\tightlist}{%
  \setlength{\itemsep}{0pt}\setlength{\parskip}{0pt}}



 


\KOMAoption{captions}{tableheading}
\makeatletter
\@ifpackageloaded{caption}{}{\usepackage{caption}}
\AtBeginDocument{%
\ifdefined\contentsname
  \renewcommand*\contentsname{Table of contents}
\else
  \newcommand\contentsname{Table of contents}
\fi
\ifdefined\listfigurename
  \renewcommand*\listfigurename{List of Figures}
\else
  \newcommand\listfigurename{List of Figures}
\fi
\ifdefined\listtablename
  \renewcommand*\listtablename{List of Tables}
\else
  \newcommand\listtablename{List of Tables}
\fi
\ifdefined\figurename
  \renewcommand*\figurename{Figure}
\else
  \newcommand\figurename{Figure}
\fi
\ifdefined\tablename
  \renewcommand*\tablename{Table}
\else
  \newcommand\tablename{Table}
\fi
}
\@ifpackageloaded{float}{}{\usepackage{float}}
\floatstyle{ruled}
\@ifundefined{c@chapter}{\newfloat{codelisting}{h}{lop}}{\newfloat{codelisting}{h}{lop}[chapter]}
\floatname{codelisting}{Listing}
\newcommand*\listoflistings{\listof{codelisting}{List of Listings}}
\makeatother
\makeatletter
\makeatother
\makeatletter
\@ifpackageloaded{caption}{}{\usepackage{caption}}
\@ifpackageloaded{subcaption}{}{\usepackage{subcaption}}
\makeatother
\usepackage{bookmark}
\IfFileExists{xurl.sty}{\usepackage{xurl}}{} % add URL line breaks if available
\urlstyle{same}
\hypersetup{
  pdftitle={Promotion Board Analysis for Analyzing Career Trajectories},
  colorlinks=true,
  linkcolor={blue},
  filecolor={Maroon},
  citecolor={Blue},
  urlcolor={Blue},
  pdfcreator={LaTeX via pandoc}}


\title{Promotion Board Analysis for Analyzing Career Trajectories}
\author{}
\date{}
\begin{document}
\maketitle


Promotion Board Analysis offers a powerful methodological framework for
understanding the factors that shape military career trajectories and
promotion outcomes. By combining rigorous statistical approaches with
comprehensive data on officer careers, this methodology can provide
valuable insights for both organizational leadership and individual
officers. The approach is particularly valuable for identifying optimal
developmental pathways, understanding the relative importance of
different career experiences, and ensuring alignment between stated
promotion criteria and actual promotion outcomes.

\subsection{1. Approach Description \&
Goal}\label{approach-description-goal}

Promotion Board Analysis is a methodological approach that examines the
decision-making processes and outcomes of formal promotion systems,
particularly in hierarchical organizations like the military. This
approach systematically analyzes the factors that influence promotion
decisions, the patterns in promotion rates across different groups, and
the relationship between promotion outcomes and various career and
personal attributes.

The primary goals of Promotion Board Analysis include: (1) understanding
which factors most significantly impact promotion success; (2)
identifying potential biases or inconsistencies in the promotion system;
(3) predicting future promotion outcomes and career trajectories; (4)
informing policy decisions related to professional development and
talent management; and (5) optimizing organizational human resource
practices to ensure fair and merit-based advancement opportunities.

In military contexts, this approach is particularly valuable as
promotion boards operate according to structured processes with clearly
defined criteria, providing rich data for statistical analysis. The
approach typically combines quantitative methods like survival analysis,
logistic regression, and multi-state modeling with qualitative
assessments of promotion board documentation and policies.

\subsection{2. Critical Variables}\label{critical-variables}

Promotion Board Analysis typically incorporates several categories of
variables as inputs:

\textbf{Personal Demographics}: - Age/year group - Gender -
Race/ethnicity - Marital status and family composition

\textbf{Educational Background}: - Commissioning source (military
academy, ROTC, OCS) - Education level (bachelor's, master's, doctorate)
- Academic performance metrics (GPA, class rank) - Academic
major/specialization - Advanced military education completion

\textbf{Professional Qualifications}: - Time in service and time in
grade - Key developmental positions held - Command time and staff
experience - Joint service assignments - Deployment experience
(locations, duration, combat vs.~non-combat)

\textbf{Performance Metrics}: - Evaluation report ratings - Awards and
decorations - Physical fitness scores - Specialized skills and
certifications

\textbf{Career Management Factors}: - Branch/specialty alignment with
organizational needs - Career field/branch transfers - Utilization of
talent management programs - Prior enlisted service experience -
Mentorship relationships

\textbf{External Factors}: - Promotion opportunity percentages -
Year-specific policies and focus areas - Organizational structure
changes - War/peacetime context

The analysis typically examines these variables both individually and in
combination to understand their relative influence on promotion outcomes
and career trajectories.

\subsection{3. Key Overviews}\label{key-overviews}

McAllaster's ``Forecasting Army Officer Attrition with Logistic
Regression Using The SAS System'' (1999) provides a foundational
overview of applying statistical methods to military personnel analysis.
The paper explains how Army personnel analysts model and forecast
officer attrition using logistic regression, applying this approach to
predict binary outcomes (whether officers stay or leave the Army).
McAllaster's methodology involves analyzing over 150 homogeneous cohorts
of officers using 14 years of legacy data. The paper demonstrates how
SAS Software can be leveraged to summarize vast amounts of data (over a
million observations), model binomial outcomes and forecast trends using
logistic regression, and repeat the statistical analysis code using
macro language. This work established important analytical frameworks
for understanding career trajectories within the military
context{[}5{]}.

Putter and colleagues' application of multi-state models in their R
package `mstate' (2024) provides a comprehensive methodological
framework for analyzing career transitions. Their approach extends
beyond simple binary outcomes to model complex career trajectories with
multiple possible states and transitions. The package implements
functions for data preparation, descriptive statistics, hazard
estimation, and prediction using either Aalen-Johansen estimators or
simulation techniques. Their methodology is particularly suited for
analyzing competing risks (such as promotion versus attrition) and
multi-state transitions (such as movements between different ranks or
positions). This statistical approach provides researchers with tools to
understand not just whether officers are promoted but also the timing
and sequencing of career events{[}10{]}.

In ``Analysis of the Survival Patterns of United States Naval
Officers,'' researchers conduct a comprehensive survival analysis on
officer cohorts who entered service between 1983 and 1990. The study
employs three survival analysis procedures---LIFETEST, LIFEREG, and
PHREG---to examine factors influencing officer career longevity. The
research findings demonstrate that commissioning source has significant
effects on survival rates, with Naval Academy graduates showing better
retention than officers from other commissioning sources. The study
provides valuable insights into how demographic variables (gender,
race), educational background, and career specialization affect officer
retention, showing different patterns for voluntary versus involuntary
separations. This work establishes survival analysis as a powerful tool
for understanding military career trajectories{[}6{]}.

``The Impact of Removing Demographic Indicators from Military Promotion
Boards'' (2024) presents findings from a comprehensive study conducted
by the Institute for Defense Analyses (IDA) on potential bias in
military promotion systems. Using a mixed-methods approach, IDA
researchers analyzed historical officer promotion data across all
military branches to assess whether removing demographic indicators from
promotion board materials would reduce bias. After controlling for
various factors including rank, year, and competitive category, their
analysis found that past policy changes removing some demographic
indicators did not significantly impact promotion rates for minority
officers. The study also examined how names potentially indicative of
minority status predicted promotion outcomes, finding no significant
associations. This research provides important methodological insights
into how to analyze potential bias in promotion systems{[}7{]}.

\subsection{4. Mathematical Approach}\label{mathematical-approach}

Promotion Board Analysis employs several statistical methodologies to
model and analyze career progression, with the most common approaches
being survival analysis, logistic regression, and multi-state modeling.

\textbf{Survival Analysis / Event History Analysis}: This approach
models the time until an event of interest (such as promotion) occurs.
The basic concept involves estimating the survival function:

\[ S(t) = Pr(T > t) \]

which represents the probability that an officer's time to promotion (T)
exceeds a specified time (t). The hazard function, representing the
instantaneous rate of promotion at time t given survival up to that
point, is defined as:

\[ h(t) = \lim_{\Delta t \to 0} \frac{Pr(t \leq T < t + \Delta t | T \geq t)}{\Delta t} \]

The Cox proportional hazards model, a semi-parametric approach commonly
used in Promotion Board Analysis, relates the hazard function to a set
of covariates:

\[ h(t|X) = h_0(t) \exp(\beta_1 X_1 + \beta_2 X_2 + ... + \beta_p X_p) \]

where h₀(t) is the baseline hazard function and X₁, X₂, \ldots, Xₚ are
covariates (such as educational background, performance ratings, etc.).

\textbf{Logistic Regression}: For binary outcomes (promoted vs.~not
promoted), logistic regression models the log-odds of promotion as a
linear function of predictor variables:

\[ \log\left(\frac{p}{1-p}\right) = \beta_0 + \beta_1 X_1 + \beta_2 X_2 + ... + \beta_p X_p \]

where p is the probability of promotion, and X₁, X₂, \ldots, Xₚ are
predictor variables. The coefficients (β) represent the change in
log-odds associated with a one-unit increase in the corresponding
predictor.

\textbf{Multi-state Modeling}: This approach models transitions between
different career states (e.g., Lieutenant to Captain to Major, or active
duty to separation). The transition intensity or hazard from state r to
state s is defined as:

\[ \alpha_{rs}(t|Z) = \lim_{\Delta t \to 0} \frac{P(S(t + \Delta t) = s | S(t) = r, Z)}{\Delta t} \]

where S(t) is the state at time t and Z represents covariates. These
transition intensities can be modeled using Cox-type models:

\[ \alpha_{rs}(t|Z) = \alpha_{rs,0}(t) \exp(\beta_{rs}^T Z) \]

\textbf{Competing Risks Analysis}: In the military context, officers
face multiple possible career outcomes (promotion, voluntary separation,
involuntary separation, etc.). Competing risks analysis, a specialized
form of survival analysis, models the cause-specific hazard for each
possible outcome:

\[ h_k(t) = \lim_{\Delta t \to 0} \frac{P(t \leq T < t + \Delta t, K = k | T \geq t)}{\Delta t} \]

where K represents the specific type of event (e.g., K=1 for promotion,
K=2 for voluntary separation).

\textbf{Practical Implementation}: In practice, Promotion Board Analysis
often involves: 1. Data preparation, including creation of person-period
datasets 2. Descriptive analysis using Kaplan-Meier survival curves to
visualize promotion rates over time 3. Model fitting using Cox
proportional hazards models or logistic regression 4. Model validation
and diagnostics 5. Interpretation of coefficients to identify
significant predictors of promotion 6. Prediction of future promotion
outcomes for individuals or cohorts

These mathematical approaches allow researchers to quantify the effects
of various factors on promotion outcomes while accounting for the
time-dependent nature of career progression.

\subsection{5. Example Applications}\label{example-applications}

``Analysis of the Survival Patterns of United States Naval Officers''
presents a comprehensive application of survival analysis to understand
officer retention and career trajectories in the U.S. Navy. The study
analyzed officer cohorts who entered service between 1983 and 1990,
using Navy Officer Data Card information and annual promotion board
results. Three survival analysis procedures---LIFETEST, LIFEREG, and
PHREG---were employed to examine factors influencing career longevity.
Key findings revealed that commissioning source significantly affects
survival rates, with Naval Academy graduates showing better retention
than officers from other sources. Interestingly, the research found that
females and African-Americans demonstrated better survival rates than
males and whites, and that prior enlisted personnel and older officers
had higher survival rates. When analyzing voluntary and involuntary
separations separately, the factors showed different effects---for
example, being African-American had negative effects on involuntary
separations but positive effects on voluntary separations. This study
exemplifies how survival analysis can provide nuanced insights into the
complex interplay of factors affecting military career
trajectories{[}6{]}.

``An Analysis of Selected Army Promotion Board Results'' examines Army
officer professional development by analyzing Lieutenant Colonel
promotion board outcomes. The study focuses on comparing the pre-OPMS
(Officer Personnel Management System) and OPMS approaches to officer
development, specifically investigating whether OPMS is meeting its
stated goals through promotion patterns. The research methodology
involved contingency table analysis and individual tests for differences
in proportions for each specialty listed as over or under aligned at the
time of the promotion board's convening. A key finding was that
promotion under OPMS was not alleviating specialty alignment problems
for the analyzed lists. The researchers recommended a two-step course of
action focused on providing better guidance to promotion boards to
address specialty alignment issues. This study demonstrates how
analyzing promotion board results can reveal systemic issues in career
management systems and inform policy recommendations{[}18{]}.

``Forecasting Army Officer Attrition with Logistic Regression'' details
how Army personnel analysts model and predict officer career decisions
using logistic regression. The paper explains a statistical application
that employs a simple logistic regression model with fiscal year as the
predictor variable to forecast whether officers stay or leave the Army.
Despite the simplicity of the underlying model, the complete application
is described as large and complex because it runs logistic regression
against more than 150 homogeneous officer cohorts. The methodology
includes a semi-automatic capability to identify and exclude
inappropriate outliers through SAS Macro Language. This approach
demonstrates how statistical modeling can process vast amounts of
historical data (fourteen years and over a million observations) to
identify patterns and predict future outcomes in military career
trajectories. The application showcases how logistic regression can be
applied at scale to inform workforce planning and talent management in
large organizations{[}5{]}.

``The Impact of Removing Demographic Indicators from Military Promotion
Boards'' presents findings from an Institute for Defense Analyses (IDA)
study examining whether removing race, ethnicity, and gender identifiers
from promotion board materials would reduce bias in military promotion
systems. IDA researchers analyzed historical officer promotion data
across all military branches, specifically examining the impact of past
policy changes that removed certain demographic indicators (such as
candidate photos) on minority officer promotion rates. After controlling
for rank, year, competitive category, and linear time trends, the
analysis found that these policy changes did not significantly impact
promotion rates for minority officers relative to non-minority officers.
Additionally, the researchers examined whether names potentially
indicative of minority status predicted promotion outcomes, finding no
significant associations. The study concluded that completely removing
all identifying information would be highly challenging from a
feasibility standpoint and might not be the most effective approach to
reducing bias. This research provides valuable insights into
methodological approaches for analyzing potential bias in promotion
systems and evaluating the effectiveness of policy interventions{[}7{]}.

\subsection{6. Critiques}\label{critiques}

Promotion Board Analysis, while valuable for understanding career
trajectories, faces several significant limitations that researchers
should consider when employing this methodology.

\textbf{Data Limitations}: A fundamental challenge in Promotion Board
Analysis is access to complete, accurate, and timely data. Military
personnel records may contain inconsistencies, missing values, or
measurement errors that compromise analysis quality. As noted in the
study of Army promotion board results, researchers often struggle to
``accurately identify in the data base used those officers who were
considered for promotion''{[}18{]}. Additionally, longitudinal data
spanning entire careers may be unavailable or incomplete, particularly
for specialized career fields or during periods of organizational
change.

\textbf{Endogeneity and Selection Bias}: Promotion outcomes are not
randomly distributed but result from complex selection processes that
may already incorporate many of the variables being studied. Officers
self-select into different career tracks, and their decisions regarding
education, assignments, and specialization are endogenous to their
promotion prospects. This creates challenges in isolating causal effects
of specific factors on promotion outcomes. As demonstrated in the naval
officer survival patterns study, there are significant differences in
survival rates between voluntary and involuntary separations that
complicate interpretation of career trajectory patterns{[}6{]}.

\textbf{Changing Organizational Contexts}: Military promotion systems
evolve over time with changing policies, priorities, and organizational
structures. This temporal instability makes it difficult to compare
promotion patterns across different time periods or to build models with
consistent predictive validity. The IDA study on demographic indicators
in promotion boards notes that ``promotion boards have already adopted
many best practices to mitigate bias,'' suggesting that the promotion
system itself is a moving target for analysis{[}7{]}.

\textbf{Limited Insight into Decision Processes}: Quantitative analysis
of promotion outcomes provides limited insight into the actual
decision-making processes of promotion boards. The deliberations of
board members are typically confidential, and the weights assigned to
different factors may vary across boards and individuals. This ``black
box'' aspect limits researchers' ability to fully understand how
different variables influence promotion decisions in practice.

\textbf{Methodological Challenges}: The statistical methods commonly
used in Promotion Board Analysis have their own limitations. Survival
analysis and logistic regression both require assumptions about the
functional form of relationships and the distribution of residuals that
may not hold in practice. Additionally, these methods may struggle to
capture complex, non-linear relationships or interaction effects between
variables that influence promotion outcomes.

\textbf{External Validity Concerns}: Findings from Promotion Board
Analysis in one military branch or time period may not generalize to
other contexts. For example, factors that predict promotion success for
Army officers may differ from those for Navy officers, or patterns
observed during peacetime may differ from those during conflict periods.

These limitations suggest that Promotion Board Analysis should be
complemented with other methodological approaches, including qualitative
studies of promotion board proceedings, experimental designs testing
specific hypotheses about promotion decisions, and comparative studies
across different organizational contexts.

\subsection{7. Software}\label{software}

\textbf{R `survival' Package}: The `survival' package is a cornerstone
tool for analyzing time-to-event data in R, maintained primarily by
Terry Therneau. As described in the CRAN documentation, this package
contains ``core survival analysis routines, including definition of Surv
objects, Kaplan-Meier and Aalen-Johansen (multi-state) curves, Cox
models, and parametric accelerated failure time models''{[}9{]}. The
package offers functionality for handling right-censored, left-censored,
and interval-censored data, as well as time-dependent
covariates---features that are particularly valuable for analyzing
military career trajectories where some officers' outcomes may be
censored (e.g., still serving at the end of the observation period). The
`survival' package provides robust visualization capabilities for
survival curves and includes methods for model diagnostics and
validation. Its integration with other R packages makes it suitable for
comprehensive analyses of promotion patterns and career duration.

\textbf{R `mstate' Package}: Developed by Hein Putter and colleagues,
the `mstate' package extends survival analysis capabilities in R to
handle multi-state models and competing risks. According to CRAN
documentation, mstate ``contains functions for data preparation,
descriptives, hazard estimation and prediction with Aalen-Johansen or
simulation in competing risks and multi-state models''{[}10{]}. This
package is particularly valuable for military career analysis because it
can model complex transitions between different ranks and positions
while accounting for competing outcomes like voluntary separation,
involuntary separation, or retirement. The package facilitates the
creation of transition matrices, state occupation probabilities, and
expected length of stay in each state. For researchers studying officer
career trajectories, `mstate' provides tools to understand not just
whether officers are promoted but also the timing and sequencing of
career events within a unified analytical framework.

\textbf{Python `lifelines' Package}: The `lifelines' library is a
comprehensive Python package for survival analysis, offering
functionality similar to R's `survival' package but within the Python
ecosystem. As demonstrated in the documentation, `lifelines' provides
implementations of various survival models including Kaplan-Meier
estimators, Cox Proportional Hazard models, and parametric survival
models{[}11{]}. The package includes tools for visualizing survival
curves, comparing survival functions across groups, and performing
statistical tests to assess differences in survival distributions.
`lifelines' offers good integration with other Python data science
libraries such as pandas and scikit-learn, making it suitable for
incorporation into broader data analysis workflows. For military
promotion analysis, `lifelines' provides an accessible entry point for
Python users to model time-to-promotion and identify factors influencing
career advancement.

\textbf{SAS System}: The SAS System provides comprehensive capabilities
for survival analysis and logistic regression through procedures like
PROC LIFETEST, PROC LIFEREG, PROC PHREG, and PROC GENMOD. As
demonstrated in McAllaster's paper on forecasting Army officer
attrition, SAS is particularly powerful for analyzing large, complex
datasets with multiple cohorts{[}5{]}. The system's macro language
capabilities enable repeatable analyses across different subgroups,
which is valuable when examining promotion patterns across different
military branches or specialties. SAS also offers advanced capabilities
for handling complex survey designs, missing data, and longitudinal
analysis. While proprietary and more expensive than open-source
alternatives, SAS remains widely used in government and military
settings where its validated procedures, comprehensive documentation,
and technical support make it a preferred choice for mission-critical
analyses of personnel data.

\textbf{Custom Military Analysis Tools}: Beyond general-purpose
statistical software, various military organizations have developed
custom tools for analyzing promotion data. These specialized
applications typically integrate with military personnel databases and
incorporate service-specific business rules and policies. For example,
the Air Force Reserve Personnel Center uses specialized tools to
calculate ``promotion eligibility criteria'' and track the ``entire
promotion process for each board''{[}2{]}. Similarly, the Army's
Military Personnel Center publishes specialized analyses of ``specialty
alignment'' that inform promotion planning{[}18{]}. These custom tools,
while not generally available to academic researchers, represent
important components of the software ecosystem for Promotion Board
Analysis within military organizations. They often incorporate
service-specific methods for calculating promotion zones, eligibility,
and selection criteria that supplement the more general analytical
capabilities of commercial statistical packages.

\subsection{8. Example Study Design}\label{example-study-design}

\subsubsection{Key Variables}\label{key-variables}

\textbf{Officer Demographic Variables}: - Year group/commissioning
cohort - Gender - Race/ethnicity - Prior enlisted service status -
Commissioning source (Military Academy, ROTC, OCS) - Education level and
academic discipline

\textbf{Branch-Specific Developmental Indicators}: - \emph{Armor}:
Vehicle qualification levels, gunnery scores, combat deployments as
armor officer - \emph{Logistics}: Supply chain certification levels,
joint logistics experience, support operation assignments -
\emph{Aviation}: Flight hours by aircraft type, instructor pilot status,
aviation maintenance experience - \emph{Cyber}: Technical
certifications, civilian-equivalent qualifications, specialized training
completion

\textbf{Career Milestone Variables}: - Timing of key developmental (KD)
position completion - Duration in KD positions (24-36 months suggested
for ``Most Qualified'' status){[}1{]} - Command time at
company/battalion levels - Joint service assignment completion -
Advanced military education timing and performance

\textbf{Performance Metrics}: - Officer evaluation report scores and
narrative assessment - Awards and decorations (categorized by type and
significance) - Selection for specialized programs or fellowships -
Previous below-zone promotion selections

\textbf{Non-Cognitive Attributes}: - Leadership assessment scores - Peer
evaluations (if available) - Subordinate feedback measures - Mentorship
and talent development activities

\subsubsection{Sample \& Data Collection}\label{sample-data-collection}

The study will analyze a comprehensive dataset of U.S. Army officers
across the four branch divisions (Armor, Logistics, Aviation, and Cyber)
who were eligible for promotion to Lieutenant Colonel (O-5) during
fiscal years 2020-2025. This timeframe provides recent data while
allowing for analysis of promotion outcomes across multiple board
cycles.

Data collection will involve aggregating information from multiple
authoritative sources: 1. The Officer Record Brief (ORB) for demographic
and career history information 2. The Official Military Personnel File
(OMPF) for performance evaluations and awards 3. Branch-specific
databases for specialized qualifications and metrics 4. Army promotion
board results identifying selected and non-selected officers 5.
Post-board statistical reports indicating promotion rates by various
categories

For each officer in the dataset, we will collect comprehensive career
history data from commissioning through the promotion board
consideration date. This longitudinal approach allows for time-dependent
analysis of career trajectories and enables the identification of
critical developmental timing effects.

A stratified random sampling approach will ensure adequate
representation across all four branches, with oversampling of smaller
branches (particularly Cyber) to enable meaningful statistical analysis
within each branch category. The target sample size is 2,000 officers
per branch, for a total sample of 8,000 officers, providing sufficient
statistical power for both branch-specific and cross-branch analyses.

\subsubsection{Analysis Approach}\label{analysis-approach}

The analysis will employ a multi-method approach combining survival
analysis, logistic regression, and multi-state modeling to
comprehensively examine promotion patterns and career trajectories:

\textbf{Step 1: Descriptive Analysis} - Generate Kaplan-Meier survival
curves to visualize time-to-promotion patterns by branch, demographic
groups, and other key variables - Conduct preliminary comparative
analyses to identify potential differences in promotion rates across
branches and subgroups - Perform exploratory data analysis to identify
potential collinearity among predictors and inform subsequent modeling
decisions

\textbf{Step 2: Cox Proportional Hazards Modeling} - Develop
branch-specific Cox models predicting time-to-promotion using relevant
predictors - Test the proportional hazards assumption and implement
time-dependent covariates where appropriate - Examine interaction
effects between branch-specific development indicators and general
career factors

\textbf{Step 3: Logistic Regression Analysis} - Implement logistic
regression models predicting in-zone promotion selection (binary
outcome) - Compare model specifications to identify the most influential
predictors of promotion success - Calculate adjusted odds ratios to
quantify the impact of specific developmental experiences on promotion
likelihood

\textbf{Step 4: Multi-state Modeling} - Develop multi-state models
capturing transitions between career states (e.g., company command →
battalion XO → battalion command) - Calculate state occupation
probabilities and transition intensities between states - Identify
optimal career progression pathways associated with promotion success

\textbf{Step 5: Competing Risks Analysis} - Model competing outcomes
including promotion, voluntary separation, and involuntary separation -
Calculate cause-specific hazards for each outcome type - Identify
factors that differentially predict various career outcomes

\textbf{Step 6: Integration and Interpretation} - Synthesize findings
across analytical approaches to identify consistent patterns - Develop
branch-specific career progression models - Compare results to official
career progression guidance to identify alignment or misalignment

\subsubsection{Potential Findings}\label{potential-findings}

The analysis is expected to yield several categories of findings:

\textbf{Branch-Specific Developmental Patterns}: - The analysis may
reveal that Armor officers with combined arms experience and combat
deployments have significantly higher promotion rates than those without
such experience - For Logistics officers, joint logistics assignments
may emerge as more predictive of promotion than branch-specific
assignments - Aviation officers might show a non-linear relationship
between flight hours and promotion likelihood, with diminishing returns
beyond a certain threshold - Cyber officers with industry certifications
and operational experience might demonstrate higher promotion rates than
those focused solely on technical specialization

\textbf{Timing Effects}: - The analysis may identify optimal timing
windows for completing key developmental positions, potentially
differing by branch - Early completion of advanced military education
might show stronger positive effects for some branches than others -
Time spent in joint assignments might demonstrate differential returns
across branches

\textbf{Demographic Influences}: - The study might reveal whether
demographic factors continue to influence promotion outcomes after
controlling for performance and developmental variables - Different
commissioning sources might be associated with different career
trajectory patterns across branches

\textbf{Integration of Non-Cognitive Attributes}: - Leadership
assessments might emerge as stronger predictors of promotion in
command-centric branches like Armor - Technical problem-solving
abilities might show stronger associations with promotion success in
Cyber

\textbf{Cross-Branch Comparisons}: - The analysis may identify common
factors that predict promotion success across all branches, such as
joint service experience - Branch-specific factors might explain varying
proportions of promotion variance, potentially indicating differences in
promotion board emphasis

\textbf{Career Trajectory Typologies}: - The multi-state modeling
approach might identify distinct ``career trajectory types'' associated
with higher promotion likelihood - These typologies might differ
substantially across branches, suggesting different optimal paths to
success

\subsubsection{Potential Implications}\label{potential-implications}

\textbf{Career Management Policy}: - Findings could inform updates to
branch-specific career maps and developmental timelines published by the
Army - Results might suggest modifications to the ``Most Qualified Looks
Like'' guidance provided to promotion boards{[}1{]} - Analysis could
identify misalignments between stated career progression policies and
actual promotion outcomes

\textbf{Officer Development Programs}: - Results could guide
branch-specific adjustments to officer development programs and
assignment priorities - Findings might inform branch managers about
optimal timing and sequencing of key developmental experiences - The
analysis could highlight developmental gaps that should be addressed
through new training or assignment opportunities

\textbf{Promotion Board Guidance}: - Findings could inform more precise
guidance to promotion boards about how to evaluate branch-specific
accomplishments - Results might suggest adjustments to promotion board
composition to ensure appropriate expertise in evaluating specialized
experiences - Analysis could identify potential unintended consequences
of current promotion criteria

\textbf{Talent Management Initiatives}: - The identification of
successful career trajectory patterns could inform talent marketplace
algorithms and assignment prioritization - Findings might suggest
opportunities for cross-branch developmental assignments that enhance
promotion potential - Results could inform retention incentives
targeting officers with high-potential career trajectories

\textbf{Organizational Equity and Effectiveness}: - Analysis might
identify systemic barriers to advancement for certain demographic groups
that need to be addressed - Findings could help ensure that promotion
outcomes align with the Army's strategic talent needs, especially in
evolving domains like Cyber - Results might suggest adjustments to how
non-traditional career paths are evaluated by promotion boards

\subsection{Sources}\label{sources}

{[}1{]} {[}PDF{]} CMF 12 Career Progression Chart - Army.mil
https://api.army.mil/e2/c/downloads/2023/08/09/73f99ae3/cmf-12-board-products-8-aug-23.pdf\\
{[}2{]} Officer Promotion Boards - Air Reserve Personnel Center
https://www.arpc.afrc.af.mil/Services/Officer-Promotion-Boards/\\
{[}3{]} Understanding The Army Prootion Review Board (PRB)
https://www.mcmilitarylaw.com/understanding-the-army-promotion-review-board-prb/\\
{[}4{]} Career Management for the Army Reserve Officer \textbar{}
Article - Army.mil
https://www.army.mil/article/253844/career\_management\_for\_the\_army\_reserve\_officer\\
{[}5{]} {[}PDF{]} 1 Forecasting Army Officer Attrition with Logistic
Regression Using \ldots{}
http://www8.sas.com/scholars/05/PREVIOUS/1999/pdf/072.pdf\\
{[}6{]} {[}PDF{]} Analysis of the Survival Patterns of United States
Naval Officers - DTIC https://apps.dtic.mil/sti/tr/pdf/ADA432824.pdf\\
{[}7{]} {[}PDF{]} The Impact of Removing Demographic Indicators from
Military \ldots{} - IDA
https://www.ida.org/-/media/feature/publications/t/th/the-impact-of-removing-demographic-indicators-from-military-promotion-boards/3003198.ashx\\
{[}8{]} {[}PDF{]} The Use of Event-History-Analysis in Career Research
https://iisg.nl/publications/moderncareer-03.pdf\\
{[}9{]} CRAN: Package survival - R Project
https://cran.r-project.org/package=survival\\
{[}10{]} Package mstate - CRAN
https://cran.r-project.org/package=mstate\\
{[}11{]} Introduction to survival analysis --- lifelines 0.30.0
documentation
https://lifelines.readthedocs.io/en/latest/Survival\%20Analysis\%20intro.html\\
{[}12{]} Officer career progression question : r/army - Reddit
https://www.reddit.com/r/army/comments/afz6xl/officer\_career\_progression\_question/\\
{[}13{]} Headquarters RIO \textgreater{} Career Management
\textgreater{} Promotions
https://www.hqrio.afrc.af.mil/Career-Management/Promotions/\\
{[}14{]} Promotion Board Approval Process - MyNavyHR
https://www.mynavyhr.navy.mil/Career-Management/Boards/General-Board-Info/Promotion-Board-Approval-Process/\\
{[}15{]} {[}PDF{]} career progression and promotion board math
https://mccareer.org/wp-content/uploads/2019/06/3-career-progression-and-promotion-board-math.pdf\\
{[}16{]} Officer Promotions - HRC
https://www.hrc.army.mil/content/5789\\
{[}17{]} Promotion Boards - RAND Corporation
https://www.rand.org/paf/projects/dopma-ropma/promotion-and-appointments/promotion-boards.html\\
{[}18{]} {[}PDF{]} An Analysis of Selected Army Promotion Board Results.
- DTIC https://apps.dtic.mil/sti/tr/pdf/ADA070262.pdf\\




\end{document}
