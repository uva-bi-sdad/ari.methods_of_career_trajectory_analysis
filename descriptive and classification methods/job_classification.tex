% Options for packages loaded elsewhere
% Options for packages loaded elsewhere
\PassOptionsToPackage{unicode}{hyperref}
\PassOptionsToPackage{hyphens}{url}
\PassOptionsToPackage{dvipsnames,svgnames,x11names}{xcolor}
%
\documentclass[
  letterpaper,
  DIV=11,
  numbers=noendperiod]{scrartcl}
\usepackage{xcolor}
\usepackage{amsmath,amssymb}
\setcounter{secnumdepth}{-\maxdimen} % remove section numbering
\usepackage{iftex}
\ifPDFTeX
  \usepackage[T1]{fontenc}
  \usepackage[utf8]{inputenc}
  \usepackage{textcomp} % provide euro and other symbols
\else % if luatex or xetex
  \usepackage{unicode-math} % this also loads fontspec
  \defaultfontfeatures{Scale=MatchLowercase}
  \defaultfontfeatures[\rmfamily]{Ligatures=TeX,Scale=1}
\fi
\usepackage{lmodern}
\ifPDFTeX\else
  % xetex/luatex font selection
\fi
% Use upquote if available, for straight quotes in verbatim environments
\IfFileExists{upquote.sty}{\usepackage{upquote}}{}
\IfFileExists{microtype.sty}{% use microtype if available
  \usepackage[]{microtype}
  \UseMicrotypeSet[protrusion]{basicmath} % disable protrusion for tt fonts
}{}
\makeatletter
\@ifundefined{KOMAClassName}{% if non-KOMA class
  \IfFileExists{parskip.sty}{%
    \usepackage{parskip}
  }{% else
    \setlength{\parindent}{0pt}
    \setlength{\parskip}{6pt plus 2pt minus 1pt}}
}{% if KOMA class
  \KOMAoptions{parskip=half}}
\makeatother
% Make \paragraph and \subparagraph free-standing
\makeatletter
\ifx\paragraph\undefined\else
  \let\oldparagraph\paragraph
  \renewcommand{\paragraph}{
    \@ifstar
      \xxxParagraphStar
      \xxxParagraphNoStar
  }
  \newcommand{\xxxParagraphStar}[1]{\oldparagraph*{#1}\mbox{}}
  \newcommand{\xxxParagraphNoStar}[1]{\oldparagraph{#1}\mbox{}}
\fi
\ifx\subparagraph\undefined\else
  \let\oldsubparagraph\subparagraph
  \renewcommand{\subparagraph}{
    \@ifstar
      \xxxSubParagraphStar
      \xxxSubParagraphNoStar
  }
  \newcommand{\xxxSubParagraphStar}[1]{\oldsubparagraph*{#1}\mbox{}}
  \newcommand{\xxxSubParagraphNoStar}[1]{\oldsubparagraph{#1}\mbox{}}
\fi
\makeatother


\usepackage{longtable,booktabs,array}
\usepackage{calc} % for calculating minipage widths
% Correct order of tables after \paragraph or \subparagraph
\usepackage{etoolbox}
\makeatletter
\patchcmd\longtable{\par}{\if@noskipsec\mbox{}\fi\par}{}{}
\makeatother
% Allow footnotes in longtable head/foot
\IfFileExists{footnotehyper.sty}{\usepackage{footnotehyper}}{\usepackage{footnote}}
\makesavenoteenv{longtable}
\usepackage{graphicx}
\makeatletter
\newsavebox\pandoc@box
\newcommand*\pandocbounded[1]{% scales image to fit in text height/width
  \sbox\pandoc@box{#1}%
  \Gscale@div\@tempa{\textheight}{\dimexpr\ht\pandoc@box+\dp\pandoc@box\relax}%
  \Gscale@div\@tempb{\linewidth}{\wd\pandoc@box}%
  \ifdim\@tempb\p@<\@tempa\p@\let\@tempa\@tempb\fi% select the smaller of both
  \ifdim\@tempa\p@<\p@\scalebox{\@tempa}{\usebox\pandoc@box}%
  \else\usebox{\pandoc@box}%
  \fi%
}
% Set default figure placement to htbp
\def\fps@figure{htbp}
\makeatother





\setlength{\emergencystretch}{3em} % prevent overfull lines

\providecommand{\tightlist}{%
  \setlength{\itemsep}{0pt}\setlength{\parskip}{0pt}}



 


\KOMAoption{captions}{tableheading}
\makeatletter
\@ifpackageloaded{caption}{}{\usepackage{caption}}
\AtBeginDocument{%
\ifdefined\contentsname
  \renewcommand*\contentsname{Table of contents}
\else
  \newcommand\contentsname{Table of contents}
\fi
\ifdefined\listfigurename
  \renewcommand*\listfigurename{List of Figures}
\else
  \newcommand\listfigurename{List of Figures}
\fi
\ifdefined\listtablename
  \renewcommand*\listtablename{List of Tables}
\else
  \newcommand\listtablename{List of Tables}
\fi
\ifdefined\figurename
  \renewcommand*\figurename{Figure}
\else
  \newcommand\figurename{Figure}
\fi
\ifdefined\tablename
  \renewcommand*\tablename{Table}
\else
  \newcommand\tablename{Table}
\fi
}
\@ifpackageloaded{float}{}{\usepackage{float}}
\floatstyle{ruled}
\@ifundefined{c@chapter}{\newfloat{codelisting}{h}{lop}}{\newfloat{codelisting}{h}{lop}[chapter]}
\floatname{codelisting}{Listing}
\newcommand*\listoflistings{\listof{codelisting}{List of Listings}}
\makeatother
\makeatletter
\makeatother
\makeatletter
\@ifpackageloaded{caption}{}{\usepackage{caption}}
\@ifpackageloaded{subcaption}{}{\usepackage{subcaption}}
\makeatother
\usepackage{bookmark}
\IfFileExists{xurl.sty}{\usepackage{xurl}}{} % add URL line breaks if available
\urlstyle{same}
\hypersetup{
  pdftitle={Job Classification as a Method for the Analysis of Career Trajectories},
  colorlinks=true,
  linkcolor={blue},
  filecolor={Maroon},
  citecolor={Blue},
  urlcolor={Blue},
  pdfcreator={LaTeX via pandoc}}


\title{Job Classification as a Method for the Analysis of Career
Trajectories}
\author{}
\date{}
\begin{document}
\maketitle


Job classification forms the foundation for analyzing and understanding
career trajectories across organizations. This comprehensive framework
enables researchers and practitioners to systematically categorize
positions based on their relative value, required competencies, and
organizational impact, providing valuable insights into career pathways
and progression patterns.

\subsection{1. Approach Description \&
Goal}\label{approach-description-goal}

Job classification is a systematic process of categorizing jobs into
different ranks based on their relative worth to an organization. It
involves evaluating positions according to standardized criteria rather
than the individuals occupying them{[}1{]}. This approach serves
multiple purposes: establishing fair and equitable compensation
structures, creating clear organizational hierarchies, defining career
progression pathways, enabling succession planning, and supporting
consistent personnel decisions{[}1{]}{[}12{]}. When applied to career
trajectory analysis, job classification provides a structured framework
for understanding how individuals move between positions over time,
identifying common pathways to success, and recognizing critical
transition points in career development. This methodical approach allows
organizations to develop evidence-based talent management strategies and
helps individuals make informed career decisions based on established
progression patterns.

\subsection{2. Critical Variables}\label{critical-variables}

Job classification systems typically incorporate several key variable
categories that serve as inputs to the evaluation process:

\subsubsection{Knowledge and Skills}\label{knowledge-and-skills}

This dimension encompasses the education, experience, specialized
training, technical capabilities, and expertise required to perform a
job effectively{[}5{]}{[}6{]}. The Hay System, a popular classification
method, refers to this as ``Know-How,'' defined as the ``sum total of
every kind of knowledge and skill, however acquired, needed for
acceptable job performance''{[}14{]}. This includes practical
procedures, specialized techniques, and professional knowledge.

\subsubsection{Problem-Solving
Complexity}\label{problem-solving-complexity}

This category evaluates the nature and complexity of challenges
typically encountered in a position and the level of analytical,
creative, and strategic thinking required to address them{[}5{]}{[}6{]}.
The Hay System measures this as ``the amount and nature of thinking
required in the job in the form of analyzing, reasoning, evaluating,
creating, using judgment, forming hypotheses, drawing inferences, and
arriving at conclusions''{[}14{]}.

\subsubsection{Accountability and
Responsibility}\label{accountability-and-responsibility}

This dimension assesses the impact of the position on organizational
outcomes, including resource management authority, decision-making
scope, and the consequences of errors{[}4{]}{[}5{]}. Variables include
budget responsibility, span of control, signature authority, and policy
influence{[}3{]}.

\subsubsection{Organizational Context}\label{organizational-context}

This category considers the position's place within the broader
organizational structure, including reporting relationships, peer
positions, stakeholder interactions, and governance
responsibilities{[}3{]}. It examines how the role interfaces with
internal and external constituencies.

\subsubsection{Working Conditions}\label{working-conditions}

This dimension evaluates the physical environment, stress factors,
hazards, and other contextual elements that affect job performance and
requirements{[}6{]}. This may include facility responsibility, equipment
operation, and unique environmental challenges.

\subsection{3. Key Overviews}\label{key-overviews}

\subsubsection{Occupational Classifications: A Machine Learning Approach
(Ikudo, Lane, Staudt, and Weinberg,
2018)}\label{occupational-classifications-a-machine-learning-approach-ikudo-lane-staudt-and-weinberg-2018}

This National Bureau of Economic Research working paper examines the
potential for using machine learning approaches to automatically
classify job titles into standardized occupational categories. The
authors develop a hierarchical occupation classification system based on
individuals' relationships to universities (faculty, students, staff)
and their functions, then test various algorithms to automatically
classify job titles. Their findings indicate that random forest
classifiers perform best, achieving accuracy rates around 90\% for
common job titles. The researchers conclude that machine learning
approaches show promise for efficiently classifying the many job titles
that have relatively few people in them, potentially yielding
substantial cost savings while maintaining classification accuracy.
However, they emphasize that entirely algorithmic approaches remain
insufficient, and human oversight continues to be necessary{[}7{]}.

\subsubsection{Job Classification: A Review on Data, Features, and
Methods
(2021)}\label{job-classification-a-review-on-data-features-and-methods-2021}

This comprehensive review examines job classification as both a process
to categorize positions and a system to recommend jobs to candidates
based on specific criteria. The paper analyzes various data types used
in classification, including nominal and numerical data, and explores
key features that drive effective job classification, such as academic
performance, technical skills, and demographic factors. The authors
review multiple classification methods, ranging from traditional
approaches to advanced machine learning techniques like Naive Bayes
classifiers. The study highlights that CGPA (Cumulative Grade Point
Average) emerges as one of the most frequently used and significant
features in job classification and recommendation systems, demonstrating
the continued importance of academic performance in professional
classification schemes despite the growing emphasis on practical skills
and experience{[}8{]}.

\subsubsection{The Hay Group Guide Chart-Profile Method of Job
Evaluation}\label{the-hay-group-guide-chart-profile-method-of-job-evaluation}

This foundational document outlines the Hay System, one of the most
widely used job classification methods in North America and Europe. The
system evaluates positions using three key factors: Know-how (skills and
expertise), Problem-solving (complexity of thinking required), and
Accountability (impact on organizational outcomes). The guide explains
the unique numerical scale employed by the method, which follows a
geometric rather than arithmetic pattern to properly represent
proportional differences in job size. The approach is based on Weber's
Law, which states that perceptions of differences between objects are
relative to their magnitude rather than absolute. This ``step
difference'' principle, set at 15\% in the Hay System, ensures
consistent relativity in judgments regardless of where they fall on the
classification scale, providing a scientifically grounded basis for
establishing the relative worth of different positions within an
organization{[}16{]}.

\subsubsection{Job Level Classification: The Complete Guide
(2025)}\label{job-level-classification-the-complete-guide-2025}

This comprehensive guide provides a practical framework for implementing
job level classification systems in modern organizations. It outlines
typical classification levels from entry-level to executive positions
and explains how job families group related roles that share similar
functions, responsibilities, and required skills. The guide describes
the development process for a classification system, beginning with
thorough job analysis and progressing through defining job families,
establishing competency models, and creating career progression
pathways. It emphasizes the importance of considering multiple criteria
when classifying positions, including industry context, functional
responsibilities, required competencies, organizational level, and
compensation structures. The document serves as a practical roadmap for
organizations seeking to implement or refine their job classification
systems to support strategic talent management objectives{[}4{]}.

\subsection{4. Mathematical Approach}\label{mathematical-approach}

Job classification employs several mathematical approaches, with
point-factor methods being among the most common. These systems assign
numerical values to job attributes according to predetermined scales,
enabling systematic comparison across positions.

\subsubsection{Point-Factor Method}\label{point-factor-method}

The fundamental formula for the Point-Factor Method is:

Total Job Points = Sum of points for (Skills + Responsibilities + Effort
+ Working Conditions)

Each factor is further broken down into subfactors with specific point
scales. For example:

Skills = Education points + Experience points + Technical expertise
points Responsibilities = Supervision points + Financial impact points +
Decision-making points Effort = Mental demand points + Physical demand
points Working Conditions = Environment points + Hazard points

A job with higher total points ranks higher in the classification
hierarchy and typically commands higher compensation{[}6{]}.

\subsubsection{Hay Method Geometric
Scale}\label{hay-method-geometric-scale}

The Hay Method uses a geometric progression rather than an arithmetic
one, with points values following a percentage progression (e.g., 100,
115, 132, 152). Each step represents a 15\% increase, based on Weber's
Law:

P₂ = P₁ × (1 + S)

Where: - P₁ is the initial point value - P₂ is the next point value - S
is the step difference (0.15 or 15\%)

This ensures that proportional differences between job levels remain
consistent regardless of their absolute size. For example, an increase
from 100 to 115 points represents the same proportional change as an
increase from 1000 to 1150 points{[}16{]}.

\subsubsection{Machine Learning
Classification}\label{machine-learning-classification}

Modern job classification increasingly employs machine learning
algorithms, particularly for automatic classification of job titles. The
general approach follows:

\begin{enumerate}
\def\labelenumi{\arabic{enumi}.}
\tightlist
\item
  Feature extraction from job titles/descriptions (using techniques like
  TF-IDF)
\item
  Model training using labeled data (jobs with known classifications)
\item
  Classification of new job titles using the trained model
\end{enumerate}

Random forest classifiers have shown particular promise, with the
mathematical framework involving: - Creation of multiple decision trees
using bootstrap sampling - Feature selection at each node using a random
subset of features - Classification by majority vote across all trees in
the forest

The accuracy of such models can be expressed as:

Accuracy = (True Positives + True Negatives) / Total Classifications

Studies have shown accuracy rates around 70-90\% for automated job
classification systems{[}7{]}{[}8{]}.

\subsection{5. Example Applications}\label{example-applications}

\subsubsection{JobHop: A Large-Scale Dataset of Career Trajectories
(2022)}\label{jobhop-a-large-scale-dataset-of-career-trajectories-2022}

This study introduces a comprehensive dataset of real-world occupation
transitions that enables detailed analysis of career trajectories and
labor market dynamics. The researchers leveraged the standardized ESCO
(European Skills, Competences, Qualifications and Occupations) taxonomy
to analyze career mobility patterns, job stability across sectors, and
the influence of education and career breaks on occupational
transitions. The study explores multiple dimensions of the Flemish labor
market, including the impact of career interruptions on subsequent job
opportunities, patterns of mobility within and across sectors, and the
quantifiable influence of university degrees on career pathways. The
structured classification approach allows for international comparisons
and integration with other labor market data sources, demonstrating how
job classification frameworks can serve as a foundation for
sophisticated career trajectory analysis. The researchers highlight the
dataset's potential applications for career path prediction, skill gap
analysis, workforce planning, and data-driven career guidance
tools{[}11{]}.

\subsubsection{Occupational Classifications: A Machine Learning
Application to Job Titles (Ikudo et al.,
2018)}\label{occupational-classifications-a-machine-learning-application-to-job-titles-ikudo-et-al.-2018}

This empirical study applies machine learning techniques to
automatically classify job titles from university human resources
records into standardized occupational categories. The researchers
developed a hierarchical classification system based on individuals'
relationships to universities and their functions, then tested various
algorithms including random forests, support vector machines, and neural
networks. Their analysis found that random forest classifiers achieved
the highest accuracy rates (approximately 90\%) for common job titles.
The authors demonstrate that machine learning approaches can
significantly reduce the costs associated with manual classification
while maintaining acceptable accuracy levels for most positions.
However, they note that completely automated approaches remain
insufficient for certain complex or unique job titles, suggesting a
hybrid model that combines algorithmic classification with human review
for challenging cases. The study provides concrete evidence of how
computational approaches can enhance traditional job classification
methods in large-scale administrative datasets{[}7{]}.

\subsubsection{A Machine Learning Approach to Job Seeker Profile
Classification Based on Social Media
Data}\label{a-machine-learning-approach-to-job-seeker-profile-classification-based-on-social-media-data}

This innovative study applies job classification techniques to social
media data to automatically categorize job seekers based on their online
presence. The researchers employed a Naive Bayes classifier algorithm in
combination with W-IDF (Weighted-Inverse Document Frequency) weighting
to analyze Twitter data and classify users' personality traits and
professional orientations. The classification system categorized
individuals according to their expressed interests, communication
patterns, and content sharing behaviors, creating profiles that could be
matched to suitable job categories. The study achieved an average
accuracy of over 90\% with certain division ratios, demonstrating the
potential for applying job classification frameworks to non-traditional
data sources. This application shows how classification techniques can
be extended beyond formal job titles to encompass broader aspects of
career development, including personality traits and behavioral
tendencies that influence career trajectories{[}8{]}.

\subsubsection{Job Recommender System Using Naïve Bayes
Classification}\label{job-recommender-system-using-nauxefve-bayes-classification}

This applied research project developed a job recommendation system that
classifies and suggests suitable positions based on candidates' profiles
and qualifications. The researchers employed a Naive Bayes classifier to
analyze multiple features including academic performance, educational
background, professional experience, publications, projects, and grants.
The system achieved high accuracy rates (91.33-92.74\% depending on
division ratio) in matching candidates to appropriate positions based on
their profiles. The study demonstrates how classification algorithms can
be applied to career guidance by identifying patterns in successful
career paths and recommending similar trajectories to individuals with
corresponding qualifications and attributes. This application
illustrates how job classification frameworks can transition from
descriptive tools to predictive models that actively shape career
development decisions{[}8{]}.

\subsection{6. Critiques}\label{critiques}

Despite its widespread adoption, job classification as an analytical
approach faces several significant limitations and criticisms:

\subsubsection{Subjectivity and Bias}\label{subjectivity-and-bias}

Despite attempts to create objective systems, job classification
inherently involves subjective judgments that may introduce bias. The
interpretation of job descriptions, the weighting of factors, and the
evaluation of position requirements all contain elements of subjectivity
that can undermine the system's objectivity{[}16{]}. This subjectivity
may perpetuate existing biases in how different types of work are
valued, potentially disadvantaging certain job categories or worker
demographics.

\subsubsection{Resource Intensity}\label{resource-intensity}

Traditional job classification methods require substantial investments
of time, expertise, and organizational resources. Manual classification
processes are labor-intensive and expensive, particularly for large
organizations with numerous positions{[}7{]}. Even when using automated
approaches, significant upfront investment is needed to develop and
validate classification frameworks, train evaluators, and maintain the
system over time.

\subsubsection{Static Nature vs.~Dynamic
Workplace}\label{static-nature-vs.-dynamic-workplace}

Job classification systems often struggle to keep pace with rapidly
evolving job roles in dynamic work environments. The standardized
frameworks may not adequately capture emerging skills, hybrid roles, or
innovative working arrangements that characterize modern
workplaces{[}4{]}. This rigidity can result in outdated classifications
that fail to reflect the actual work being performed, limiting the
approach's effectiveness for analyzing contemporary career trajectories.

\subsubsection{Algorithmic Limitations}\label{algorithmic-limitations}

While machine learning approaches offer promising efficiency gains, they
still face significant limitations. As noted by Ikudo et al., even
advanced algorithms achieve accuracy rates that necessitate human
oversight, particularly for uncommon or specialized job titles{[}7{]}.
The ``black box'' nature of some algorithms also raises questions about
transparency and fairness in classification decisions, potentially
introducing new forms of bias while attempting to eliminate traditional
ones.

\subsubsection{Oversimplification of Complex
Work}\label{oversimplification-of-complex-work}

Critics argue that job classification systems inevitably reduce the
complexity and nuance of actual work to standardized categories and
numerical values. This reductionist approach may fail to capture
important qualitative aspects of jobs that influence career development
but resist easy quantification{[}8{]}. The focus on formal job
attributes may also overlook informal roles, tacit knowledge, and
relationship dynamics that significantly impact career trajectories.

\subsection{7. Software}\label{software}

\subsubsection{ONETr (R Package)}\label{onetr-r-package}

ONETr is an R package that facilitates interaction with the O\emph{NET
API, providing researchers with programmatic access to the O}NET
database---America's primary source of occupational information. The
package enables users to search occupational data based on keywords or
O\emph{NET-SOC codes and parse the XML output into structured list
objects for analysis. Users must register for an account with O}NET Web
Services to obtain the necessary API credentials. ONETr includes
functions to extract specific occupational data components such as
skills, abilities, work activities, and job zones, making it
particularly valuable for researchers analyzing career pathways and
occupational requirements. The package supports both exploratory
analysis of occupational characteristics and systematic comparison
across job categories, providing a foundation for evidence-based career
trajectory research{[}9{]}.

\subsubsection{labourR (R Package)}\label{labourr-r-package}

labourR is a specialized R package that performs occupational coding for
multilingual free-form text using the ESCO hierarchical classification
model. Initially developed to analyze work experience histories from
Curricula Vitae, the package generates term frequency-inverse document
frequency statistics for terms in the ESCO occupations corpus and scores
input queries based on matched terms. Classification is performed
through a plurality vote in the corresponding hierarchical level of the
ESCO ontologies with the highest score. The package includes the
complete ESCO corpus and ESCO-to-ISCO mappings, supports multilingual
text input, and provides fully vectorized, memory-efficient
computations. labourR is particularly valuable for researchers analyzing
career trajectories across international contexts or working with
non-standardized job title data from diverse sources{[}10{]}.

\subsubsection{O*NET Web Services}\label{onet-web-services}

O\emph{NET Web Services provides programmatic access to the O}NET
database, containing standardized and occupation-specific descriptors
for hundreds of occupations. This comprehensive platform includes
detailed information on the knowledge, skills, abilities, work
activities, and work contexts associated with each occupation in the
O\emph{NET-SOC taxonomy. The system is regularly updated through surveys
of workers across occupations, ensuring currency and relevance. O}NET
Web Services enables researchers to access this rich occupational data
programmatically, supporting sophisticated analyses of career
requirements, progression patterns, and occupational transitions. The
service requires registration and API credentials but provides free
access to this authoritative occupational information, making it an
essential resource for career trajectory research{[}2{]}{[}13{]}.

\subsubsection{Deel Engage}\label{deel-engage}

Deel Engage includes career leveling functionality as part of its
broader HR management platform, specifically designed to implement and
maintain job classification systems. The software supports the creation
of job leveling matrices that evaluate and assign levels to different
roles within an organization based on customizable criteria such as
required competencies, responsibilities, and impact. The platform
enables organizations to define clear job titles, develop consistent
salary structures, and establish visible career progression paths for
employees. Deel Engage integrates job classification with broader talent
management functions, allowing organizations to connect classification
frameworks to performance management, compensation, and development
planning. The system supports both traditional classification approaches
and more flexible models adapted to evolving workplace
structures{[}4{]}.

\subsection{8. Example Study Design: Job Classification Analysis of U.S.
Army Officer Career
Trajectories}\label{example-study-design-job-classification-analysis-of-u.s.-army-officer-career-trajectories}

\subsubsection{Key Variables}\label{key-variables}

This study will utilize a multi-dimensional job classification framework
tailored to military contexts, incorporating the following key
variables:

\textbf{Functional Competencies}: Branch-specific technical skills and
knowledge (e.g., armor tactical proficiency, logistics planning
capabilities, aviation operational expertise, cyber defense
competencies) that define core performance requirements within each
division.

\textbf{Leadership Dimensions}: Command responsibilities, scope of
authority, personnel management requirements, and strategic influence
measured across organizational levels from platoon to division.

\textbf{Problem-Solving Complexity}: Tactical, operational, and
strategic decision-making requirements, including time pressure,
information ambiguity, and consequence severity typical of different
positions.

\textbf{Organizational Impact}: Measurable outcomes attributable to the
position, including resource management scope, mission criticality, and
institutional influence.

\textbf{Career Velocity Indicators}: Time-in-grade metrics, promotion
rates relative to peers, selection for prestigious assignments, and
advanced educational achievements.

\textbf{Non-Cognitive Attributes}: Leadership adaptability, stress
resilience, cultural intelligence, and other psychological factors that
influence performance across different military contexts.

\subsubsection{Sample \& Data Collection}\label{sample-data-collection}

The study will employ a stratified longitudinal sampling approach that
balances representation across the four branch divisions (Armor,
Logistics, Aviation, and Cyber) while ensuring sufficient career span
coverage:

\textbf{Sample Structure}: 400 officers (100 per branch) with complete
career records spanning at least 15 years of service, stratified by
commissioning source (USMA, ROTC, OCS) and entry year cohorts
(2005-2010).

\textbf{Data Sources}: Integrated personnel records including Officer
Evaluation Reports (OERs), promotion board results, assignment
histories, training records, and educational achievements from Army
Human Resources Command databases.

\textbf{Supplementary Data Collection}: Semi-structured interviews with
40 senior officers (10 per branch) to validate classification frameworks
and provide context for quantitative findings.

\textbf{Temporal Coverage}: Historical data spanning 2005-2025 to
capture complete career trajectories, including critical transition
points between company and field-grade ranks.

\textbf{Classification Reference Data}: Expert panel assessments of
position classifications for standardization across branches, including
Hay Method evaluations of knowledge requirements, problem-solving
complexity, and accountability levels.

\subsubsection{Analysis Approach}\label{analysis-approach}

The analysis will implement a multi-method job classification strategy
that combines traditional evaluation methods with advanced analytical
techniques:

\textbf{Position Classification Development}: Create a standardized
Army-specific job classification framework using modified Hay Method
criteria, with branch-specific weighting of factors relevant to each
division's unique requirements.

\textbf{Career Pattern Identification}: Apply sequence analysis
techniques to identify common career trajectories within and across
branches, mapping the progression through classified position types.

\textbf{Transition Analysis}: Employ Markov models to calculate
transition probabilities between job classifications, identifying
critical junctures and career acceleration points.

\textbf{Success Predictor Modeling}: Develop random forest
classification models to identify early career indicators that predict
successful progression to senior ranks, with separate models for each
branch division.

\textbf{Cross-Branch Comparison}: Conduct comparative analysis of
classification structures and career velocity across branches,
controlling for cohort effects and external factors.

\textbf{Validation Testing}: Apply the classification framework to a
hold-out sample of 100 officers to validate prediction accuracy and
refine the model.

\subsubsection{Potential Findings}\label{potential-findings}

This job classification analysis is expected to yield several
significant insights into Army officer career trajectories:

\textbf{Differentiated Success Pathways}: The analysis will likely
reveal distinct optimal career sequences for each branch division, with
significant variation in the timing and nature of key developmental
positions that predict successful progression to senior ranks.

\textbf{Critical Classification Transitions}: We expect to identify
specific job classification levels where career velocity diverges most
significantly between high-performing and average officers, potentially
highlighting previously unrecognized career decision points.

\textbf{Branch-Specific Competency Weightings}: The classification
framework will likely demonstrate that the relative importance of
technical expertise versus leadership competencies varies substantially
across branches, with Cyber placing higher value on technical
specialization compared to more leadership-weighted branches like Armor.

\textbf{Cross-Pollination Effects}: The analysis may reveal the career
impact of cross-branch assignments, potentially showing that officers
who temporarily serve in positions classified within other branches gain
valuable perspective that accelerates their subsequent advancement.

\textbf{Non-Cognitive Attribute Influence}: We anticipate finding that
certain non-cognitive attributes correlate strongly with successful
navigation of specific job classification transitions, with different
attributes being more predictive in different branch contexts.

\subsubsection{Potential Implications}\label{potential-implications}

The findings from this job classification study would have several
significant implications for Army talent management and officer
development:

\textbf{Tailored Development Programs}: Results could inform the
creation of branch-specific career development roadmaps that highlight
optimal timing and sequencing of key developmental positions, enabling
more strategic career planning for both officers and their mentors.

\textbf{Classification-Based Talent Management}: The Army could
implement a more sophisticated talent management approach that matches
officers to positions based not only on technical qualifications but
also on demonstrated success patterns in navigating similar
classification transitions.

\textbf{Recruitment Refinement}: Findings about the predictive value of
early career indicators could refine recruitment and commissioning
standards, potentially with branch-specific emphasis on attributes most
predictive of success in each domain.

\textbf{Training Investment Optimization}: Understanding which
competencies drive successful transitions between classification levels
would allow more targeted investment in developing those specific
capabilities at the appropriate career stages.

\textbf{Cross-Branch Development Strategy}: Insights about the value of
cross-branch experience could inform a more deliberate approach to
developmental assignments outside an officer's primary branch,
potentially creating a more flexible and adaptable officer corps.

\textbf{Promotion System Refinement}: Classification-based understanding
of career trajectories could lead to reforms in the promotion system to
better recognize and reward developmental diversity while maintaining
branch-specific expertise.

\subsection{Sources}\label{sources}

{[}1{]} Job Classification: A Practitioner's Guide - AIHR
https://www.aihr.com/blog/job-classification/\\
{[}2{]} About O\emph{NET at O}NET Resource Center
https://www.onetcenter.org/overview.html\\
{[}3{]} Classification Variables \textbar{} Department Resources -
William \& Mary
https://www.wm.edu/offices/uhr/departments/hiring-officials-business-partners/classification-compensation/class-comp-guide/class-variables/\\
{[}4{]} The Complete Guide to Job Level Classification: Examples,
Criteria \ldots{} https://www.deel.com/blog/job-level-classification/\\
{[}5{]} Hay System 101 - Job Evaluation Method {[}2024 Updated{]}
https://www.peoplebox.ai/blog/hay-system/\\
{[}6{]} Job Evaluation Using Point Factor \textbar{} Synergogy
https://synergogy.com/job-evaluation-using-point-factor/\\
{[}7{]} {[}PDF{]} Occupational Classifications: A Machine Learning
Approach
https://www.nber.org/system/files/working\_papers/w24951/w24951.pdf\\
{[}8{]} {[}PDF{]} Job classification: A Review on Data, Features, and
Methods - JEESR
https://jeesr.uitm.edu.my/v1/JEESR/Vol.21/article12.pdf\\
{[}9{]} eknud/ONETr: A small R package for interacting with the
O\emph{NET API. https://github.com/eknud/ONETr\\
{[}10{]} Introduction to labourR - CRAN
https://cran.r-project.org/web/packages/labourR/vignettes/labourR\_blogpost.html\\
{[}11{]} JobHop: A Large-Scale Dataset of Career Trajectories - arXiv
https://arxiv.org/html/2505.07653v1\\
{[}12{]} Understanding Job Classification \textbar{} BambooHR
https://www.bamboohr.com/resources/hr-glossary/job-classification\\
{[}13{]} O}NET OnLine Help: O*NET Overview
https://www.onetonline.org/help/onet/\\
{[}14{]} Hay evaluation method \textbar{} Human Resources \textbar{}
University of Waterloo
https://uwaterloo.ca/human-resources/support-managers/compensation-information-managers/hay-evaluation-method\\
{[}15{]} How Does Job Classification Work? (With Definition and Types)
https://www.indeed.com/career-advice/career-development/how-does-job-classification-work\\
{[}16{]} {[}PDF{]} The Hay Group Guide Chart-ProfileSM method of job
evaluation
https://professionals.lincolnshire.gov.uk/downloads/file/576/hay-group-je-method\\
{[}17{]} Determining Job Classification - UCSB Human Resources
https://www.hr.ucsb.edu/hr-units/compensation/determining-job-classification\\
{[}18{]} {[}PDF{]} Internal Equity Workshop -- Hay Group Work
Measurement Process
https://lincolnhr.org/wp-content/uploads/2009/12/Internal-Equity-Workshop-Hay-Group.pdf\\
{[}19{]} Job Classification: Aligning Roles and Compensation
https://www.cpshr.us/blog-article/job-classification/\\
{[}20{]} Job Classification: HR Terms Explained \textbar{} Pelago
https://www.pelagohealth.com/resources/hr-glossary/job-classification/\\




\end{document}
