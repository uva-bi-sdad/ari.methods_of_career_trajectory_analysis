\documentclass[main.tex]{subfiles}
\begin{document}

System Dynamics Modeling represents a methodology for understanding and analyzing the nonlinear behavior of complex systems over time through the use of stocks, flows, internal feedback loops, and time delays[2][15]. Originally developed by Jay W. Forrester at MIT in the 1950s, this approach provides a structured framework for modeling dynamic complexity in social, managerial, and organizational systems[4][15]. When applied to career trajectory analysis, System Dynamics captures the interconnected factors that influence professional advancement, retention decisions, and organizational workforce flows, enabling researchers to understand how individual career decisions and organizational policies interact to produce emergent patterns of career progression over time[7][8].

\subsubsection{Approach Description \& Goal}

System Dynamics is fundamentally designed as "a perspective and set of conceptual tools that enable us to understand the structure and dynamics of complex systems," serving purposes beyond mere simulation methodology[3]. The approach aims to help decision-makers "make better decisions when confronted with complex, dynamic systems" by providing methods and tools to model and analyze dynamic systems through simulation modeling based on feedback systems theory[2]. In the context of career trajectory analysis, System Dynamics seeks to understand how the complex interplay of individual characteristics, organizational policies, and external factors creates patterns of career progression, retention, and advancement over time.

The methodology's primary goal is to reveal the underlying causal structure that drives system behavior, moving beyond simple correlation to understand the feedback mechanisms that sustain or change career patterns[10]. System Dynamics recognizes that career trajectories emerge from the interaction of multiple feedback loops, where individual decisions affect organizational outcomes, which in turn influence future individual choices and organizational policies[15]. This approach is particularly valuable for analyzing career systems because it can capture the long-term consequences of policy interventions and help identify leverage points where small changes might produce significant improvements in career outcomes.

\subsubsection{Critical Variables}

System Dynamics models for career trajectory analysis typically incorporate several categories of variables that capture both individual and systemic factors. **Stock variables** represent accumulations of people at different career stages, such as the number of officers at each rank or years of service[11]. **Flow variables** represent the rates of change between these stocks, including promotion rates, retention rates, and recruitment flows[11]. **Auxiliary variables** capture factors that influence these flows, such as performance evaluations, educational achievements, and assignment experiences.

Individual-level variables commonly include demographic characteristics (age, gender, race, marital status), cognitive abilities (test scores, educational background), performance measures (evaluation ratings, assessment scores), and career experiences (key developmental assignments, broadening experiences, joint service credit)[1][8]. Organizational variables encompass policy parameters such as promotion timelines, selection criteria, force structure requirements, and resource allocation decisions. Environmental variables capture external factors like economic conditions, competing career opportunities, and societal changes that affect career decisions.

The model structure also requires **feedback variables** that capture how outcomes influence future inputs, such as how promotion success rates affect recruitment quality or how retention challenges influence promotion opportunities[7]. **Time delay variables** represent the lag between actions and their consequences, such as the time between educational investments and career advancement or between policy changes and their observable effects on career outcomes[11].

\subsubsection{Key Overviews}

\subsubsubsection{Sterman's Business Dynamics}

John Sterman's "Business Dynamics: Systems Thinking and Modeling for a Complex World" represents a comprehensive foundation for understanding System Dynamics methodology[3]. Sterman presents System Dynamics as more than just a simulation tool, emphasizing its value as a conceptual framework for understanding complex systems characterized by dynamic complexity, limited information, and inadequate mental models. The book provides extensive guidance on model building, from conceptualization through implementation, with particular emphasis on making models transparent and accessible to users. Sterman argues that simulation is essential because closed solutions for realistic nonlinear representations of complex real-world systems do not exist, and qualitative discussions alone cannot lead to reliable understanding. The work draws examples from diverse domains including organizational behavior, economic phenomena, and social systems, demonstrating the broad applicability of System Dynamics to understanding human systems like career progression.

\subsubsubsection{Forrester's Industrial Dynamics}

Jay Forrester's seminal work "Industrial Dynamics" established the foundational principles of System Dynamics and remains highly relevant for understanding organizational systems[6]. Originally published in 1961, the book demonstrates how systemic problems in organizations arise from internal structure rather than external forces, using the example of employment instability at General Electric to show how feedback loops within corporate decision-making create unintended consequences. Forrester's approach emphasizes the importance of understanding policy and decision-making processes in social systems, providing comprehensive guidance on problem definition, model building, and result interpretation. The book argues that systems planning requires understanding the feedback processes that govern organizational behavior, making it particularly relevant for career trajectory analysis where individual and organizational decisions interact in complex ways over extended time periods.

\subsubsubsection{Building a System Dynamics Model Guide}

The MIT guide "Building a System Dynamics Model Part 1: Conceptualization" provides practical methodology for the model development process[20]. The guide outlines four recursive stages of modeling: conceptualization, formulation, testing, and implementation, with detailed steps for each stage. The conceptualization stage, which the guide covers in detail, includes defining model purpose and boundaries, identifying key variables, describing reference modes of behavior, and diagramming basic feedback mechanisms. The guide emphasizes that modeling stages are recursive, requiring modelers to return to previous stages as new insights emerge. This systematic approach is particularly valuable for career trajectory analysis, where the complexity of human career decisions and organizational policies requires careful attention to model scope and structure to ensure meaningful and actionable results.

\subsubsection{Mathematical Approach}

System Dynamics employs a mathematical framework based on differential equations that represent the rates of change in system stocks over time. The fundamental building blocks are **stocks** (accumulations) and **flows** (rates of change), connected through feedback loops that create the dynamic behavior of the system[11]. The basic mathematical relationship is expressed as:

$$ \text{Stock}(t) = \int_{t_0}^{t} [\text{Inflows}(s) - \text{Outflows}(s)] ds + \text{Stock}(t_0) $$

where the stock at time $t$ equals the integral of net flows from the initial time $t_0$ plus the initial stock value[15].

For career trajectory analysis, this translates to tracking the accumulation of personnel at different career stages. For example, the number of officers at a particular rank would be represented as:

$$ \text{Officers}_{\text{rank}}(t) = \int_{t_0}^{t} [\text{Promotions\_In}(s) - \text{Promotions\_Out}(s) - \text{Separations}(s)] ds + \text{Officers}_{\text{rank}}(t_0) $$

Flow rates are typically modeled as functions of multiple variables, incorporating both linear and nonlinear relationships. A promotion flow might be expressed as:

$$ \text{Promotion\_Rate} = f(\text{Eligible\_Population}, \text{Performance\_Scores}, \text{Selection\_Rate}, \text{Time\_Delays}) $$

where $f$ represents a function that may include table functions for nonlinear relationships, conditional logic for policy rules, and time delays for processing periods[3].

**Feedback loops** are mathematically represented through the interconnection of these equations, where the output of one equation becomes an input to another, creating circular causal chains. **Positive feedback loops** amplify changes through relationships where increases in one variable lead to increases in another variable that eventually reinforces the original increase. **Negative feedback loops** create balancing behavior where increases in one variable eventually lead to forces that counteract the original increase[2][15].

The system is solved numerically using integration methods such as Euler's method or Runge-Kutta methods, updating all variables simultaneously in small time steps to capture the dynamic interactions between system components[15]. This approach allows the model to simulate how career trajectories emerge from the complex interaction of individual decisions, organizational policies, and external constraints over time.

\subsubsection{Example Applications}

\subsubsubsection{IBM Workforce Lifecycle Management}

Yoshizuimi and Okano's research presents a novel application of System Dynamics to workforce management by introducing the concept of a "workforce supply chain" that addresses both demand-side project management and supply-side human resource management issues[7]. Their approach uses System Dynamics modeling to capture causal relationships and feedback loops in workforce systems, demonstrating how the methodology can expose dynamic behavior in workforce management systems and enable adaptive control mechanisms. The study shows how System Dynamics can integrate individual career progression with organizational workforce planning, providing insights into how to recruit, develop, and deploy personnel effectively across different organizational needs. This application demonstrates the value of System Dynamics for understanding the complex interactions between individual career aspirations and organizational workforce requirements, particularly in service industries where human capital represents the primary organizational asset.

\subsubsubsection{U.S. Army Officer Talent Retention}

Richard Dulce II's MIT thesis provides a comprehensive application of System Dynamics to analyze U.S. Army officer retention, demonstrating how the methodology can address complex factors influencing career decisions in military organizations[8]. The research synthesizes data from multiple surveys to identify key variables affecting retention and integrates these into a qualitative System Dynamics model that reveals intricate feedback loops and interdependencies. The study emphasizes how System Dynamics enables a holistic approach to policy development, moving beyond simple cause-and-effect relationships to understand how retention policies interact with career satisfaction, advancement opportunities, and organizational culture. This application illustrates the methodology's capacity to inform evidence-based policy recommendations by revealing systemic issues that might not be apparent through traditional analytical approaches, highlighting the importance of understanding interconnections in complex organizational systems.

\subsubsubsection{Military Capability and Proxy Force Analysis}

Recent research has applied System Dynamics to analyze military capability development and force structure, particularly in the context of proxy force management and personnel advancement systems[17]. This work demonstrates how System Dynamics can model the complex dynamics of military personnel progression through aging chains that evaluate promotion requirements against individual advancement patterns. The research shows how officer recruitment, promotion, and dismissal flows interact to shape overall organizational capability, with particular attention to the rigid hierarchical nature of military advancement systems. These applications highlight System Dynamics' strength in modeling career progression systems where advancement follows strict rules and requirements, while also capturing the dynamic interactions between individual career decisions and organizational force structure needs.

\subsubsubsection{RAND Officer Corps Analysis}

The RAND Corporation's analysis of officer career management systems demonstrates how System Dynamics concepts can inform the design of alternative career management approaches[19]. Their research identifies how binary choices about entry points and separation mechanisms fundamentally shape career flow patterns, showing how "up-or-out" systems create different dynamics than alternative approaches that allow lateral entry or natural attrition. The study illustrates how System Dynamics thinking can reveal the long-term consequences of career management policies, particularly how changes in promotion timing, selection criteria, or tenure requirements create ripple effects throughout the officer corps. This application shows the value of System Dynamics for strategic workforce planning, enabling decision-makers to understand how policy changes will affect career patterns and organizational capability over extended time horizons.

\subsubsection{Critiques}

System Dynamics faces several significant limitations when applied to career trajectory analysis. **Data availability and quality** represent primary constraints, as System Dynamics models require reliable data to parameterize relationships between variables, and if such data is scarce, model accuracy suffers significantly[9]. **Model complexity and simplification** create fundamental tensions, as the methodology often simplifies complex real-world systems through omission of key variables, aggregation errors that lose specificity, and inappropriate linearity assumptions when relationships are actually nonlinear[9].

**Subjectivity in model building** poses another major limitation, as model construction involves subjective choices about variable inclusion, relationship representation, and behavioral assumptions that can be influenced by modeler biases and perspectives[9]. This subjectivity becomes particularly problematic in career analysis where different stakeholders may have conflicting views about what factors drive career success. **Parameter uncertainty and sensitivity** issues arise because System Dynamics models rely on parameter values that are often uncertain or estimated, requiring computationally intensive sensitivity analysis that may not reveal all critical sensitivities, especially when relationships are nonlinear or complex[9].

**Validation difficulties** represent perhaps the most significant challenge for career trajectory applications, as validating System Dynamics models can be extremely challenging when systems are difficult to observe or experiment with[9]. Traditional validation methods like comparing outputs to historical data may not be feasible for career systems, and even models that fit historical data well may not accurately predict future behavior if systems undergo significant changes. **Scale and scope limitations** also constrain the methodology, as it can be difficult to model very large and complex systems due to computational constraints and data limitations, while models typically focus on particular systems without adequately accounting for interactions between different systems or broader contextual factors[9].

\subsubsection{Software}

\subsubsubsection{PySD Python Library}

PySD represents a Python-based library specifically designed for running System Dynamics models, with the primary purpose of improving integration of Big Data and Machine Learning into the System Dynamics workflow[12]. The library translates model files into Python modules and provides comprehensive methods to modify, simulate, and observe translated models through an intermediate representation that facilitates adding builders in other languages. PySD's design philosophy emphasizes accessibility and integration with the broader Python data science ecosystem, enabling researchers to combine System Dynamics modeling with advanced analytics capabilities. For career trajectory analysis, PySD offers particular advantages in handling large datasets and applying machine learning techniques to enhance model calibration and validation, making it well-suited for organizations with substantial personnel data and computational resources.

\subsubsubsection{BPTK-Py Python Framework}

The Business Prototyping Toolkit for Python (BPTK-Py) provides a comprehensive simulation and plotting engine designed for both System Dynamics and Agent-Based models[13]. This framework offers the capability to simulate models within Python environments and create sophisticated visualizations for use in Jupyter Lab/Notebooks, with options to export simulation results for external processing. BPTK-Py includes transentis' sdcc parser for transpiling visual models created in environments like Stella into Python code, supporting the XMILE format standard. The framework particularly excels in creating interactive plots from simulation results and retrieving data as Pandas DataFrame timeseries, making it highly suitable for career trajectory analysis where visualization and data manipulation are essential for understanding complex temporal patterns and communicating results to diverse stakeholders.

\subsubsubsection{simecol R Package}

The simecol package provides an object-oriented framework for simulating ecological and other dynamic systems within the R environment[14]. While originally designed for ecological modeling, simecol supports differential equations, individual-based models, and other dynamic system types that are relevant for career trajectory analysis. The package emphasizes structuring simulation scenarios to improve code readability and reusability, with strong integration into R's statistical computing ecosystem. For career trajectory research, simecol offers advantages in statistical analysis and hypothesis testing, providing seamless integration with R's extensive statistical libraries and enabling sophisticated analysis of model outputs, parameter estimation, and uncertainty quantification that are essential for rigorous career trajectory research.

\subsubsubsection{Specialized System Dynamics Software}

Several commercial and academic software packages have been specifically developed for System Dynamics modeling. **Vensim** by Ventana Systems represents one of the most widely used commercial System Dynamics software packages, offering both graphical model building interfaces and advanced analysis capabilities including Monte Carlo simulation and optimization. **Stella/iThink** by isee systems provides another popular commercial option with strong educational applications and user-friendly interfaces for model building and simulation. **PowerSim** offers enterprise-focused System Dynamics software with particular strength in business applications. **AnyLogic** provides a multi-method modeling platform that combines System Dynamics with Agent-Based and Discrete Event modeling capabilities. These specialized tools often provide more sophisticated model building interfaces and analysis capabilities than general-purpose programming libraries, but may offer less flexibility for integration with other analytical tools and custom data processing workflows.

\subsubsection{Example Study Design}

\subsubsubsection{Key Variables}

The study would incorporate multiple categories of variables derived from the Army officer indicator framework. **Stock variables** would include officer populations at each rank (Captain, Major, Lieutenant Colonel, Colonel) within each branch (Armor, Logistics, Aviation, Cyber), with separate tracking by years of service and key demographic characteristics[1]. **Flow variables** would encompass promotion rates between ranks, retention rates at each career stage, and recruitment flows into each branch, with flows differentiated by performance levels and assignment types.

**Individual performance variables** would include Officer Evaluation Report (OER) scores, Battalion Commander Assessments, specialized assessment results (such as CATA and CASP for Cyber), and educational achievements including commission source, graduate education, and skill identifiers[1]. **Experience variables** would capture Key Development (KD) assignments, broadening experiences, joint service credit, and specialized training completion. **Personal characteristic variables** would include age, gender, race, marital status, ASVAB scores, and non-cognitive attributes such as hardiness measures, motivation, complex problem-solving ability, creative thinking, and responsibility[1].

**Policy variables** would represent organizational parameters such as promotion timelines, selection rates, assignment policies, and educational requirements that vary by branch and career stage. **Environmental variables** would capture external factors affecting career decisions, including civilian employment opportunities, economic conditions, and societal attitudes toward military service.

\subsubsubsection{Sample \& Data Collection}

The study would utilize a comprehensive longitudinal dataset combining multiple Army personnel databases to track officer careers from commissioning through separation or retirement. **Primary data sources** would include the Army's Officer Record Brief (ORB) system, promotion board results, assignment histories, and evaluation records spanning 20 years to capture complete career trajectories. **Supplementary data** would incorporate survey responses on career satisfaction, retention intentions, and personal factors from existing Army surveys such as the Center for Army Leadership Annual Survey of Army Leadership.

**Sample design** would include all officers commissioned between 2000-2010 in the four target branches (Armor, Logistics, Aviation, Cyber), providing sufficient time to observe career progression through field-grade ranks while maintaining relevance to current personnel systems[1]. **Data preprocessing** would involve creating standardized performance metrics across different evaluation systems, coding assignment types for developmental value, and constructing composite measures for non-cognitive attributes based on available assessment data. **Ethical considerations** would require de-identification procedures and appropriate institutional review board approval for using personnel records in research contexts.

\subsubsubsection{Analysis Approach}

The analysis would follow the established System Dynamics modeling methodology beginning with **conceptualization** to define model boundaries, identify key feedback loops, and develop reference mode behaviors for retention and promotion patterns by branch[20]. **Causal loop diagrams** would map the relationships between individual characteristics, organizational policies, career satisfaction, and retention decisions, with particular attention to feedback mechanisms where retention challenges affect promotion opportunities and organizational culture.

**Stock and flow model construction** would represent officer populations as stocks with flows representing promotions, lateral transfers, and separations, using rate equations that incorporate performance, timing, and policy parameters[11]. **Model calibration** would use historical data to estimate parameter values and validate model behavior against observed promotion and retention patterns. **Sensitivity analysis** would examine how changes in key parameters (promotion criteria, assignment policies, evaluation standards) affect long-term career outcomes and force structure. **Policy simulation** would test alternative career management approaches, such as modified promotion timelines, expanded lateral entry, or enhanced retention incentives, to identify interventions that improve career outcomes while maintaining organizational effectiveness.

\subsubsubsection{Potential Findings}

The study would likely reveal **complex feedback relationships** between individual career decisions and organizational outcomes, showing how retention challenges in specific branches create cascading effects on promotion opportunities, assignment quality, and organizational culture that further exacerbate retention problems[8]. **Branch-specific patterns** would emerge, with Cyber and Aviation potentially showing different career dynamics due to high civilian demand for technical skills, while Armor and Logistics might demonstrate more traditional military career progression patterns[1].

**Policy intervention effects** would demonstrate how seemingly minor changes in evaluation criteria, assignment policies, or promotion timing create significant long-term impacts on career trajectories and force structure. **Performance predictors** would reveal which early career indicators most strongly predict long-term success and retention, potentially showing that factors like hardiness, complex problem-solving ability, and key developmental assignments have greater predictive power than traditional measures. **Unintended consequences** of current policies would likely be identified, such as how competitive promotion processes might discourage collaboration or how rigid assignment requirements might reduce career satisfaction and retention.

\subsubsubsection{Potential Implications}

**Policy recommendations** would focus on leverage points where modest changes could produce significant improvements in career outcomes, such as modifying evaluation systems to better recognize collaborative leadership, expanding opportunities for broadening assignments, or adjusting promotion timelines to better align with branch-specific career development needs[8]. **Talent management strategies** would emphasize the importance of early identification and development of high-potential officers, with particular attention to providing challenging assignments and professional development opportunities that enhance both capability and retention.

**Force structure planning** would benefit from understanding how career management policies affect long-term officer availability and quality, enabling more accurate projections of future leadership capacity and identification of potential capability gaps. **Research implications** would suggest the value of ongoing monitoring and evaluation of career management systems using System Dynamics approaches, recognizing that complex organizational systems require continuous attention to feedback mechanisms and unintended consequences. **Broader organizational learning** would demonstrate how System Dynamics thinking can improve policy development by encouraging consideration of long-term, systemic effects rather than short-term, local optimizations.

\begin{thebibliography}{21}

\bibitem{army_indicators}
Potential Indicators. U.S. Army Branch Indicators. (Attached PDF document).

\bibitem{systemdynamics_org}
System Dynamics Society. (2022). What is System Dynamics. Retrieved from https://systemdynamics.org/what-is-system-dynamics/

\bibitem{sterman_review}
Sterman, J. (2000). Business Dynamics: Systems Thinking and Modeling for a Complex World. Irwin/McGraw-Hill, Homewood, IL. ProQuest Document Review.

\bibitem{forrester_origin}
System Dynamics Society. Origin of System Dynamics. Retrieved from https://systemdynamics.org/origin-of-system-dynamics/

\bibitem{senge_fifth}
Senge, P. (2003). The Fifth Discipline: The Art and Practice of the Learning Organization. Wikipedia entry.

\bibitem{industrial_dynamics}
Forrester, J. W. (1999). Industrial Dynamics. System Dynamics Society. Originally published 1961.

\bibitem{ibm_workforce}
Yoshizuimi, T., \& Okano, H. (2007). Effective workforce lifecycle management via System Dynamics modeling and simulation. WSC 2007.

\bibitem{dulce_thesis}
Dulce, R. II. (2024). A System Dynamics Approach to Analyzing U.S. Army Officer Talent Retention. MIT SDM Thesis.

\bibitem{sd_limitations}
Sustainability Directory. (2025). What Are the Limitations of System Dynamics Approach? Retrieved from https://pollution.sustainability-directory.com/

\bibitem{sterman_tools}
Sterman, J. D. (2001). System Dynamics Modeling: Tools for Learning in a Complex World. California Management Review, 43(4).

\bibitem{systems_analysis}
IISD SAVI Academy. (2020). Introduction to Systems Analysis and System Dynamics. PDF Document.

\bibitem{pysd}
PySD Development Team. (2015). PySD - Python System Dynamics. Retrieved from https://pysd.readthedocs.io

\bibitem{bptk_py}
transentis. (2019). BPTK-Py: Business Prototyping Toolkit for Python. PyPI package version 0.5.

\bibitem{simecol}
Petzoldt, T. (2025). simecol: Simulation of Ecological (and Other) Dynamic Systems. CRAN R package version 0.9-2.

\bibitem{sd_wikipedia}
Wikipedia. (2002). System Dynamics. Retrieved from https://en.wikipedia.org/wiki/System\_dynamics

\bibitem{forrester_decade}
Forrester, J. W. (1968). Industrial Dynamics—After the First Decade. Management Science, 14(7), 398-415.

\bibitem{military_capability}
Proceedings of the System Dynamics Society. (2024). Bolstering a Proxy Actor's Military Capability. Conference Paper P1250.

\bibitem{umbrex_framework}
Umbrex. (2025). What is the System Dynamics Model? Retrieved from https://umbrex.com/resources/change-management-frameworks/

\bibitem{rand_officer}
RAND Corporation. (2004). Building an Officer Corps for the Future. Research Brief RB-7503.

\bibitem{building_model}
MIT. (1998). Building a System Dynamics Model Part 1: Conceptualization. OCW Document D-4597.

\bibitem{fourweekmba}
FourWeekMBA. (2024). System Dynamics. Retrieved from https://fourweekmba.com/system-dynamics/

\end{thebibliography}

\end{document}
