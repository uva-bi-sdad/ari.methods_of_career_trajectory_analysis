% Options for packages loaded elsewhere
% Options for packages loaded elsewhere
\PassOptionsToPackage{unicode}{hyperref}
\PassOptionsToPackage{hyphens}{url}
\PassOptionsToPackage{dvipsnames,svgnames,x11names}{xcolor}
%

\documentclass[./main.tex]{subfiles}

\usepackage{xcolor}
\usepackage{amsmath,amssymb}
\setcounter{secnumdepth}{-\maxdimen} % remove section numbering
\usepackage{iftex}
\ifPDFTeX
  \usepackage[T1]{fontenc}
  \usepackage[utf8]{inputenc}
  \usepackage{textcomp} % provide euro and other symbols
\else % if luatex or xetex
  \usepackage{unicode-math} % this also loads fontspec
  \defaultfontfeatures{Scale=MatchLowercase}
  \defaultfontfeatures[\rmfamily]{Ligatures=TeX,Scale=1}
\fi
\usepackage{lmodern}
\ifPDFTeX\else
  % xetex/luatex font selection
\fi
% Use upquote if available, for straight quotes in verbatim environments
\IfFileExists{upquote.sty}{\usepackage{upquote}}{}
\IfFileExists{microtype.sty}{% use microtype if available
  \usepackage[]{microtype}
  \UseMicrotypeSet[protrusion]{basicmath} % disable protrusion for tt fonts
}{}
\makeatletter
\@ifundefined{KOMAClassName}{% if non-KOMA class
  \IfFileExists{parskip.sty}{%
    \usepackage{parskip}
  }{% else
    \setlength{\parindent}{0pt}
    \setlength{\parskip}{6pt plus 2pt minus 1pt}}
}{% if KOMA class
  \KOMAoptions{parskip=half}}
\makeatother
% Make \paragraph and \subparagraph free-standing
\makeatletter
\ifx\paragraph\undefined\else
  \let\oldparagraph\paragraph
  \renewcommand{\paragraph}{
    \@ifstar
      \xxxParagraphStar
      \xxxParagraphNoStar
  }
  \newcommand{\xxxParagraphStar}[1]{\oldparagraph*{#1}\mbox{}}
  \newcommand{\xxxParagraphNoStar}[1]{\oldparagraph{#1}\mbox{}}
\fi
\ifx\subparagraph\undefined\else
  \let\oldsubparagraph\subparagraph
  \renewcommand{\subparagraph}{
    \@ifstar
      \xxxSubParagraphStar
      \xxxSubParagraphNoStar
  }
  \newcommand{\xxxSubParagraphStar}[1]{\oldsubparagraph*{#1}\mbox{}}
  \newcommand{\xxxSubParagraphNoStar}[1]{\oldsubparagraph{#1}\mbox{}}
\fi
\makeatother


\usepackage{longtable,booktabs,array}
\usepackage{calc} % for calculating minipage widths
% Correct order of tables after \paragraph or \subparagraph
\usepackage{etoolbox}
\makeatletter
\patchcmd\longtable{\par}{\if@noskipsec\mbox{}\fi\par}{}{}
\makeatother
% Allow footnotes in longtable head/foot
\IfFileExists{footnotehyper.sty}{\usepackage{footnotehyper}}{\usepackage{footnote}}
\makesavenoteenv{longtable}
\usepackage{graphicx}
\makeatletter
\newsavebox\pandoc@box
\newcommand*\pandocbounded[1]{% scales image to fit in text height/width
  \sbox\pandoc@box{#1}%
  \Gscale@div\@tempa{\textheight}{\dimexpr\ht\pandoc@box+\dp\pandoc@box\relax}%
  \Gscale@div\@tempb{\linewidth}{\wd\pandoc@box}%
  \ifdim\@tempb\p@<\@tempa\p@\let\@tempa\@tempb\fi% select the smaller of both
  \ifdim\@tempa\p@<\p@\scalebox{\@tempa}{\usebox\pandoc@box}%
  \else\usebox{\pandoc@box}%
  \fi%
}
% Set default figure placement to htbp
\def\fps@figure{htbp}
\makeatother





\setlength{\emergencystretch}{3em} % prevent overfull lines

\providecommand{\tightlist}{%
  \setlength{\itemsep}{0pt}\setlength{\parskip}{0pt}}



 


\KOMAoption{captions}{tableheading}
\makeatletter
\@ifpackageloaded{caption}{}{\usepackage{caption}}
\AtBeginDocument{%
\ifdefined\contentsname
  \renewcommand*\contentsname{Table of contents}
\else
  \newcommand\contentsname{Table of contents}
\fi
\ifdefined\listfigurename
  \renewcommand*\listfigurename{List of Figures}
\else
  \newcommand\listfigurename{List of Figures}
\fi
\ifdefined\listtablename
  \renewcommand*\listtablename{List of Tables}
\else
  \newcommand\listtablename{List of Tables}
\fi
\ifdefined\figurename
  \renewcommand*\figurename{Figure}
\else
  \newcommand\figurename{Figure}
\fi
\ifdefined\tablename
  \renewcommand*\tablename{Table}
\else
  \newcommand\tablename{Table}
\fi
}
\@ifpackageloaded{float}{}{\usepackage{float}}
\floatstyle{ruled}
\@ifundefined{c@chapter}{\newfloat{codelisting}{h}{lop}}{\newfloat{codelisting}{h}{lop}[chapter]}
\floatname{codelisting}{Listing}
\newcommand*\listoflistings{\listof{codelisting}{List of Listings}}
\makeatother
\makeatletter
\makeatother
\makeatletter
\@ifpackageloaded{caption}{}{\usepackage{caption}}
\@ifpackageloaded{subcaption}{}{\usepackage{subcaption}}
\makeatother
\usepackage{bookmark}
\IfFileExists{xurl.sty}{\usepackage{xurl}}{} % add URL line breaks if available
\urlstyle{same}
\hypersetup{
  pdftitle={Retention and Attrition Patterns for Analyzing Career Trajectories},
  colorlinks=true,
  linkcolor={blue},
  filecolor={Maroon},
  citecolor={Blue},
  urlcolor={Blue},
  pdfcreator={LaTeX via pandoc}}



\begin{document}



Retention and attrition patterns analysis is a methodological approach
that applies event history analysis (also known as survival analysis)
techniques to understand and predict career trajectories within
organizations. This approach focuses specifically on analyzing when and
why individuals leave an organization (attrition) or stay (retention)
over time, and how various factors influence these patterns. The
methodology allows researchers to model the ``time until event'' (such
as departure from an organization) while accounting for censored
observations (individuals who haven't experienced the event by the end
of the observation period).

\subsection{1. Approach Description \&
Goal}\label{approach-description-goal}

Retention and Attrition Patterns analysis applies survival
analysis/event history analysis techniques to understand career
trajectories, particularly focusing on when and why employees leave or
stay in an organization. The approach models the timing of critical
career events such as job changes, promotions, or organizational exits,
while accounting for censored observations (when individuals haven't
experienced the event by the end of the study period).

The primary goals of this approach include predicting which employees
are at risk of leaving an organization and when; identifying key factors
that influence retention and attrition decisions; understanding patterns
of career progression within organizations; developing targeted
interventions to improve retention rates; and modeling career
trajectories to inform talent management strategies. As noted in
research, understanding the causes of events like employee turnover
requires analyzing both the timing of events and the factors that
influence them, making this approach particularly valuable for
organizations facing high turnover costs{[}6{]}{[}13{]}.

\subsection{2. Critical Variables}\label{critical-variables}

Typical variable categories used in retention and attrition pattern
analysis include:

\begin{enumerate}
\def\labelenumi{\arabic{enumi}.}
\item
  \textbf{Individual Characteristics}: Demographic variables (age,
  gender, race/ethnicity), educational background (level, field of
  study), cognitive ability (AFQT scores), personality traits (TAPAS
  dimensions), and prior experience{[}10{]}.
\item
  \textbf{Job-Related Factors}: Compensation and benefits, job
  satisfaction, workload, work-life balance, career development
  opportunities, and recognition. Research shows that lack of career
  development remains a key driver of employee attrition---cited by 40\%
  of departing employees{[}2{]}.
\item
  \textbf{Organizational Factors}: Corporate culture measures (which can
  be 10.4 times more powerful than compensation in predicting turnover),
  leadership quality, team dynamics, training opportunities, and
  innovation levels{[}19{]}.
\item
  \textbf{Time-Related Variables}: Tenure in organization, time since
  last promotion, career stage, and historical timing (economic
  conditions, labor market).
\item
  \textbf{Event-Specific Variables}: Type of attrition (voluntary
  vs.~involuntary), reason for departure (medical/physical, misconduct,
  performance), and destination after departure{[}10{]}.
\end{enumerate}

These variables serve as predictors, controls, or outcome measures,
depending on the specific research question.

\subsection{3. Key Overviews}\label{key-overviews}

Allison, P.D. (1982) ``Discrete-Time Methods for the Analysis of Event
Histories'' {[}6{]}: Paul Allison's seminal paper introduces
discrete-time methods for analyzing event history data, particularly
beneficial when dealing with tied event times (multiple events occurring
simultaneously). The paper addresses the challenges of conducting event
history analysis, primarily focusing on two significant issues:
censoring (when the event has not occurred by the end of observation)
and time-varying explanatory variables. Allison demonstrates how to
transform event history data into a person-period format suitable for
logistic regression analysis, allowing researchers to estimate the
effects of various factors on the probability of an event occurring
within discrete time intervals. This approach has become foundational in
career trajectory analysis because it enables researchers to account for
time-dependent covariates and censored observations while using familiar
statistical techniques like logistic regression.

Li, H., Ge, Y., Zhu, H., Xiong, H., \& Zhao, H. (2016) ``Prospecting the
Career Development of Talents: A Survival Analysis Perspective''
{[}4{]}: This innovative paper applies survival analysis techniques
specifically to model talent career paths, focusing on two critical
issues in talent management: turnover and career progression. The
authors formulate the prediction of talent retention status across a
sequence of time intervals as a multi-task learning problem, where
prediction at each time interval is considered a separate task. They
incorporate ranking constraints to effectively model both censored and
uncensored data, capturing the intrinsic properties exhibited in
lifetime modeling with non-recurrent and recurrent events. For modeling
career progression, they predict relative occupational levels at
different time intervals, with ranking constraints imposed on different
occupational levels to reduce prediction error. The authors demonstrate
the effectiveness of their approach through evaluation against baseline
methods using real-world talent data, providing a sophisticated
framework for predicting both turnover and career advancement patterns.

Maas, I. ``The Use of Event-History-Analysis in Career Research''
{[}13{]}: This comprehensive introduction to event-history analysis in
career research highlights how this methodological approach has become
an essential tool for sociologists studying contemporary careers. Maas
contrasts event-history analysis with alternative methods like status
attainment models and log-linear models, emphasizing that event-history
models are better equipped for causal analysis because they clarify the
direction of causality, distinguish strengths of reciprocal effects, and
model effects of previous history. The chapter addresses practical
challenges, including the extensive data requirements (complete
information on all jobs and their timing), and discusses three types of
data sources: retrospective designs, panel studies, and administrative
records. Additionally, Maas explains how event-history models allow for
the introduction of time-specific independent variables and describes
three main types: semi-parametric models (Cox proportional hazard
model), discrete-time models, and parametric models. The chapter
concludes by surveying leading issues and findings from applying event
history analysis to careers, examining individual characteristics,
social influences, and broader socioeconomic factors.

``Introducing Survival and Event History Analysis'' {[}16{]}: This
introductory text provides a comprehensive overview of survival and
event history analysis techniques, explaining that these methods are
particularly suited for analyzing categorical sequences of events to
model entire event history career trajectories. The book focuses on the
order and timing of events rather than gradual changes, making it ideal
for studying career transitions, promotions, and job changes. It
introduces fundamental concepts such as hazard functions, survival
curves, and censoring, while explaining both parametric and
non-parametric approaches to survival analysis. The text addresses
practical implementation issues, including data preparation, model
specification, interpretation of results, and diagnostics. Through
illustrative examples, the book demonstrates how these techniques can
reveal patterns in career development that might remain hidden with
conventional statistical methods, making it an essential resource for
researchers studying career trajectories, employee retention, and
professional mobility.

\subsection{4. Mathematical Approach}\label{mathematical-approach}

Retention and attrition pattern analysis primarily utilizes event
history analysis (survival analysis) methods, which focus on modeling
the time until an event occurs. In the context of career trajectories,
the ``event'' typically refers to leaving an organization (attrition) or
receiving a promotion (career progression). The mathematical approach
involves several key concepts:

The fundamental concept is the hazard function, which represents the
instantaneous rate at which events occur, given that the individual has
survived up to that point. For career trajectory analysis, the hazard
function h(t) represents the probability that an employee leaves or gets
promoted in a small time interval {[}t, t+Δt{]}, given that they have
remained in their current state until time t:

h(t) = limΔt→0 {[}P(t ≤ T \textless{} t+Δt \textbar{} T ≥ t) / Δt{]}

Where T is the random variable representing the time until the event
occurs.

The survival function S(t), which represents the probability of
``surviving'' (remaining in the current state) beyond time t:

S(t) = P(T \textgreater{} t) = exp{[}-∫0t h(u)du{]}

For analyzing censored data, the Kaplan-Meier estimator provides a
non-parametric estimate of the survival function:

S(t) = ∏ti≤t {[}1 - (di/ni){]}

Where ti represents the distinct event times, di is the number of events
at time ti, and ni is the number of individuals at risk just before
ti{[}14{]}.

For modeling the effects of covariates, Cox proportional hazards
regression is commonly used:

h(t\textbar X) = h0(t) × exp(β1X1 + β2X2 + \ldots{} + βpXp)

Where h0(t) is the baseline hazard function and X1, X2, \ldots, Xp are
the covariates with corresponding coefficients β1, β2, \ldots,
βp{[}14{]}.

For discrete-time data, logistic regression can be used to model the
conditional probability of an event occurring at time t, given survival
up to that point:

logit{[}P(T = t \textbar{} T ≥ t, X){]} = αt + β1X1 + β2X2 + \ldots{} +
βpXp

Where αt represents the baseline logit-hazard at time t{[}6{]}.

For competing risks (multiple types of events, such as different reasons
for leaving an organization), the cause-specific hazard function for
event type k is:

hk(t) = limΔt→0 {[}P(t ≤ T \textless{} t+Δt, K = k \textbar{} T ≥ t) /
Δt{]}

Where K denotes the type of event{[}14{]}.

\subsection{5. Example Applications}\label{example-applications}

``An Analysis of Career Trajectories and Occupational Transitions''
(U.S. Department of Labor) {[}12{]}: This comprehensive study by the
Department of Labor examines wage growth and occupational transitions
over 10-year periods to identify ``launchpad'' occupations that
facilitate career advancement. Using longitudinal panel surveys, the
researchers analyze variations in trajectories among workers starting in
mid-level occupations, identifying which occupations lead to higher wage
growth and better long-term outcomes. The study incorporates both
occupational characteristics (skill requirements, licensing) and worker
demographics to understand disparities in wage growth among different
populations. By tracking actual career paths rather than theoretical
possibilities, the research reveals which occupations consistently serve
as reliable springboards for career advancement across different sectors
and demographic groups. The trajectory analysis methodology allows for
the examination of multiple outcomes beyond wage growth, including
unemployment periods and educational attainment, providing a holistic
view of career development patterns that can inform workforce
development strategies and career counseling approaches.

``Attrition and reenlistment in the Army: Using the Tailored Adaptive
Personality Assessment System'' {[}10{]}: This military-focused study
employs survival analysis to predict attrition and reenlistment among
U.S. Army Soldiers, evaluating both cognitive measures (Armed Forces
Qualification Test) and non-cognitive assessments (Tailored Adaptive
Personality Assessment System, TAPAS). The researchers examine the
complete timeline of enlisted Soldiers' first term of service,
categorizing attrition by reason for separation (medical/physical,
misconduct, performance-related) and analyzing variations in first-term
reenlistment across different Military Occupational Specialties. The
analysis reveals that attrition associated with medical/physical and
performance reasons tends to occur early in Soldiers' careers, while
misconduct-related attrition emerges after several months of service.
The study demonstrates that personality characteristics as measured by
TAPAS contribute incrementally beyond cognitive ability in predicting
retention outcomes, with different attrition categories showing
distinctive temporal patterns. By modeling attrition over time and
across categories, the research provides insights for developing
targeted interventions at specific career stages to enhance force
readiness and reduce the substantial costs associated with early Soldier
attrition.

``Dynamic work trajectories and their interplay with family over the
life course'' {[}18{]}: This innovative study applies event history
analysis to examine how work trajectories and family dynamics interact
throughout individuals' life courses, capturing the bidirectional
relationship between career development and family transitions. The
researchers utilize sequence analysis and event history methods to model
entire career paths while accounting for critical family events such as
marriage, childbirth, and caregiving responsibilities. The study reveals
how family-related transitions influence career trajectories differently
for men and women, identifying critical junctures where career paths may
diverge based on family decisions. By analyzing transitions into the
labor market and subsequent career movements in relation to family
formation patterns, the research highlights how early career decisions
create path dependencies that shape long-term outcomes. The application
of survival analysis techniques allows the researchers to quantify the
timing and likelihood of career disruptions following family events,
providing insights into the complex interplay of work and family that
traditional cross-sectional analyses would miss.

``Predicting employee attrition and explaining its determinants''
{[}1{]}: This cutting-edge study applies machine learning models to
real-world data from a prominent Italian financial institution to
predict employee attrition and identify its key determinants. The
researchers employ survival analysis techniques to model the time until
employee departure, incorporating both static factors (demographics,
education, personality) and time-varying covariates (performance
ratings, compensation changes, role transitions). By formulating
attrition prediction as a multi-task learning problem across different
time horizons, the study captures both immediate attrition risk and
long-term retention patterns. The methodology accounts for competing
risks by distinguishing between voluntary departures and involuntary
separations, allowing for targeted interventions based on attrition
type. The study demonstrates the superiority of this approach over
traditional turnover models by capturing the temporal dynamics of
attrition risk throughout the employee lifecycle, identifying critical
periods when intervention might be most effective. The insights
generated help organizations develop proactive retention strategies
tailored to specific employee segments and career stages.

\subsection{6. Critiques}\label{critiques}

Based on the search results, particularly from {[}9{]}, there are
several significant limitations of using survival analysis/event history
analysis for studying retention and attrition patterns:

\begin{enumerate}
\def\labelenumi{\arabic{enumi}.}
\item
  \textbf{Censoring Assumptions}: The approach requires specific
  assumptions about censored data that may not reflect reality. This can
  potentially bias results, especially if censoring is related to the
  likelihood of experiencing the event (informative censoring){[}9{]}.
\item
  \textbf{Complexity for Non-Statisticians}: The interpretation and
  implementation of survival analysis can be difficult for practitioners
  without specialized statistical knowledge, limiting its practical
  application in organizational settings{[}9{]}.
\item
  \textbf{Modeling of Time-Dependent Covariates}: Incorporating
  variables that change over time (such as job satisfaction or
  compensation) can be methodologically complex and may lead to model
  misspecification if not handled properly{[}9{]}.
\item
  \textbf{High Variability with Small Sample Sizes}: Estimates can be
  unstable with small numbers of events or a high censoring rate, which
  is particularly problematic when studying rare events or specific
  subgroups{[}9{]}.
\item
  \textbf{Assumption of Independent Censoring}: The approach assumes
  that censoring is unrelated to prognosis, which may not always be true
  in organizational settings{[}9{]}.
\item
  \textbf{Competing Risks Challenges}: Standard methods do not
  adequately account for competing risks (different types of
  departures), which can lead to biased risk estimates{[}9{]}.
\item
  \textbf{Data Requirements}: The method requires high-quality, complete
  longitudinal data over follow-up periods, which can be difficult and
  expensive to obtain. As noted by Maas, ``Not only information on all
  jobs, but also on the timing of the transition from one job to the
  other is necessary''{[}13{]}.
\item
  \textbf{Proportional Hazards Assumption}: The commonly used Cox
  proportional hazards model assumes that the effect of predictors
  remains constant over time, which may not hold for many career-related
  factors{[}9{]}.
\end{enumerate}

\subsection{7. Software}\label{software}

\textbf{R ``survival'' package {[}7{]}:} The ``survival'' package is the
foundation of survival analysis in R, maintained by Terry Therneau of
the Mayo Clinic, and serves as the backbone for most survival analysis
functions in the R ecosystem. It provides comprehensive implementations
of essential survival analysis methods including Kaplan-Meier survival
curves, Cox proportional hazards models, parametric survival models,
competing risks analysis, and methods for handling time-dependent
covariates. The package includes robust functions for visualizing
survival data through various plotting methods, conducting hypothesis
tests to compare survival curves across groups, and implementing
diagnostics to assess model assumptions. With its extensive
documentation and long development history (spanning over two decades),
the survival package offers both basic functionality for beginners and
advanced options for complex survival modeling scenarios. Its
integration with other R packages extends its capabilities, making it
ideal for comprehensive retention and attrition studies across various
organizational contexts.

\textbf{Python ``lifelines'' library {[}8{]}:} The ``lifelines'' library
is Python's premier survival analysis package, providing a complete
suite of tools for modeling and analyzing time-to-event data with an
intuitive, scikit-learn inspired API that makes it accessible to data
scientists already familiar with Python's data science ecosystem. The
library implements core survival analysis techniques including
Kaplan-Meier estimators, Cox Proportional Hazards models, Aalen's
Additive model, and various parametric models (Weibull, exponential,
log-normal, log-logistic), all optimized for performance with large
datasets common in modern retention analysis. It features robust
handling of left, right, and interval censored data, comprehensive
visualization tools for survival curves and hazard functions, and
built-in statistical tests for comparing survival distributions between
groups. The package includes model selection tools, residual analysis
functions for checking assumptions, and methods for determining variable
importance, making it particularly valuable for predictive modeling of
employee attrition. Lifelines' extensive documentation, tutorials, and
integration with the broader Python data science ecosystem (pandas,
numpy, matplotlib) make it an excellent choice for organizations using
Python for their data analytics infrastructure.

\textbf{SAS Survival Analysis Procedures:} SAS offers comprehensive
survival analysis capabilities through procedures like PROC LIFETEST,
PROC PHREG, and PROC ICPHREG, making it a powerful tool for retention
and attrition analysis in large organizations and research institutions
with existing SAS infrastructure. The software provides exceptional
handling of complex survey designs, robust procedures for competing
risks analysis, and sophisticated methods for dealing with
time-dependent covariates that are crucial for modeling dynamic aspects
of career trajectories. SAS's superior performance with extremely large
datasets makes it particularly valuable for enterprise-level workforce
analyses spanning thousands of employees over long time periods. The
software includes extensive diagnostics for assessing proportional
hazards assumptions, identifying influential observations, and checking
functional forms of continuous predictors. While requiring a commercial
license and specialized programming knowledge, SAS remains the gold
standard in many industries for its validated statistical procedures,
regulatory acceptance, and integration with broader enterprise data
systems, making it particularly valuable for regulated industries where
methodological reliability is paramount.

\textbf{STATA Survival Analysis Commands:} STATA's survival analysis
suite combines statistical rigor with accessibility through intuitive
commands like stcox, streg, and stcurve, alongside a graphical user
interface that makes it particularly approachable for researchers and HR
analysts with limited programming experience. The software excels in
handling complex survey designs, implementing flexible parametric
survival models through its stpm2 command, and offering extensive
post-estimation capabilities for predictive modeling of employee
retention. STATA's strengths include exceptional documentation with
practical examples, built-in publication-quality graphics for survival
curves and cumulative incidence functions, and seamless integration of
survival analysis with STATA's broader statistical capabilities. The
software provides comprehensive support for frailty models (accounting
for unobserved heterogeneity), multi-state modeling for complex career
transitions, and competing risks analysis through the stcrreg command.
While requiring a commercial license, STATA's balance of power and
usability, combined with its strong presence in academic research, makes
it particularly valuable for organizations collaborating with academic
partners on retention research or seeking methodological approaches
grounded in peer-reviewed literature.

\subsection{8. Example Study Design}\label{example-study-design}

\subsubsection{Example Retention and Attrition Patterns Study of Career
Trajectories of U.S. Army
Officers}\label{example-retention-and-attrition-patterns-study-of-career-trajectories-of-u.s.-army-officers}

\paragraph{Key Variables}\label{key-variables}

The study would incorporate variables from multiple domains to
comprehensively analyze officer retention and attrition patterns:

\begin{enumerate}
\def\labelenumi{\arabic{enumi}.}
\tightlist
\item
  \textbf{Branch-Specific Indicators:}

  \begin{itemize}
  \tightlist
  \item
    \textbf{Armor Division:} Technical proficiency scores, tactical
    evaluation results, field exercise performance ratings
  \item
    \textbf{Logistics Division:} Supply chain management effectiveness
    metrics, resource allocation efficiency ratings, logistical planning
    assessment scores
  \item
    \textbf{Aviation Division:} Flight hour accumulation, aircraft
    qualification levels, mission success rates
  \item
    \textbf{Cyber Division:} Technical certification achievement, cyber
    competition performance, problem-solving assessment scores
  \end{itemize}
\item
  \textbf{Career Progression Indicators:}

  \begin{itemize}
  \tightlist
  \item
    Time between promotions
  \item
    Performance evaluation scores
  \item
    Command position assignments
  \item
    Specialized training completion
  \item
    Deployment history (frequency, duration, locations)
  \end{itemize}
\item
  \textbf{Non-Cognitive Attributes:}

  \begin{itemize}
  \tightlist
  \item
    TAPAS (Tailored Adaptive Personality Assessment System) scores on
    dimensions such as achievement, dominance, attention seeking, and
    selflessness
  \item
    Resilience measures
  \item
    Leadership style assessments
  \item
    Team cohesion ratings
  \end{itemize}
\item
  \textbf{Demographic and Background Variables:}

  \begin{itemize}
  \tightlist
  \item
    Age at commission
  \item
    Commissioning source (ROTC, West Point, OCS)
  \item
    Educational background (degree field, advanced degrees)
  \item
    Prior enlisted service
  \item
    Family status (marital status, children, military spouse)
  \end{itemize}
\item
  \textbf{Organizational Context Factors:}

  \begin{itemize}
  \tightlist
  \item
    Unit morale indicators
  \item
    Leadership quality metrics
  \item
    Operational tempo
  \item
    Geographic location preferences versus assignments
  \end{itemize}
\item
  \textbf{Event Specifics:}

  \begin{itemize}
  \tightlist
  \item
    Type of separation (voluntary, involuntary)
  \item
    Reason for departure (categorized as medical/physical,
    performance-related, or misconduct)
  \item
    Career destination after Army (civilian sector, different military
    branch, retirement)
  \end{itemize}
\end{enumerate}

\paragraph{Sample \& Data Collection}\label{sample-data-collection}

The study would utilize a longitudinal design following a cohort of Army
officers from their commissioning date through either separation or a
predetermined endpoint (e.g., 10 years of service). The sample would
include:

\begin{itemize}
\tightlist
\item
  Officers commissioned between 2015-2020 across all four branch
  divisions (Armor, Logistics, Aviation, and Cyber)
\item
  Stratified sampling to ensure adequate representation of each branch
  and demographic subgroups
\item
  Target sample size of approximately 5,000 officers to allow for
  branch-specific analyses while maintaining statistical power
\end{itemize}

Data collection would involve: 1. \textbf{Administrative Data:}
Integration of personnel records from military Human Resources Command
databases, including promotion histories, performance evaluations,
assignment records, and separation codes 2. \textbf{Assessment Data:}
TAPAS scores and other psychological assessments conducted during
officer accession 3. \textbf{Survey Data:} Annual surveys to capture
time-varying factors such as job satisfaction, career intentions,
perceived leadership quality, and work-life balance 4. \textbf{Exit
Interviews:} Detailed interviews with separating officers to capture
qualitative insights on departure reasons 5. \textbf{Post-Separation
Tracking:} Limited follow-up on career trajectories of officers after
separation (where permissible)

Data would be integrated into a person-period format suitable for event
history analysis, with one record per officer per time period (quarterly
observations).

\paragraph{Analysis Approach}\label{analysis-approach}

The analysis would employ a multi-method survival analysis approach:

\begin{enumerate}
\def\labelenumi{\arabic{enumi}.}
\tightlist
\item
  \textbf{Descriptive Analysis:}

  \begin{itemize}
  \tightlist
  \item
    Kaplan-Meier survival curves to visualize retention patterns by
    branch, commissioning source, and other key variables
  \item
    Cumulative incidence functions to examine competing risks (different
    types of separation)
  \item
    Temporal patterns of attrition across career stages and historical
    periods
  \end{itemize}
\item
  \textbf{Cox Proportional Hazards Modeling:}

  \begin{itemize}
  \tightlist
  \item
    Branch-specific models to identify risk factors for attrition unique
    to each division
  \item
    Time-varying covariates to account for changing factors throughout
    officers' careers
  \item
    Stratified models to account for potential violations of the
    proportional hazards assumption
  \end{itemize}
\item
  \textbf{Discrete-Time Hazard Modeling:}

  \begin{itemize}
  \tightlist
  \item
    Logistic regression on person-period data to predict quarterly
    attrition probability
  \item
    Multinomial logistic regression to model competing risks (different
    separation types)
  \item
    Inclusion of branch-specific fixed effects and interactions
  \end{itemize}
\item
  \textbf{Sequence Analysis:}

  \begin{itemize}
  \tightlist
  \item
    Optimal matching techniques to identify typical career trajectory
    patterns
  \item
    Cluster analysis of career sequences to categorize career path
    typologies
  \item
    Association of trajectory clusters with retention outcomes
  \end{itemize}
\item
  \textbf{Machine Learning Extensions:}

  \begin{itemize}
  \tightlist
  \item
    Random forest survival models to identify non-linear relationships
    and interactions
  \item
    Survival trees to identify critical decision points in officer
    careers
  \end{itemize}
\end{enumerate}

The analysis would specifically test for: - Branch-specific retention
factors - Interaction between non-cognitive attributes and
branch-specific requirements - Time-varying effects of predictor
variables across career stages - Competing risks for different types of
separations

\paragraph{Potential Findings}\label{potential-findings}

The study could potentially yield several important findings:

\begin{enumerate}
\def\labelenumi{\arabic{enumi}.}
\tightlist
\item
  \textbf{Branch-Specific Retention Patterns:}

  \begin{itemize}
  \tightlist
  \item
    Identification of unique attrition timelines and risk periods for
    each branch (e.g., Aviation officers may show higher retention
    during specialized training periods but increased attrition after
    minimum service obligation completion)
  \item
    Branch-specific protective factors (e.g., technical certification in
    Cyber Division may reduce attrition risk by 30\%)
  \item
    Differential impact of deployment on retention across branches
  \end{itemize}
\item
  \textbf{Career Trajectory Typologies:}

  \begin{itemize}
  \tightlist
  \item
    Identification of 4-6 common career path patterns across branches
  \item
    Association between early career experiences and long-term retention
  \item
    Critical transition points where intervention would be most
    effective
  \end{itemize}
\item
  \textbf{Non-Cognitive Predictors:}

  \begin{itemize}
  \tightlist
  \item
    TAPAS dimensions most predictive of retention in each branch (e.g.,
    dominance may predict retention in Armor but not in Cyber)
  \item
    Interaction between resilience measures and operational tempo in
    predicting retention
  \item
    Changes in the predictive power of non-cognitive attributes across
    career stages
  \end{itemize}
\item
  \textbf{Competing Risks Insights:}

  \begin{itemize}
  \tightlist
  \item
    Different predictor profiles for various separation types (medical,
    performance, misconduct)
  \item
    Timing patterns showing when different types of attrition are most
    likely to occur
  \item
    Branch differences in predominant separation reasons
  \end{itemize}
\item
  \textbf{Intervention Timing:}

  \begin{itemize}
  \tightlist
  \item
    Identification of optimal timing for retention interventions by
    branch
  \item
    Critical periods where retention risk spikes across all branches
  \item
    Early warning indicators that precede separation decisions by 6-12
    months
  \end{itemize}
\end{enumerate}

\paragraph{Potential Implications}\label{potential-implications}

The findings could have several significant implications for Army talent
management:

\begin{enumerate}
\def\labelenumi{\arabic{enumi}.}
\tightlist
\item
  \textbf{Tailored Retention Strategies:}

  \begin{itemize}
  \tightlist
  \item
    Development of branch-specific retention incentives targeting the
    unique factors driving attrition in each division
  \item
    Personalized career counseling based on officer profiles and
    identified risk factors
  \item
    Timing interventions to coincide with identified high-risk periods
    for specific officer types
  \end{itemize}
\item
  \textbf{Selection and Assessment:}

  \begin{itemize}
  \tightlist
  \item
    Refinement of officer selection criteria to include non-cognitive
    attributes most strongly associated with retention in specific
    branches
  \item
    Enhanced matching of officer candidates to branches based on
    predictive profiles
  \item
    Improved identification of candidates likely to thrive in specific
    branch environments
  \end{itemize}
\item
  \textbf{Career Path Design:}

  \begin{itemize}
  \tightlist
  \item
    Restructuring of career development programs to address critical
    attrition points
  \item
    Creation of alternative career paths within branches to accommodate
    diverse motivations
  \item
    Development of ``bridge assignments'' to facilitate retention during
    high-risk transition periods
  \end{itemize}
\item
  \textbf{Leadership Development:}

  \begin{itemize}
  \tightlist
  \item
    Training programs for commanders focusing on factors within their
    control that influence retention
  \item
    Building leadership awareness of branch-specific retention
    challenges
  \item
    Cultivating leadership styles shown to enhance subordinate retention
  \end{itemize}
\item
  \textbf{Organizational Policy:}

  \begin{itemize}
  \tightlist
  \item
    Evidence-based revisions to assignment policies to optimize
    retention
  \item
    Family support programs targeted to address branch-specific
    work-life challenges
  \item
    Operational tempo management informed by retention impact models
  \end{itemize}
\item
  \textbf{Cost-Benefit Analysis:}

  \begin{itemize}
  \tightlist
  \item
    Quantification of the return on investment for various retention
    initiatives
  \item
    Prioritization of interventions based on potential retention impact
    and implementation cost
  \item
    Branch-specific benchmarking for retention targets based on
    realistic expectations
  \end{itemize}
\end{enumerate}

This study design would provide the Army with unprecedented insights
into officer career trajectories across different branches, allowing for
more strategic, evidence-based approaches to talent management and
retention.

\subsection{Sources}\label{sources}

{[}1{]} Predicting employee attrition and explaining its determinants
https://www.sciencedirect.com/science/article/pii/S0957417425001976\\
{[}2{]} Lack Of Career Development Drives Employee Attrition - Gartner
https://www.gartner.com/smarterwithgartner/lack-of-career-development-drives-employee-attrition\\
{[}3{]} How to Reduce Employee Attrition: 11 Strategies to Retain Your
\ldots{} https://www.deel.com/blog/how-to-reduce-attrition-rate/\\
{[}4{]} Prospecting the Career Development of Talents: A Survival
Analysis \ldots{}
https://www.kdd.org/kdd2017/papers/view/prospecting-the-career-development-of-talents-a-survival-analysis-perspecti\\
{[}5{]} Introducing Survival and Event History Analysis - Sequence
Analysis
https://methods.sagepub.com/book/mono/introducing-survival-and-event-history-analysis/chpt/sequence-analysis\\
{[}6{]} {[}PDF{]} Discrete-Time Methods for the Analysis of Event
Histories Author(s)
https://statisticalhorizons.com/wp-content/uploads/Allison.SM82.pdf\\
{[}7{]} therneau/survival: Survival package for R - GitHub
https://github.com/therneau/survival\\
{[}8{]} lifelines --- lifelines 0.30.0 documentation
https://lifelines.readthedocs.io\\
{[}9{]} Limitations of survival analysis - Cross Validated
https://stats.stackexchange.com/questions/242477/limitations-of-survival-analysis\\
{[}10{]} Attrition and reenlistment in the Army: Using the Tailored
Adaptive \ldots{} https://pmc.ncbi.nlm.nih.gov/articles/PMC10013530/\\
{[}11{]} Why Do Employees Stay? A Clear Career Path and Good Pay, for
\ldots{}
https://hbr.org/2017/03/why-do-employees-stay-a-clear-career-path-and-good-pay-for-starters\\
{[}12{]} {[}PDF{]} An Analysis of Career Trajectories and Occupational
Transitions
https://www.dol.gov/sites/dolgov/files/OASP/evaluation/pdf/Building\%20Better\%20Pathways\%20An\%20Analysis\%20of\%20Career\%20Trajectories\%20and\%20Occupational\%20Transitions.pdf\\
{[}13{]} {[}PDF{]} The Use of Event-History-Analysis in Career Research
https://iisg.nl/publications/moderncareer-03.pdf\\
{[}14{]} {[}PDF{]} Event History / Survival Analysis - Code Horizons
https://codehorizons.com/wp-content/uploads/2020/04/SA-Online-Sample-Materials.pdf\\
{[}15{]} Employee Attrition Explained: Types, Calculation, Impact, \&
Strategies https://www.clicdata.com/blog/employee-attrition/\\
{[}16{]} {[}PDF{]} An Introduction to Event History Analysis -
spia@uga.edu
https://spia.uga.edu/faculty\_pages/rbakker/pols8501/OxfordOneNotes.pdf\\
{[}17{]} How to Use Data to Better Understand Attrition and Retention
https://www.omnihr.co/blog/attrition-and-retention\\
{[}18{]} Dynamic work trajectories and their interplay with family over
the life \ldots{} https://pmc.ncbi.nlm.nih.gov/articles/PMC10250675/\\
{[}19{]} Toxic Culture Is Driving the Great Resignation
https://sloanreview.mit.edu/article/toxic-culture-is-driving-the-great-resignation/\\
{[}20{]} Unraveling attrition and retention: A qualitative study with
\ldots{} https://pmc.ncbi.nlm.nih.gov/articles/PMC11492032/\\
{[}21{]} {[}PDF{]} The Effects of Perstempo on Officer Retention in the
U.S. Military
https://www.rand.org/content/dam/rand/pubs/monograph\_reports/2007/MR1556.pdf\\
{[}22{]} {[}PDF{]} department of the army career engagement survey
https://talent.army.mil/wp-content/uploads/2023/09/DACES-Third-Annual-Report\_Final.pdf\\
{[}23{]} {[}PDF{]} Event History Analysis - NCRM EPrints Repository
https://eprints.ncrm.ac.uk/id/eprint/4467/1/MethodsReviewPaperNCRM-004.pdf\\
{[}24{]} {[}PDF{]} The Contribution of Life Course Research, Part II:
Causation as \ldots{}
https://emergingtrends.stanford.edu/files/original/2c61672d320e9a6fc63a6a58177377560de64b7d.pdf\\
{[}25{]} Survival Analysis Using SAS: A Practical Guide, Second Edition
https://www.goodreads.com/book/show/8333054-survival-analysis-using-sas\\
{[}26{]} Employee-Turnover-Insights-using-Survival-Analysis - GitHub
https://github.com/razamehar/Employee-Turnover-Insights-using-Survival-Analysis\\
{[}27{]} Understanding Life-Course Analysis - Doc McKee
https://docmckee.com/oer/soc/sociology-glossary/life-course-analysis-definition/\\
{[}28{]} Introducing Survival and Event History \ldots{} - Sage Research
Methods
https://methods.sagepub.com/book/mono/introducing-survival-and-event-history-analysis/back-matter/d239\\
{[}29{]} Event History Analysis - Sage Research Methods
https://methods.sagepub.com/book/mono/event-history-analysis/toc\\
{[}30{]} Event History and Survival Analysis - Sage Research Methods
https://methods.sagepub.com/book/mono/event-history-analysis-2e/toc\\
{[}31{]} {[}PDF{]} Event History Analysis - Statistical Horizons
https://statisticalhorizons.com/wp-content/uploads/EventHistoryChapter\_HandbookOfDataAnalysis.pdf\\
{[}32{]} Event History Analysis With Stata - Google Books
https://books.google.com/books/about/Event\_History\_Analysis\_With\_Stata.html?id=tolbBAAAQBAJ\\
{[}33{]} ``Event History \& Survival Analysis'' - Penn Arts \& Sciences
https://www.sas.upenn.edu/\textasciitilde allison/Allison2.html\\
{[}34{]} {[}PDF{]} Analysis of Officer Retention and Success in the US
Army by \ldots{}
https://digitalcommons.uri.edu/cgi/viewcontent.cgi?article=1886\&context=srhonorsprog\\
{[}35{]} {[}PDF{]} army officer retention: historical context
https://talent.army.mil/wp-content/uploads/pdf\_uploads/course-material/Retaining-Officer-Talent-Historical-Context.pdf\\
{[}36{]} So honestly though, is Officer retention really that bad? :
r/army
https://www.reddit.com/r/army/comments/1e7ht83/so\_honestly\_though\_is\_officer\_retention\_really/\\
{[}37{]} Sustaining a resilient force through retention - Army.mil
https://www.army.mil/article-amp/149542/sustaining\_a\_resilient\_force\_through\_retention\\
{[}38{]} {[}PDF{]} An Analysis of the Effect of Commissioning Sources on
Retention \ldots{} https://apps.dtic.mil/sti/tr/pdf/ADA424559.pdf\\
{[}39{]} Minority officers stay in the Army longer, receive fewer
promotions
https://taskandpurpose.com/news/army-minority-retention-promotion-study/\\
{[}40{]} {[}PDF{]} Analysis of First-Term Army Attrition. - DTIC
https://apps.dtic.mil/sti/tr/pdf/ADA362968.pdf\\




\end{document}
