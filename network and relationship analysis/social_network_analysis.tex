% Options for packages loaded elsewhere
% Options for packages loaded elsewhere
\PassOptionsToPackage{unicode}{hyperref}
\PassOptionsToPackage{hyphens}{url}
\PassOptionsToPackage{dvipsnames,svgnames,x11names}{xcolor}
%
\documentclass[
  letterpaper,
  DIV=11,
  numbers=noendperiod]{scrartcl}
\usepackage{xcolor}
\usepackage{amsmath,amssymb}
\setcounter{secnumdepth}{-\maxdimen} % remove section numbering
\usepackage{iftex}
\ifPDFTeX
  \usepackage[T1]{fontenc}
  \usepackage[utf8]{inputenc}
  \usepackage{textcomp} % provide euro and other symbols
\else % if luatex or xetex
  \usepackage{unicode-math} % this also loads fontspec
  \defaultfontfeatures{Scale=MatchLowercase}
  \defaultfontfeatures[\rmfamily]{Ligatures=TeX,Scale=1}
\fi
\usepackage{lmodern}
\ifPDFTeX\else
  % xetex/luatex font selection
\fi
% Use upquote if available, for straight quotes in verbatim environments
\IfFileExists{upquote.sty}{\usepackage{upquote}}{}
\IfFileExists{microtype.sty}{% use microtype if available
  \usepackage[]{microtype}
  \UseMicrotypeSet[protrusion]{basicmath} % disable protrusion for tt fonts
}{}
\makeatletter
\@ifundefined{KOMAClassName}{% if non-KOMA class
  \IfFileExists{parskip.sty}{%
    \usepackage{parskip}
  }{% else
    \setlength{\parindent}{0pt}
    \setlength{\parskip}{6pt plus 2pt minus 1pt}}
}{% if KOMA class
  \KOMAoptions{parskip=half}}
\makeatother
% Make \paragraph and \subparagraph free-standing
\makeatletter
\ifx\paragraph\undefined\else
  \let\oldparagraph\paragraph
  \renewcommand{\paragraph}{
    \@ifstar
      \xxxParagraphStar
      \xxxParagraphNoStar
  }
  \newcommand{\xxxParagraphStar}[1]{\oldparagraph*{#1}\mbox{}}
  \newcommand{\xxxParagraphNoStar}[1]{\oldparagraph{#1}\mbox{}}
\fi
\ifx\subparagraph\undefined\else
  \let\oldsubparagraph\subparagraph
  \renewcommand{\subparagraph}{
    \@ifstar
      \xxxSubParagraphStar
      \xxxSubParagraphNoStar
  }
  \newcommand{\xxxSubParagraphStar}[1]{\oldsubparagraph*{#1}\mbox{}}
  \newcommand{\xxxSubParagraphNoStar}[1]{\oldsubparagraph{#1}\mbox{}}
\fi
\makeatother


\usepackage{longtable,booktabs,array}
\usepackage{calc} % for calculating minipage widths
% Correct order of tables after \paragraph or \subparagraph
\usepackage{etoolbox}
\makeatletter
\patchcmd\longtable{\par}{\if@noskipsec\mbox{}\fi\par}{}{}
\makeatother
% Allow footnotes in longtable head/foot
\IfFileExists{footnotehyper.sty}{\usepackage{footnotehyper}}{\usepackage{footnote}}
\makesavenoteenv{longtable}
\usepackage{graphicx}
\makeatletter
\newsavebox\pandoc@box
\newcommand*\pandocbounded[1]{% scales image to fit in text height/width
  \sbox\pandoc@box{#1}%
  \Gscale@div\@tempa{\textheight}{\dimexpr\ht\pandoc@box+\dp\pandoc@box\relax}%
  \Gscale@div\@tempb{\linewidth}{\wd\pandoc@box}%
  \ifdim\@tempb\p@<\@tempa\p@\let\@tempa\@tempb\fi% select the smaller of both
  \ifdim\@tempa\p@<\p@\scalebox{\@tempa}{\usebox\pandoc@box}%
  \else\usebox{\pandoc@box}%
  \fi%
}
% Set default figure placement to htbp
\def\fps@figure{htbp}
\makeatother





\setlength{\emergencystretch}{3em} % prevent overfull lines

\providecommand{\tightlist}{%
  \setlength{\itemsep}{0pt}\setlength{\parskip}{0pt}}



 


\KOMAoption{captions}{tableheading}
\makeatletter
\@ifpackageloaded{caption}{}{\usepackage{caption}}
\AtBeginDocument{%
\ifdefined\contentsname
  \renewcommand*\contentsname{Table of contents}
\else
  \newcommand\contentsname{Table of contents}
\fi
\ifdefined\listfigurename
  \renewcommand*\listfigurename{List of Figures}
\else
  \newcommand\listfigurename{List of Figures}
\fi
\ifdefined\listtablename
  \renewcommand*\listtablename{List of Tables}
\else
  \newcommand\listtablename{List of Tables}
\fi
\ifdefined\figurename
  \renewcommand*\figurename{Figure}
\else
  \newcommand\figurename{Figure}
\fi
\ifdefined\tablename
  \renewcommand*\tablename{Table}
\else
  \newcommand\tablename{Table}
\fi
}
\@ifpackageloaded{float}{}{\usepackage{float}}
\floatstyle{ruled}
\@ifundefined{c@chapter}{\newfloat{codelisting}{h}{lop}}{\newfloat{codelisting}{h}{lop}[chapter]}
\floatname{codelisting}{Listing}
\newcommand*\listoflistings{\listof{codelisting}{List of Listings}}
\makeatother
\makeatletter
\makeatother
\makeatletter
\@ifpackageloaded{caption}{}{\usepackage{caption}}
\@ifpackageloaded{subcaption}{}{\usepackage{subcaption}}
\makeatother
\usepackage{bookmark}
\IfFileExists{xurl.sty}{\usepackage{xurl}}{} % add URL line breaks if available
\urlstyle{same}
\hypersetup{
  pdftitle={Social Network Analysis as a Method for Analyzing Career Trajectories},
  colorlinks=true,
  linkcolor={blue},
  filecolor={Maroon},
  citecolor={Blue},
  urlcolor={Blue},
  pdfcreator={LaTeX via pandoc}}


\title{Social Network Analysis as a Method for Analyzing Career
Trajectories}
\author{}
\date{}
\begin{document}
\maketitle


Social Network Analysis (SNA) provides powerful analytical tools for
understanding the complex patterns that shape career paths. By mapping
the interconnections between individuals, positions, and organizations,
researchers can gain valuable insights into career mobility, identify
influential factors in professional development, and understand the
structural aspects of career trajectories. This report examines SNA
methodology specifically in the context of analyzing career paths,
including its theoretical foundations, practical applications,
limitations, and implementation strategies.

\subsection{1. Approach Description \&
Goal}\label{approach-description-goal}

Social Network Analysis is a methodological approach that investigates
social structures through the application of network and graph theory
principles. SNA characterizes systems as networks composed of nodes
(individual actors, positions, or organizations) and the ties or edges
(relationships or interactions) that connect them{[}9{]}. When applied
to career trajectory analysis, SNA transforms individual work histories
into structured networks that reveal patterns of job transitions,
professional relationships, and career mobility pathways that might
otherwise remain hidden in conventional analyses.

The primary goals of SNA in career research include: identifying common
career pathways within professions; understanding how relationships
facilitate career advancement; recognizing ``gateway'' positions that
enable access to higher-level roles; quantifying career mobility
potential across different entry points; and developing evidence-based
career guidance by mapping the universe of likely career paths available
from specific starting positions{[}13{]}. By visualizing and analyzing
career networks, researchers can identify structural barriers to
advancement, discover high-potential career paths, and uncover the
relationships between skills acquisition and career mobility across
different demographic groups.

\subsection{2. Critical Variables}\label{critical-variables}

Social Network Analysis of career trajectories typically incorporates
several categories of variables:

\subsubsection{Network Structure
Variables}\label{network-structure-variables}

\begin{itemize}
\tightlist
\item
  \textbf{Nodes/Vertices}: Representing individuals, job positions,
  occupations, or organizations{[}9{]}
\item
  \textbf{Edges/Ties}: Representing job transitions, mentoring
  relationships, collaborations, or other professional connections
  between nodes{[}9{]}
\item
  \textbf{Network Density}: The proportion of potential connections that
  are actually present in the network
\item
  \textbf{Clustering Coefficient}: Measures how nodes tend to cluster
  together in the network
\end{itemize}

\subsubsection{Node Attribute Variables}\label{node-attribute-variables}

\begin{itemize}
\tightlist
\item
  \textbf{Demographic Data}: Age, gender, race/ethnicity, and
  educational background
\item
  \textbf{Professional Characteristics}: Skills, credentials, tenure,
  and performance metrics
\item
  \textbf{Position Attributes}: Salary level, hierarchical rank,
  department, functional area
\item
  \textbf{Industry-Specific Variables}: In military contexts, these
  might include rank, branch, deployment history
\end{itemize}

\subsubsection{Edge Attribute Variables}\label{edge-attribute-variables}

\begin{itemize}
\tightlist
\item
  \textbf{Tie Strength}: Frequency or duration of relationships
\item
  \textbf{Directionality}: Whether transitions or relationships are
  one-way or reciprocal
\item
  \textbf{Temporal Data}: When transitions or relationships occurred
\item
  \textbf{Transition Type}: Lateral moves, promotions, demotions, or
  cross-industry shifts{[}2{]}
\end{itemize}

\subsubsection{Centrality Measures}\label{centrality-measures}

\begin{itemize}
\tightlist
\item
  \textbf{Degree Centrality}: Number of direct connections a node has
\item
  \textbf{Betweenness Centrality}: How often a node lies on shortest
  paths between other nodes
\item
  \textbf{Closeness Centrality}: Average distance from a node to all
  other nodes
\item
  \textbf{Eigenvector Centrality}: Measure of influence based on
  connections to other influential nodes{[}4{]}
\end{itemize}

\subsubsection{Mobility Indicators}\label{mobility-indicators}

\begin{itemize}
\tightlist
\item
  \textbf{Career Velocity}: Rate of upward movement through positions
\item
  \textbf{Path Diversity}: Number of distinct career pathways identified
\item
  \textbf{Structural Holes}: Gaps between different career clusters that
  may represent barriers or opportunities{[}1{]}
\item
  \textbf{Social Networking Potential (SNP)}: A numeric coefficient
  representing both network size and influence capacity{[}9{]}
\end{itemize}

\subsection{3. Key Overviews}\label{key-overviews}

\textbf{Wasserman and Faust (1994) ``Social Network Analysis: Methods
and Applications''} represents a foundational text in the field of
social network analysis. The authors provide comprehensive coverage of
SNA methodology, starting with basic concepts and progressing to
advanced analytical techniques. The book systematically addresses
network data collection, representation through matrices and graphs,
measurement of centrality and prestige, identification of cohesive
subgroups, structural equivalence, blockmodels, and statistical analysis
of social networks. Particularly valuable are the chapters on dyadic and
triadic analysis, which help researchers understand the building blocks
of larger network structures. The authors also include numerous
practical examples and applications across various disciplines, making
this text essential reading for understanding the theoretical
foundations and methodological approach of SNA{[}11{]}{[}14{]}.

\textbf{``Social Network Analysis 101: Ultimate Guide'' (2023) by
Visible Network Labs} provides an accessible introduction to SNA
concepts and applications. The guide covers the theoretical background
of SNA, including the ``Strength of Weak Ties Theory,'' which posits
that weaker connections often provide more novel information and
resources compared to strong ties. It also explores the ``Structural
Hole Theory,'' which explains how individuals bridging gaps between
different groups in a network gain strategic advantages in controlling
information and resource flows. This resource serves as an excellent
entry point for those new to network analysis, offering practical
explanations of complex network concepts and their applications in
understanding social and professional relationships{[}1{]}.

\textbf{``Social Network Analysis: Understanding Centrality Measures''
(2020) by Cambridge Intelligence} offers a focused examination of
network centrality concepts, which are critical metrics for identifying
influential positions in career networks. The article explains different
centrality measures---degree, betweenness, closeness, eigenvector
centrality, and PageRank---and their interpretations in social contexts.
Understanding these measures is particularly important for career
trajectory analysis, as they help identify pivotal roles that serve as
gateways to career advancement or positions that offer greater access to
information and resources. The article presents these complex
mathematical concepts in an accessible manner, making it valuable for
researchers seeking to apply appropriate centrality measures in their
career network analyses{[}4{]}.

\textbf{``Social Network Analysis: A Methodological Introduction''
(2008) by Carter T. Butts} provides a comprehensive overview of SNA
methodology with particular attention to the mathematical foundations
and practical applications. The article covers network representation,
centrality concepts, local network structures, and dyadic and triadic
analysis. Butts explains the theoretical underpinnings of SNA while
connecting them to practical research applications across multiple
disciplines. The paper serves as a bridge between theoretical concepts
and applied research, offering guidance on methodological decisions in
network analysis. The author's discussion of statistical approaches to
network data is particularly valuable for researchers looking to move
beyond descriptive analysis to inferential testing in career trajectory
studies{[}14{]}.

\subsection{4. Mathematical Approach}\label{mathematical-approach}

Social Network Analysis employs graph theory as its mathematical
foundation, representing relationships as mathematical structures
composed of nodes (vertices) and edges. In career trajectory analysis,
this approach allows researchers to quantify and analyze patterns of
movement between positions{[}9{]}.

\subsubsection{Basic Network
Representation}\label{basic-network-representation}

A network is formally represented as a graph G = (V, E), where: - V is a
set of vertices or nodes (representing individuals, positions, or
organizations) - E is a set of edges or ties (representing relationships
or transitions between nodes)

Networks can be undirected (edges have no direction) or directed (edges
have direction, as in career progressions). In matrix form, networks are
typically represented as adjacency matrices where:

\[ A_{ij} = \begin{cases} 
1 & \text{if there is an edge from node } i \text{ to node } j \\
0 & \text{otherwise}
\end{cases} \]

For weighted networks (where ties have varying strengths, such as
frequency of transitions):

\[ W_{ij} = \text{weight of tie from node } i \text{ to node } j \]

\subsubsection{Centrality Measures}\label{centrality-measures-1}

Several mathematical formulas quantify the importance of nodes in a
network{[}4{]}:

\textbf{Degree Centrality}: For a node i in an undirected graph, degree
centrality is: \[ C_D(i) = \sum_{j \in V} A_{ij} \]

In directed graphs, we distinguish between in-degree and out-degree:
\[ C_{D}^{in}(i) = \sum_{j \in V} A_{ji} \]
\[ C_{D}^{out}(i) = \sum_{j \in V} A_{ij} \]

\textbf{Betweenness Centrality}: The proportion of shortest paths
between all node pairs that pass through node i:
\[ C_B(i) = \sum_{s \neq i \neq t} \frac{\sigma_{st}(i)}{\sigma_{st}} \]
where σ\_st is the total number of shortest paths from node s to node t,
and σ\_st(i) is the number of those paths that pass through node i.

\textbf{Closeness Centrality}: The reciprocal of the sum of the shortest
distances to all other nodes:
\[ C_C(i) = \frac{n-1}{\sum_{j \in V} d(i,j)} \] where d(i,j) is the
shortest path distance between nodes i and j.

\textbf{Eigenvector Centrality}: A measure where a node's importance
depends on the importance of its neighbors:
\[ C_E(i) = \frac{1}{\lambda} \sum_{j \in V} A_{ij} C_E(j) \] where λ is
the largest eigenvalue of the adjacency matrix A.

\subsubsection{Path Analysis and
Connectivity}\label{path-analysis-and-connectivity}

In career trajectory analysis, paths represent possible career
progressions. A path from node v₁ to vₙ is a sequence of edges
connecting a sequence of distinct vertices. The path length is the
number of edges traversed.

Key metrics include: - \textbf{Average Path Length}: The mean of
shortest path lengths between all node pairs - \textbf{Diameter}: The
maximum shortest path length in the network - \textbf{Reachability}:
Whether a path exists between two nodes

\subsubsection{Community Detection}\label{community-detection}

Career trajectory analysis often involves identifying clusters or
communities of positions that are closely connected. Modularity (Q) is a
common measure:

\[ Q = \frac{1}{2m} \sum_{ij} \left[ A_{ij} - \frac{k_i k_j}{2m} \right] \delta(c_i, c_j) \]

where m is the number of edges, k\_i is the degree of node i, c\_i is
the community of node i, and δ is the Kronecker delta function.

\subsubsection{Temporal Network
Analysis}\label{temporal-network-analysis}

For analyzing career progressions over time, researchers use temporal
network measures including: - Sequence analysis to identify typical
career patterns - Transition probability matrices to quantify the
likelihood of moving between positions - Career velocity metrics to
measure the rate of progression through hierarchical levels{[}13{]}

\subsection{5. Example Applications}\label{example-applications}

\textbf{``Using Network Analysis of Job Transitions to Inform Career
Advice'' (2022) by Axelle Clochard} applies SNA to study career mobility
patterns using U.S. job transitions data. Clochard constructs a network
where nodes represent occupations and edges represent significant flows
of workers between jobs. The research focuses on identifying career
paths from entry-level or precarious occupations that lead to stable,
high-wage employment. The study reveals that although opportunities
exist for workers with various educational backgrounds, upward mobility
prospects are generally limited for workers without a bachelor's degree.
Low-wage or shrinking occupations typically offer restricted access to
stable, high-wage employment. However, the analysis also identifies
``bright spot'' occupations that provide reliable pathways to
sustainable employment even for workers starting in lower-wage
positions. The research demonstrates how network analysis can inform
evidence-based career guidance by mapping the likely universe of career
paths available from specific starting points{[}13{]}.

\textbf{``The Road Less Traveled: Analyzing the Career Paths of Women
Athletic Directors Utilizing Social Network Analysis'' (2025) by Motley,
Jensen, Weight, and Bates} employs SNA to analyze career paths of NCAA
Division I women athletic directors (ADs) and build a hiring network
within intercollegiate athletics. The researchers tracked career changes
of current Division I women ADs to identify influential institutions and
compare career trajectories between women and men ADs. Their findings
reveal that women ADs typically navigate longer career paths within a
much sparser network compared to their male counterparts. Women who
successfully attained AD positions generally advanced through various
positions in college athletics, with certain ``hub'' institutions
serving as accelerators for their careers. The study also found that
institutional authorities often hired women at higher positions,
frequently at senior executive levels, before they became ADs. This
application demonstrates how SNA can illuminate gender differences in
career progression and identify influential organizations that serve as
critical stepping stones for advancement in specific professional
fields{[}18{]}.

\textbf{``Using social network analysis to track career trajectories of
women STEM faculty'' (2019) by Erin McCullagh et al.} applies SNA to
investigate co-authorship networks among STEM faculty, with particular
focus on understanding the career paths of women in academia. The
researchers mapped faculty publication networks to visualize
collaboration patterns and identify influential positions and
connections. For individuals, these network maps function as a ``GPS
system for career navigation,'' showing faculty members their network
positions relative to others and helping them strategize future career
moves. The study reveals how network mapping can help faculty identify
potential collaborators, understand how to strengthen their academic
position, and recognize the impact of their collaboration choices on
career trajectories. The research demonstrates the value of SNA in
identifying both structural barriers and opportunities for career
advancement in academic settings, particularly for underrepresented
groups like women in STEM fields{[}6{]}.

\textbf{``Hidden patterns: Using social network analysis to track career
trajectories of women STEM faculty'' (2019)} applies SNA to investigate
the career progression of women faculty in Science, Technology,
Engineering, and Mathematics (STEM) disciplines. The researchers used
co-authorship networks as proxies for professional relationships and
collaboration opportunities, analyzing how these networks influenced
career advancement for women academics. The study found that women
faculty in STEM fields often have different network structures compared
to their male counterparts, with potential implications for promotion,
tenure, and research opportunities. By examining publication patterns
and collaboration networks, the researchers were able to identify
specific structural challenges facing women in academic STEM careers.
The study demonstrates how SNA can reveal hidden patterns of gender
inequality embedded in professional relationships and suggests
interventions to create more equitable academic environments{[}10{]}.

\subsection{6. Critiques}\label{critiques}

Despite its analytical power, Social Network Analysis has several
limitations when applied to career trajectory research:

\textbf{Methodological Limitations}: SNA often sacrifices the rich
longitudinal structure of individual career histories to identify
broader patterns. While this trade-off enables the discovery of
structural trends, it may obscure important details of personal career
development. As noted in Clochard's research, ``the twin forces of
personal preference and structural constraints would have made it
difficult to extrapolate any kind of general trend from individual
career trajectories,'' yet these individual variations matter
significantly in real career outcomes{[}13{]}.

\textbf{Boundary Specification Problems}: Defining network boundaries
appropriately presents significant challenges in career trajectory
analysis. Researchers must make critical decisions about which actors to
include or exclude and how to demarcate system boundaries. These
decisions can significantly impact findings, potentially excluding
important influences from outside the defined network. As Wasserman and
Faust note, boundary specification and sampling decisions are
fundamental methodological challenges in network analysis{[}11{]}.

\textbf{Data Quality and Availability Concerns}: SNA requires
comprehensive data about relationships, which may be difficult to obtain
accurately. Career transition data often comes from resumes,
professional networks, or surveys, all of which have inherent biases and
reliability issues. Records may be incomplete, subject to recall bias,
or systematically missing certain types of transitions. Moreover, as
noted by Clochard, different data sources (like resumes versus
government surveys) may capture different populations and transition
patterns{[}13{]}.

\textbf{Causality Attribution Problems}: While SNA effectively reveals
patterns and correlations in career paths, it often struggles to
establish causality. The identification of common pathways does not
necessarily explain why these paths exist or what individual factors
lead to successful transitions. This limitation makes it challenging to
develop prescriptive career advice based solely on network patterns
without complementary qualitative or experimental
research{[}10{]}{[}13{]}.

\textbf{Static Representation of Dynamic Processes}: Traditional SNA
methods often provide static snapshots of networks, which may
inadequately capture the dynamic nature of career development over time.
Although temporal network analysis techniques exist, they remain less
developed than static approaches. This limitation is particularly
problematic for understanding how career networks evolve and how timing
affects career transitions{[}9{]}.

\textbf{Homophily and Selection Effects}: Networks often display
homophily (the tendency to associate with similar others), which can
make it difficult to distinguish between selection effects and genuine
influence in career patterns. In career trajectory analysis, this
manifests as uncertainty about whether certain career paths result from
institutional structures, skill requirements, or social sorting
processes based on demographic characteristics{[}19{]}.

\subsection{7. Software}\label{software}

\textbf{UCINET} is one of the most comprehensive and widely used
software packages for social network analysis, developed by Analytic
Technologies. It provides tools for matrix algebra, multivariate
statistics, and graph theory specifically designed for analyzing social
network data. UCINET includes functionality for calculating various
centrality measures, identifying subgroups, performing role and
positional analysis, and statistical testing of network hypotheses. The
software can handle networks with thousands of nodes and includes
routines for QAP regression, blockmodeling, and multidimensional
scaling. For career trajectory analysis, UCINET offers valuable tools
for identifying structural patterns in job transition networks and
quantifying the positional importance of specific roles. The software is
available as a 60-day free trial and includes the NetDraw visualization
tool, making it particularly suitable for researchers new to SNA who
need a comprehensive analysis platform{[}14{]}.

\textbf{NetworkX} is a Python package that provides robust tools for
creating, manipulating, and studying the structure, dynamics, and
functions of complex networks. It supports various network types
(directed, undirected, multi-graphs) and includes implementations of
many standard graph algorithms. NetworkX's strengths include its
integration with the broader Python data science ecosystem (including
pandas, NumPy, and matplotlib), allowing for seamless workflow with
other data analysis tools. For career trajectory research, NetworkX
excels at handling large-scale networks and implementing custom metrics
relevant to career path analysis. The package supports temporal
networks, making it suitable for analyzing career transitions over time.
Its open-source nature, extensive documentation, and active development
community make it an excellent choice for researchers with programming
experience who need flexibility in their analysis approach.

\textbf{igraph} is available as both an R package and a Python library,
providing powerful tools for network analysis with an emphasis on
efficiency, portability, and ease of use. The package is particularly
strong for large-scale network analysis thanks to its C core
implementation, which enables fast computation of complex network
metrics. The igraph package includes comprehensive functionality for
community detection, which is valuable for identifying clusters of
related positions or common career paths. For career trajectory
analysis, igraph offers excellent performance when dealing with large
job transition networks and sophisticated algorithms for path analysis.
The package's visualization capabilities are more limited than
specialized visualization tools, but it integrates well with plotting
libraries in both R and Python, making it a versatile choice for
researchers comfortable with programming.

\textbf{statnet} is a suite of R packages for network analysis,
modeling, and visualization, with particular strengths in statistical
modeling of networks. The suite includes packages such as `network' for
data manipulation, `sna' for descriptive analysis, and `ergm' for
exponential random graph modeling. For career trajectory research,
statnet is especially valuable when researchers need to move beyond
descriptive analysis to statistical inference and hypothesis testing.
The ergm package allows for modeling the probability of career
transitions based on individual attributes, structural factors, and
their interactions. This capability is particularly useful for
understanding what factors predict specific career paths or for testing
theories about career mobility barriers. Statnet is open-source,
actively maintained, and well-documented, making it an excellent choice
for researchers with statistical expertise who need rigorous inferential
capabilities.

\textbf{Gephi} is an open-source network visualization and exploration
platform designed to help researchers intuitively discover patterns and
trends in network data. Unlike the programming-based packages, Gephi
provides an interactive graphical user interface that allows users to
manipulate, analyze, and visualize networks without coding. It features
real-time visualization, supports large networks, and offers
sophisticated filtering capabilities to focus on specific network
segments. For career trajectory analysis, Gephi excels at creating
compelling visual representations of career paths and transition
patterns that can communicate findings to non-technical audiences. Its
dynamic filtering capabilities allow researchers to explore how career
networks evolve over time or differ across demographic groups. Gephi
also includes standard network metrics and community detection
algorithms, making it a good option for researchers who prioritize
visualization and interactive exploration over advanced statistical
modeling.

\textbf{Pajek} is specialized software designed specifically for
analysis and visualization of large networks. Its name means ``spider''
in Slovenian, reflecting its focus on web-like network structures. Pajek
is particularly effective for handling very large networks (with
millions of nodes) that might be computationally challenging for other
software. The program includes implementations of various decomposition,
clustering, and blockmodeling algorithms that are valuable for
identifying structural patterns in career networks. For career
trajectory analysis, Pajek's partitioning capabilities allow researchers
to identify hierarchical structures and career clusters efficiently. The
software is free for non-commercial use and has a dedicated user
community, particularly in academia. While its interface may appear less
modern than some alternatives, Pajek remains a powerful tool for
researchers dealing with extremely large career networks or those
requiring specialized network decomposition techniques{[}14{]}.

\subsection{8. Example Study Design: Social Network Analysis of U.S.
Army Officer Career
Trajectories}\label{example-study-design-social-network-analysis-of-u.s.-army-officer-career-trajectories}

\subsubsection{Key Variables}\label{key-variables}

\textbf{Node Variables (Officers and Positions)}: - \textbf{Individual
Attributes}: Officer IDs, demographics, commissioning source, entry
branch - \textbf{Position Attributes}: Rank (O-1 through O-6+),
functional branch (Armor, Logistics, Aviation, Cyber), position type
(command, staff, broadening, joint) - \textbf{Performance Indicators}:
Officer Evaluation Report (OER) ratings, awards received, selection for
competitive positions - \textbf{Educational Achievements}: Military
education level (Basic Officer Leader Course, Captain's Career Course,
Command and General Staff College, War College), civilian education
degrees - \textbf{Branch-Specific Indicators}: - \textbf{Armor}: Combat
training center rotation performance, gunnery qualification scores -
\textbf{Logistics}: Maintenance readiness rates, supply discipline
metrics - \textbf{Aviation}: Flight hours, aircraft qualifications,
accident-free operations - \textbf{Cyber}: Technical certifications,
technical assessment scores - \textbf{Non-Cognitive Attributes}:
Leadership style assessments, cognitive flexibility scores,
psychological resilience metrics

\textbf{Edge Variables (Career Transitions)}: - \textbf{Transition
Type}: Promotion, lateral move, special assignment, branch transfer -
\textbf{Timing Variables}: Time in grade, time between transitions,
career phase - \textbf{Relationship Variables}: Mentorship connections,
superior-subordinate relationships - \textbf{Selection Process}:
Competitive vs.~non-competitive assignments

\subsubsection{Sample \& Data Collection}\label{sample-data-collection}

The study will analyze career trajectory data from a cohort of U.S. Army
officers (N=5,000) who were commissioned between 2000-2010, allowing for
the tracking of at least 15 years of career progression. The sample will
include officers from all four branch divisions (Armor, Logistics,
Aviation, and Cyber) with proportional representation based on the
actual distribution in the Army.

Data collection will involve aggregating information from multiple Army
personnel databases: 1. \textbf{Official Military Personnel Files
(OMPF)}: For comprehensive career history including assignments,
promotions, and evaluations 2. \textbf{Army Training Information
System}: For education and qualification records 3. \textbf{Army Talent
Management System}: For skills inventories and competency assessments 4.
\textbf{Branch-specific performance databases}: For technical
proficiency metrics 5. \textbf{Army Mentorship Program records}: To
capture formal mentoring relationships

Additionally, a supplementary survey will be administered to a
stratified subsample (n=1,000) to collect data on informal mentoring
relationships, career satisfaction, and non-cognitive attributes not
captured in official records. To ensure data quality, the research team
will conduct verification through triangulation of multiple data sources
and follow strict data anonymization protocols to protect officer
privacy while maintaining analytical integrity.

\subsubsection{Analysis Approach}\label{analysis-approach}

The analysis will employ a multi-phase SNA approach to comprehensively
examine career trajectories:

\begin{enumerate}
\def\labelenumi{\arabic{enumi}.}
\tightlist
\item
  \textbf{Network Construction}:

  \begin{itemize}
  \tightlist
  \item
    Build a primary position transition network where nodes represent
    specific military positions and directed edges represent officer
    movements between positions
  \item
    Create a parallel officer network where nodes represent individual
    officers and edges represent professional relationships (mentorship,
    command relationships)
  \item
    Develop temporal networks to capture how career paths evolve over
    different commissioning cohorts
  \end{itemize}
\item
  \textbf{Structural Analysis}:

  \begin{itemize}
  \tightlist
  \item
    Calculate centrality measures to identify ``gateway'' positions
    critical for career advancement
  \item
    Apply community detection algorithms to identify clusters of
    commonly linked positions within and across branches
  \item
    Analyze structural holes to identify potential mobility barriers
    between career segments
  \item
    Calculate network density within and between branches to assess
    cross-branch mobility
  \end{itemize}
\item
  \textbf{Path Analysis}:

  \begin{itemize}
  \tightlist
  \item
    Identify common career trajectories using sequence analysis of
    position transitions
  \item
    Calculate transition probabilities between key positions
  \item
    Measure career velocity (rate of promotion/advancement) across
    different starting points
  \item
    Compare path diversity across branches to identify differing career
    flexibility
  \end{itemize}
\item
  \textbf{Comparative Analysis}:

  \begin{itemize}
  \tightlist
  \item
    Compare network structures across the four branches to identify
    structural differences in career paths
  \item
    Analyze differences in promotion velocity and career ceiling based
    on branch, commissioning source, and key early career positions
  \item
    Assess the impact of broadening assignments and education on career
    trajectories
  \item
    Evaluate the influence of mentorship relationships on career
    outcomes
  \end{itemize}
\item
  \textbf{Predictive Modeling}:

  \begin{itemize}
  \tightlist
  \item
    Develop exponential random graph models (ERGMs) to identify factors
    that predict successful transitions to key career positions
  \item
    Use machine learning approaches to identify combinations of factors
    associated with reaching senior ranks
  \end{itemize}
\end{enumerate}

\subsubsection{Potential Findings}\label{potential-findings}

The SNA of U.S. Army officer career trajectories may reveal several
important patterns:

\begin{enumerate}
\def\labelenumi{\arabic{enumi}.}
\item
  \textbf{Critical Path Positions}: Certain key assignments likely
  function as ``gateways'' to senior leadership, with high betweenness
  centrality in the career network. These might differ significantly
  across branches, with technical branches like Cyber potentially
  showing more diverse paths than traditional combat arms like Armor.
\item
  \textbf{Branch Mobility Patterns}: Network analysis may reveal
  structural differences in career flexibility, with some branches
  (potentially Aviation and Cyber) displaying greater specialization and
  others (potentially Logistics) showing more diverse career path
  options and cross-functional mobility.
\item
  \textbf{Mentorship Impact}: Officers with stronger mentorship
  connections (higher degree centrality in the officer relationship
  network) may show accelerated career progression, particularly in
  transitions to command positions.
\item
  \textbf{Educational Influence}: Advanced civilian education might
  create alternative pathways to senior positions, particularly in
  technical branches like Cyber, while traditional military education
  might remain more critical in Armor and Aviation.
\item
  \textbf{Performance Threshold Effects}: Network analysis may reveal
  that performance indicators function not as linear predictors but as
  threshold requirements at critical career junctures, with different
  thresholds across branches.
\item
  \textbf{Temporal Evolution}: Career networks for newer branches like
  Cyber likely show more rapid structural evolution over time compared
  to established branches like Armor with more stable traditional career
  paths.
\item
  \textbf{Cross-Branch Differences}: Aviation and Cyber branches may
  demonstrate more specialized and siloed career networks with fewer
  transitions to general leadership roles, while Armor officers might
  show greater representation in joint and strategic positions.
\end{enumerate}

\subsubsection{Potential Implications}\label{potential-implications}

The findings from this SNA study could have significant implications for
Army talent management:

\begin{enumerate}
\def\labelenumi{\arabic{enumi}.}
\item
  \textbf{Talent Management System Refinement}: Identified critical
  pathway positions could be incorporated into career planning tools,
  helping officers make more informed decisions about assignment
  preferences and helping the Army optimize officer placement.
\item
  \textbf{Mentorship Program Enhancement}: Understanding the impact of
  mentorship networks could inform the development of more effective
  formal mentorship programs, particularly for officers in branches with
  more technical specialization.
\item
  \textbf{Branch-Specific Development Programs}: Differences in career
  network structures across branches might suggest the need for
  customized professional development approaches rather than a
  one-size-fits-all career management model.
\item
  \textbf{Diversity and Inclusion Strategies}: Network analysis might
  reveal structural barriers that disproportionately affect certain
  demographic groups, informing targeted interventions to ensure
  equitable advancement opportunities.
\item
  \textbf{Cross-Functional Leader Development}: Identifying structural
  holes between branch communities could highlight opportunities for
  creating new broadening assignments that build cross-functional
  capabilities, particularly important as warfare becomes more
  multi-domain.
\item
  \textbf{Future Force Planning}: Understanding career path evolution
  over time, particularly in newer branches like Cyber, could help
  predict future senior leadership composition and identify potential
  gaps in leadership development pipelines.
\item
  \textbf{Retention Strategy Development}: Network analysis might
  identify critical career transition points with high attrition,
  allowing for targeted retention efforts at these junctures.
\item
  \textbf{Educational Investment Allocation}: Findings regarding the
  impact of various educational experiences on career trajectories could
  inform more strategic allocation of limited educational opportunity
  slots and resources.
\end{enumerate}

\subsection{Sources}\label{sources}

{[}1{]} Social Network Analysis 101: Ultimate Guide
https://visiblenetworklabs.com/guides/social-network-analysis-101/\\
{[}2{]} What Is a Career Trajectory? (And How To Create One) \textbar{}
Indeed.com
https://www.indeed.com/career-advice/career-development/what-is-career-trajectory\\
{[}3{]} Variables - Broadcom Tech Docs
https://techdocs.broadcom.com/us/en/ca-mainframe-software/performance-and-storage/ca-netmaster-network-automation/12-2/administrating/variables.html\\
{[}4{]} Social Network Analysis: Understanding Centrality Measures
https://cambridge-intelligence.com/keylines-faqs-social-network-analysis/\\
{[}5{]} SNA Software LLC - Project Management Symposium
https://pmsymposium.umd.edu/pm2023/sponsor/sna-software-llc/\\
{[}6{]} Using social network analysis to track career trajectories of
women \ldots{}
https://www.emerald.com/insight/content/doi/10.1108/edi-09-2017-0183/full/html\\
{[}7{]} Social Network Analysis - Cambridge University Press
https://www.cambridge.org/core/books/social-network-analysis/90030086891EB3491D096034684EFFB8\\
{[}8{]} {[}PDF{]} Leadership Network Analysis - Kellogg School of
Management
https://www.kellogg.northwestern.edu/faculty/uzzi/ftp/teaching\%20materials/Example\_std\_papers/LeadershipNetworks\_EMP70.pdf\\
{[}9{]} Social network analysis - Wikipedia
https://en.wikipedia.org/wiki/Social\_network\_analysis\\
{[}10{]} Using social network analysis to track career trajectories of
women \ldots{}
https://researchwith.njit.edu/en/publications/hidden-patterns-using-social-network-analysis-to-track-career-tra\\
{[}11{]} Social network analysis: methods and applications by Stanley
\ldots{}
https://archive.org/details/SocialnetworkanalysisWassermanFaust1994\\
{[}12{]} Social Network Analysis - an overview \textbar{} ScienceDirect
Topics
https://www.sciencedirect.com/topics/social-sciences/social-network-analysis\\
{[}13{]} {[}PDF{]} Using Network Analysis of Job Transitions to Inform
Career Advice
https://dspace.mit.edu/bitstream/handle/1721.1/143136/clochard\_axellecb\_sm\_tpp\_eecs\_2022.pdf?sequence=1\&isAllowed=y\\
{[}14{]} {[}PDF{]} Social Network Analysis: An Introduction - ICPSR
https://www.icpsr.umich.edu/summerprog/biblio/2012/Social\%20Network\%20Analysis\%20An\%20Introduction.pdf\\
{[}15{]} Social Network Analysis: An Introduction by Orgnet,LLC
http://www.orgnet.com/sna.html\\
{[}16{]} What are some real life use cases of network analysis? - Reddit
https://www.reddit.com/r/datascience/comments/wn9ti3/what\_are\_some\_real\_life\_use\_cases\_of\_network/\\
{[}17{]} Social Network Analysis
https://www.publichealth.columbia.edu/research/population-health-methods/social-network-analysis\\
{[}18{]} ``Women AD Career Paths'' by Mary C. Motley, Jonathan A. Jensen
\ldots{} https://scholarcommons.sc.edu/jiia/vol18/iss1/3/\\
{[}19{]} International Network for Social Network Analysis \textbar{}
INSNA https://www.insna.org\\
{[}20{]} Full article: Managers' career paths and interlocal
collaboration
https://www.tandfonline.com/doi/full/10.1080/14719037.2023.2189902\\
{[}21{]} There's an SNA review in my children's school. So I have
\ldots{} - Instagram https://www.instagram.com/reel/DJH9P\_BM\_Uh/\\
{[}22{]} Using Network Analysis to Assess Confidence in Research
Synthesis
http://ischool.illinois.edu/research/projects/career-using-network-analysis-assess-confidence-research-synthesis\\
{[}23{]} Modeling scientific workforce dynamics using social network
analysis https://reporter.nih.gov/project-details/8798219\\
{[}24{]} Social Network Analysis 1ed by Stanley Wasserman -
Books-A-Million
https://www.booksamillion.com/p/Social-Network-Analysis/Stanley-Wasserman/9780521387071\\
{[}25{]} 1 - Social Network Analysis in the Social and Behavioral
Sciences
https://www.cambridge.org/core/books/social-network-analysis/social-network-analysis-in-the-social-and-behavioral-sciences/3C435D0A16BA2D9F6566FD4A06FC6FAB\\
{[}26{]} {[}PDF{]} Social Network Analysis - Department of Economics
Cybernetics
https://www.asecib.ase.ro/mps/Social\%20Network\%20Analysis\%20\%5B1994\%5D.pdf\\
{[}27{]} Social Network Analysis: Methods and Applications - Google
Books
https://books.google.com/books?id=CAm2DpIqRUIC\&printsec=copyright\\
{[}28{]} Wasserman, S. and Faust, K. (1994) Social Network Analysis
\ldots{} https://www.scirp.org/reference/referencespapers\\
{[}29{]} SNA Review (Mainstream) -- National Council for Special
Education https://ncse.ie/sna-review-mainstream\\
{[}30{]} The SNA review: main points and recommendations
http://forsatradeunion.newsweaver.com/designtest/1p1kqgvjncj\\
{[}31{]} Collaborative Strength \& Needs Assessment (SNA) - Etio Global
https://blog.etioglobal.org/blog/collaborative-strength-needs-assessment-sna-gaining-insight-through-problems-of-practice\\
{[}32{]} Readers react to SNA criticism - FoodService Director
https://www.foodservicedirector.com/foodservice-operations/readers-react-to-sna-criticism\\
{[}33{]} Several key issues outlined for SNA contract review
http://forsatradeunion.newsweaver.com/designtest/1in1eybwdzh\\
{[}34{]} SNA review mustn't rob Peter to pay Paul - Forsa
https://www.forsa.ie/sna-review-mustnt-rob-peter-to-pay-paul/\\




\end{document}