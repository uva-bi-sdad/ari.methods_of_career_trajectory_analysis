\documentclass[./main.tex]{subfiles}
\begin{document}

The study of career trajectories represents one of the most complex challenges in organizational and workforce research, as individual career paths are influenced by a multitude of interconnected factors including personal decisions, organizational structures, market dynamics, technological changes, and broader economic conditions. Traditional analytical approaches often fall short in capturing the dynamic, non-linear nature of career progression and the emergent patterns that arise from the complex interactions between individual agents and their environments. This chapter explores six distinct but complementary modeling and simulation methodologies that have proven particularly effective in understanding and predicting career trajectory patterns: Agent-Based Modeling, which captures individual decision-making and interactions within organizational systems; Monte Carlo simulation, which addresses uncertainty and probabilistic outcomes in career progression; force structure projections, which model workforce composition and hierarchy changes over time; system dynamics modeling, which examines feedback loops and systemic influences on career development; strategic workforce planning modeling, which aligns organizational needs with individual career pathways; and training pipeline optimization, which focuses on skill development and succession planning processes.

\end{document}