\documentclass[./main.tex]{subfiles}

\begin{document}

Longitudinal and cohort analysis provides a robust framework for examining how careers evolve over time by systematically tracking individuals or groups across critical stages of professional development. This approach emphasizes three interconnected methodological pillars: longitudinal cohort studies, retention and attrition pattern analysis, and promotion board analysis. Longitudinal cohort studies focus on observing defined groups—sharing entry points, roles, or demographic traits—over extended periods to identify trends in career progression, transitions, or stagnation. Retention and attrition pattern analysis complements this by quantifying stability and turnover within organizations or professions, distinguishing between voluntary departures, involuntary exits, and systemic factors influencing career longevity. Promotion board analysis investigates the structural mechanisms governing advancement, including decision-making criteria, biases, and alignment between individual qualifications and organizational needs. 

\end{document}