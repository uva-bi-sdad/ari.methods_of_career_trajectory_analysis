\documentclass[main.tex]{subfiles}
\begin{document}

Training Pipeline Optimization represents an innovative methodological approach for analyzing career trajectories that adapts computational optimization techniques originally developed for machine learning workflows to the domain of human capital development. This method conceptualizes career progression as a dynamic pipeline where individuals move through sequential stages, roles, and skill development phases, with the goal of optimizing outcomes such as promotion probability, job satisfaction, skill acquisition, and organizational value creation[11][5]. By leveraging dynamic programming algorithms, network analysis, and machine learning techniques, this approach enables systematic identification of optimal career pathways while accounting for individual characteristics, organizational constraints, and temporal dynamics inherent in professional development trajectories[21][12].

\subsubsection{Approach Description \& Goal}

Training Pipeline Optimization for career trajectory analysis is a data-driven methodology that treats career development as a multi-stage optimization problem where individuals progress through interconnected phases of professional growth[11]. The approach borrows core concepts from machine learning pipeline optimization, where complex computational workflows are decomposed into sequential stages that can be individually optimized and collectively coordinated to achieve optimal performance[2][17]. In the career context, this translates to modeling career progression as a series of decision points involving role transitions, skill development, educational pursuits, and assignment selections that collectively determine long-term career outcomes[5].

The primary goal of this approach is to identify optimal career pathways that maximize multiple objective functions simultaneously, such as advancement potential, job satisfaction, compensation growth, and organizational value contribution[5][11]. Unlike traditional career planning methods that rely primarily on intuition or limited historical analysis, Training Pipeline Optimization employs sophisticated mathematical algorithms to evaluate thousands of potential career combinations and identify those with the highest probability of success given individual characteristics and organizational constraints[11]. The methodology is particularly valuable for large organizations seeking to optimize talent development at scale while providing personalized career guidance to individual employees[14].

\subsubsection{Critical Variables}

Training Pipeline Optimization for career trajectory analysis typically incorporates several categories of variables that capture different dimensions of professional development[1][11]. **Performance indicators** form the foundational variable category, including metrics such as evaluation scores, assessment results, and demonstrated competencies in key developmental assignments[1]. For military contexts, these include Officer Evaluation Reports (OER), Battalion Commander Assessments, and performance in Key Development (KD) positions that heavily influence promotion potential[1].

**Educational and training variables** constitute another critical category, encompassing formal education credentials, professional military education completion, skill identifiers, and participation in competitive training programs[1]. These variables often include commission source information, which research indicates has significant effects on officer survival curves, as well as specialized certifications and advanced degree attainment[1]. **Experience-based variables** capture the breadth and depth of professional assignments, including broadening assignments, joint service credit, and leadership positions that provide diverse operational experience[1].

**Demographic and personal characteristics** represent important control variables, including age, gender, race, marital status, and number of dependents, all of which research has shown to correlate with career survival and advancement patterns[1]. **Cognitive and non-cognitive attributes** form an increasingly important variable category, incorporating measures such as hardiness dimensions (control, commitment, challenge), motivation levels, complex problem-solving abilities, creative thinking capacity, and responsibility orientation[1]. Finally, **temporal and contextual variables** account for the dynamic nature of career progression, including time-in-grade requirements, promotion timeline constraints, and organizational needs that influence advancement opportunities[1].

\subsubsection{Key Overviews}

**Somavarapu \& Singh (2025)** present a comprehensive framework for optimizing employee career pathways using network analysis and machine learning at scale[11]. Their approach integrates network analysis to map complex interconnections between roles, skills, and organizational hierarchies, revealing hidden career opportunities and advancement bottlenecks that traditional hierarchical models might overlook. The methodology employs machine learning algorithms to analyze historical employee data, identifying patterns and predicting career outcomes with high accuracy while enabling personalized recommendations for individual employees. Their simulation-based research demonstrates the feasibility of using data-driven approaches for career pathway optimization, achieving 92% accuracy in predicting career transitions and providing tailored career guidance that improves employee satisfaction and engagement.

**Oentaryo et al. (2016)** introduce JobComposer, a novel data-driven approach for automating career path planning and optimization through multicriteria utility learning[5]. Their methodology treats career planning as a multicriteria optimization problem where observed career trajectories involve multiple competing objectives such as salary, job level, and other professional goals. JobComposer assembles career paths from all possible job transitions found in observed career trajectories and employs a multicriteria utility learning procedure that jointly considers multiple payoff criteria when optimizing career paths. The approach enables individuals to specify their preferred criteria and returns corresponding optimized career paths, demonstrating significant improvements over greedy baseline methods by reducing advancement time while maximizing desirability gains.

**Epistasis Lab's TPOT framework** provides foundational concepts for pipeline optimization that can be adapted to career trajectory analysis[6][15]. TPOT (Tree-based Pipeline Optimization Tool) represents a genetic programming-based automated machine learning system that optimizes sequences of feature preprocessors and machine learning models to maximize performance on supervised learning tasks. The tool intelligently explores thousands of possible pipeline configurations to identify optimal combinations, providing both automated optimization capabilities and interpretable results through generated Python code. While originally designed for machine learning workflows, TPOT's genetic programming approach and pipeline optimization methodology offer valuable insights for developing career trajectory optimization systems that can explore vast spaces of possible career combinations.

**Atkeson \& Liu (2013)** present trajectory-based dynamic programming as a method for accelerating optimization in complex decision-making environments[9]. Their approach combines local optimizations with global optimization strategies to solve problems that traditional grid-based dynamic programming cannot handle due to computational constraints. The methodology is particularly relevant for career trajectory optimization because it addresses the curse of dimensionality that arises when considering multiple career variables simultaneously across extended time horizons. Their work demonstrates how trajectory-based approaches can solve optimization problems with fewer computational resources while maintaining optimality guarantees, providing a foundation for scaling career optimization methods to large organizational contexts.

\subsubsection{Mathematical Approach}

The mathematical foundation of Training Pipeline Optimization for career trajectories builds upon dynamic programming principles adapted for multi-stage decision processes with stochastic elements[9][10]. The core optimization problem can be formulated as a Markov Decision Process where career states $$s_t$$ at time $$t$$ are defined by an individual's current role, skills, experience, and other relevant characteristics[5]. The objective is to find an optimal policy $$\pi^*$$ that maximizes expected cumulative reward over a career horizon $$T$$:

$$
\pi^* = \arg\max_\pi \mathbb{E}\left[\sum_{t=0}^T \gamma^t r(s_t, a_t)\right]
$$

where $$\gamma$$ is a discount factor, $$r(s_t, a_t)$$ represents the immediate reward for taking action $$a_t$$ in state $$s_t$$, and the expectation is taken over the stochastic career progression dynamics[10].

For multicriteria optimization, the approach extends to vector-valued rewards $$\mathbf{r}(s_t, a_t) = [r_1(s_t, a_t), r_2(s_t, a_t), \ldots, r_k(s_t, a_t)]^T$$ representing different career objectives such as advancement probability, compensation growth, and job satisfaction[5]. The multicriteria utility learning problem becomes:

$$
\max_\pi \mathbb{E}\left[\sum_{t=0}^T \gamma^t \mathbf{w}^T \mathbf{r}(s_t, a_t)\right]
$$

where $$\mathbf{w}$$ is a weight vector capturing individual preferences across different career criteria[5].

The dynamic programming solution employs Bellman's optimality principle, where the value function $$V(s)$$ represents the maximum expected future reward from state $$s$$:

$$
V(s) = \max_a \left[r(s,a) + \gamma \sum_{s'} P(s'|s,a) V(s')\right]
$$

For career trajectory optimization, the transition probabilities $$P(s'|s,a)$$ capture the likelihood of career progression outcomes given current state and chosen actions, which can be estimated from historical career data using machine learning techniques[11][21].

To address computational complexity in large state spaces, the approach incorporates dynamic micro-batching strategies adapted from pipeline parallelism[21]. Career trajectories are grouped into micro-batches of similar characteristics, enabling efficient parallel processing of optimization computations:

$$
\text{Batch}_{i,t} = \{(s_j, a_j) : j \in \mathcal{S}_i, \text{sim}(s_j, s_k) > \theta, \forall k \in \mathcal{S}_i\}
$$

where $$\mathcal{S}_i$$ represents the set of individuals in batch $$i$$, and $$\text{sim}(\cdot, \cdot)$$ is a similarity function ensuring homogeneous processing within batches[21].

\subsubsection{Example Applications}

**Oentaryo et al. (2016)** demonstrate the practical application of career path optimization through their JobComposer system, which was evaluated on real-world professional network data from online platforms[5]. Their study analyzed career trajectories across multiple industries, showing how their multicriteria optimization approach could identify career paths that simultaneously optimize for advancement speed, compensation growth, and job desirability. In one notable example, their method recommended that a Technical Officer in the Telecommunications industry remain within the same sector but transition to a smaller company, ultimately reaching a Project Leader position 70 months faster than baseline approaches while achieving higher desirability gains. The study demonstrated significant improvements over greedy baseline methods, with optimized paths typically saving 50-70 months in career progression time while maintaining or improving multiple career outcome measures.

**Somavarapu \& Singh (2025)** present a comprehensive case study applying network analysis and machine learning to employee career pathway optimization across technology, healthcare, and finance industries[11]. Their simulation-based evaluation achieved 92% accuracy in predicting career transitions, demonstrating the effectiveness of combining network-derived career pathway insights with machine learning predictions. The study revealed previously hidden career pathways through network analysis, including lateral movements between departments that traditional hierarchical models failed to identify. Their framework successfully provided personalized career development recommendations, with the recommender system improving employee satisfaction and engagement scores in simulated environments while identifying skill gaps and suggesting targeted training programs to close advancement bottlenecks.

**The U.S. Army's talent pipeline research** provides extensive empirical evidence for applying optimization approaches to military career trajectories[1]. Research by Doganca (2006) demonstrated that commission source has significant effects on survival curves of U.S. Army officers, while performance in Key Development assignments heavily influences promotion potential with target promotion rates declining from 80% at O-3 to 10% at senior levels. Studies by Spain (2020) found that military GPA serves as a strong determinant of officer success, being similar to jobs performed as officers, while research by Bartone (2013) identified specific non-cognitive attributes like hardiness-control and hardiness-commitment as significant correlates of military performance both at West Point and throughout officer careers.

**Dynamic programming applications in workforce optimization** have been extensively studied in operations research contexts, with trajectory-based approaches showing particular promise for career optimization problems[9]. Atkeson and Liu's work on trajectory-based dynamic programming has been successfully applied to problems with high-dimensional state spaces that traditional tabular methods cannot handle, demonstrating up to 10x improvements in computational efficiency while maintaining solution quality. Their approach enables solving career optimization problems that consider multiple simultaneous career variables across extended time horizons, providing practical scalability for organizational workforce planning applications where thousands of employee trajectories must be optimized simultaneously.

\subsubsection{Critiques}

Training Pipeline Optimization for career trajectory analysis faces several significant limitations that constrain its practical applicability and theoretical validity[11]. **Data quality and availability challenges** represent perhaps the most fundamental critique, as the approach requires extensive historical career data that may not exist in sufficient quantity or quality for many organizations[11]. The methodology's dependence on past career patterns may perpetuate existing biases and inequities in career advancement, particularly affecting underrepresented groups whose historical trajectories may not reflect their true potential or optimal pathways[11]. This bias propagation concern is especially problematic when machine learning models trained on biased historical data are used to recommend future career paths.

**Computational complexity and scalability issues** pose substantial practical challenges, particularly when attempting to optimize career trajectories across large organizations with diverse roles and career pathways[21]. The curse of dimensionality becomes acute when considering multiple career variables simultaneously over extended time horizons, potentially making the optimization problem intractable for real-world applications[9]. Additionally, **the assumption of stable organizational structures and career pathways** may be unrealistic in rapidly changing business environments where new roles emerge frequently and traditional career progressions become obsolete[14].

**Limited incorporation of human agency and preferences** represents another significant critique, as the approach may oversimplify the complex personal, familial, and lifestyle factors that influence career decisions[5]. The methodology's focus on quantifiable outcomes may inadequately capture qualitative aspects of career satisfaction and personal fulfillment that are crucial for individual career success[11]. Furthermore, **the dynamic nature of individual preferences and life circumstances** may not be adequately modeled in static optimization frameworks, potentially leading to recommendations that become obsolete as personal situations evolve[14].

\subsubsection{Software}

**TPOT (Tree-based Pipeline Optimization Tool)** is a Python-based automated machine learning framework that uses genetic programming to optimize machine learning pipelines[6][15]. Originally designed for feature preprocessing and model selection in supervised learning tasks, TPOT's genetic programming approach can be adapted for career trajectory optimization by treating career decisions as features and career outcomes as targets. The tool automatically explores thousands of possible pipeline configurations, making it particularly valuable for identifying optimal sequences of career actions and developmental activities. TPOT provides interpretable results through generated Python code, enabling career counselors and HR professionals to understand and modify the recommended career pathways. Its ability to handle complex multi-objective optimization problems makes it suitable for balancing competing career goals such as advancement speed, compensation growth, and job satisfaction.

**mlr3pipelines** offers a comprehensive R framework for creating modular machine learning workflows that can be effectively adapted for career trajectory analysis[7]. The package represents workflows as directed graphs where nodes (PipeOps) perform specific transformations or analyses, and edges represent data flow between operations. For career applications, this graph-based approach enables modeling complex career pathways where multiple developmental activities, role transitions, and skill acquisitions occur in parallel or sequence. The framework's flexibility allows for incorporating domain-specific career constraints and organizational policies as custom PipeOps, while its parameter management system facilitates sensitivity analysis and hyperparameter tuning for career optimization models. The package's state tracking capabilities enable monitoring intermediate results and understanding how different career interventions contribute to final outcomes.

**Auto-sklearn 2.0** represents an advanced automated machine learning system that incorporates meta-learning techniques particularly relevant for career trajectory optimization[8]. The system's hands-free AutoML approach addresses the complexity of setting up machine learning pipelines for career analysis, automatically selecting appropriate algorithms and hyperparameters based on dataset characteristics. Its meta-learning capabilities enable leveraging knowledge from previous career optimization studies to improve performance on new organizational contexts or career domains. The system's bandit strategy for budget allocation is particularly valuable for career optimization scenarios where computational resources must be allocated efficiently across multiple career pathway evaluations. Auto-sklearn 2.0's ability to work effectively under rigid time constraints makes it practical for real-time career counseling applications where rapid recommendations are needed.

**DynaPipe** is a specialized framework for optimizing multi-task training through dynamic pipelines that offers unique capabilities for career trajectory optimization[12][21]. The system's dynamic micro-batching approach enables efficient processing of career trajectories with varying lengths and complexity, addressing the challenge that different career paths may require different amounts of time and developmental activities. DynaPipe's dynamic programming-based optimization algorithms can be adapted to handle variable-length career sequences while maintaining computational efficiency. The framework's ability to handle micro-batch execution time variation through dynamic scheduling makes it particularly suitable for career optimization scenarios where different career actions have varying implementation timelines and resource requirements. Its source code availability and integration with modern machine learning frameworks facilitate customization for specific organizational career development needs.

\subsubsection{Example Study Design}

\subsubsubsection{Key Variables}

This study would examine U.S. Army officer career trajectories using the comprehensive indicator framework provided, incorporating **performance variables** including Officer Evaluation Reports (OER), Battalion Commander Assessments, and Key Development assignment performance across the four branch divisions (Armor, Logistics, Aviation, Cyber)[1]. **Educational indicators** would encompass commission source, graduate education, skill identifiers, Senior Service College completion, and branch-specific training such as Pre-Combat Checks for Aviation officers and Cyber Assessment and Selection Program (CASP) results for Cyber officers[1]. **Experience metrics** would include broadening assignments, joint service credit, and specialized positions such as BSB/CSSB S3 roles for exceptional captains[1].

**Demographic control variables** would include age, gender, race, and marital status, all identified as significant correlates of officer survival curves[1]. **Cognitive measures** would incorporate Armed Services Vocational Aptitude Battery (ASVAB) scores as predictors of job performance, while **non-cognitive attributes** would include hardiness dimensions (control, commitment, challenge), motivation levels, complex problem solving abilities, creative thinking capacity, and responsibility orientation[1]. **Temporal progression variables** would track advancement timing against branch-specific benchmarks, such as Captain at 4-5 years, Major at 10-11 years, Lieutenant Colonel at 16-17 years, and Colonel at 20-23 years depending on branch[1].

\subsubsubsection{Sample \& Data Collection}

The study would utilize a longitudinal cohort design following approximately 10,000 U.S. Army officers across all four branches (Armor, Logistics, Aviation, Cyber) over a 25-year career span from initial commissioning through potential retirement[1]. **Data collection** would integrate multiple Army personnel databases including Officer Record Briefs (ORBs), evaluation systems, assignment histories, and education records to construct comprehensive career trajectory profiles. **Stratified sampling** would ensure adequate representation across branches, with minimum samples of 2,000 officers per branch to enable branch-specific optimization modeling while maintaining statistical power for comparative analyses[11].

**Historical data extraction** would span cohorts commissioned between 1995-2010 to capture complete career trajectories, supplemented by real-time data collection for current officers to validate optimization model predictions[14]. **Assessment data integration** would incorporate standardized cognitive assessments, personality evaluations from reception day batteries, and performance metrics from Key Development assignments[1]. Special attention would be paid to **missing data patterns** that may reflect selection bias or attrition, with multiple imputation techniques employed to address systematic missingness in career progression data[11].

\subsubsubsection{Analysis Approach}

The analysis would employ a **multi-stage optimization framework** combining dynamic programming with machine learning techniques to identify optimal career pathways for each branch[21][5]. **Phase 1** would involve constructing branch-specific Markov Decision Process models where career states incorporate role, rank, experience, and skill profiles, with transition probabilities estimated using historical progression data[9][10]. **Dynamic programming algorithms** would optimize expected career outcomes across multiple objective functions including promotion probability, assignment satisfaction, and long-term retention likelihood[5].

**Phase 2** would implement **dynamic micro-batching approaches** to group officers with similar characteristics and career stages, enabling efficient parallel processing of optimization computations across the large sample[21]. **Machine learning ensemble methods** would predict career transition probabilities and outcome likelihoods, with separate models trained for each branch to capture branch-specific advancement patterns and requirements[11]. **Network analysis techniques** would identify hidden career pathways and lateral movement opportunities that traditional hierarchical models might overlook[11].

**Validation procedures** would employ cross-validation and out-of-sample testing to assess optimization model accuracy, with particular attention to model performance across demographic subgroups to identify potential bias issues[11]. **Sensitivity analyses** would examine how changes in organizational priorities or promotion policies affect optimal career pathway recommendations[14].

\subsubsubsection{Potential Findings}

The study would likely reveal **significant branch-specific differences** in optimal career progression strategies, with Aviation officers potentially benefiting from earlier specialization due to limited Key Development choices, while Cyber officers might optimize through diverse assignment portfolios given their broader KD options[1]. **Commission source effects** would likely show differential impacts across branches, with certain commissioning programs providing advantages for specific career trajectories but not others[1]. The analysis might demonstrate that **traditional linear progression models** significantly underestimate optimal career pathways, with lateral movements and timing variations offering substantial improvement in advancement probability[11].

**Non-cognitive attributes** such as hardiness-control and motivation would likely emerge as stronger predictors of long-term career success than previously recognized, particularly for challenging assignments and leadership roles[1]. The optimization models might reveal **critical career decision points** where specific choices disproportionately influence long-term outcomes, such as timing of Key Development assignments or selection of broadening experiences[1]. **Gender and demographic disparities** in optimal pathways might be identified, revealing systemic barriers or advantages that could inform policy interventions[11].

\subsubsubsection{Potential Implications}

The findings would have **immediate policy implications** for Army personnel management, potentially informing revisions to promotion timelines, assignment policies, and professional development requirements across branches[14]. **Personalized career counseling systems** could be developed based on optimization algorithms, providing officers with data-driven guidance for career decisions while accounting for individual preferences and circumstances[11][5]. The results might justify **differential career management approaches** by branch, recognizing that optimal pathways for Cyber officers may fundamentally differ from those in Armor or Aviation[1].

**Organizational implications** could include restructuring Key Development assignment allocations, modifying professional military education sequencing, and creating new lateral movement opportunities to optimize talent utilization[14]. The study might demonstrate **return on investment** for specific training programs or educational investments, informing resource allocation decisions across branches[1]. **Broader applications** could extend the methodology to other military services or civilian organizations, providing a scalable framework for evidence-based career development that balances individual aspirations with organizational needs[11].

\begin{thebibliography}{21}

\bibitem{army_indicators}
Potential Indicators. U.S. Army Branch Indicators. Internal document.

\bibitem{superannotate_ml}
SuperAnnotate. How to optimize your machine learning pipeline: Rules to consider. 2022. Available at: https://www.superannotate.com/blog/how-to-optimize-machine-learning-pipeline

\bibitem{talent_pipelines}
Innovative Approaches to Building Comprehensive Talent Pipelines: Helping to Grow a Strong and Diverse Professional Workforce. IIIS Conference Proceedings, 2015.

\bibitem{amazon_dynapipe}
Amazon Science. Optimizing multi-task training through dynamic pipelines. 2024. Available at: https://www.amazon.science/code-and-datasets/optimizing-multi-task-training-through-dynamic-pipelines

\bibitem{jobcomposer}
Oentaryo, Richard J., Xavier Jayaraj Siddarth Ashok, Ee-Peng Lim, and Philips Kokoh Prasetyo. "JobComposer: Career Path Optimization via Multicriteria Utility Learning." arXiv preprint arXiv:1809.01062 (2018).

\bibitem{tpot}
TPOT - Epistasis Lab. Available at: http://epistasislab.github.io/tpot/

\bibitem{mlr3pipelines}
Binder, Martin, Florian Pfisterer, Michel Lang, and Bernd Bischl. "mlr3pipelines: Machine Learning Pipelines as Graphs." UseR! 2019 Conference.

\bibitem{autosklearn}
Papers with Code. "Auto-Sklearn 2.0: Hands-free AutoML via Meta-Learning." 2020. Available at: https://paperswithcode.com/paper/auto-sklearn-2-0-the-next-generation

\bibitem{trajectory_dp}
Atkeson, Christopher G., and Chenggang Liu. "Trajectory-Based Dynamic Programming." In Modeling, Simulation and Optimization, pp. 1-15. Springer, 2013.

\bibitem{dynamic_programming}
Dynamic Programming. MIT Course Materials, Chapter 11. Available at: http://web.mit.edu/15.053/www/AMP-Chapter-11.pdf

\bibitem{career_optimization}
Somavarapu, Sushira, and Anand Singh. "Optimizing Employee Career Pathways Using Network Analysis and Machine Learning at Scale." International Research Journal of Modernization in Engineering Technology and Science 7, no. 4 (2025): 5266-5275.

\bibitem{dynapipe_amazon}
Amazon Science. "DynaPipe: Optimizing multi-task training through dynamic pipelines." 2025. Available at: https://www.amazon.science/publications/dynapipe-optimizing-multi-task-training-through-dynamic-pipelines

\bibitem{mlops_pipelines}
MLOps Community. "Bag of Tricks for Optimizing Machine Learning Training Pipelines." 2023. Available at: https://mlops.community/optimizing-machine-learning-training-pipelines/

\bibitem{succession_pipelines}
Cezanne HR. "Career \& succession pipelines: your 8 steps to success." 2024. Available at: https://cezannehr.com/hr-blog/2024/12/career-succession-pipelines-your-8-steps-to-success/

\bibitem{tpot_kdnuggets}
KDnuggets. "Machine Learning Pipeline Optimization with TPOT." 2022. Available at: https://www.kdnuggets.com/2021/05/machine-learning-pipeline-optimization-tpot.html

\bibitem{dynapipe_arxiv}
Jiang, Chenyu, et al. "DynaPipe: Optimizing Multi-task Training through Dynamic Pipelines." arXiv preprint arXiv:2311.10418 (2023).

\bibitem{shelf_optimization}
Shelf.io. "Optimize Machine Learning Pipelines for Faster Deployment." 2024. Available at: https://shelf.io/blog/how-to-optimize-machine-learning-pipelines-for-faster-deployment/

\bibitem{tpot_mlr}
Olson, Randal S., et al. "TPOT: A Tree-based Pipeline Optimization Tool for Automating Machine Learning." Proceedings of the Workshop on Automatic Machine Learning (2016): 66-74.

\bibitem{preprocessing_optimization}
Scientific Deep Learning Preprocessing Pipeline Optimization. Lawrence Berkeley National Laboratory, 2021.

\bibitem{restack_optimization}
Restack.io. "Advanced Training Optimization Techniques." 2025. Available at: https://www.restack.io/p/enhancing-ml-pipeline-performance-answer-advanced-training-optimization-techniques-cat-ai

\bibitem{dynapipe_eurosys}
Jiang, Chenyu, et al. "DynaPipe: Optimizing Multi-task Training through Dynamic Pipelines." Proceedings of the European Conference on Computer Systems (EuroSys), 2024.

\end{thebibliography}

\end{document}
