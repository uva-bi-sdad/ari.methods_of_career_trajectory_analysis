\documentclass[12pt]{article}
\usepackage{geometry}
\usepackage{array}
\usepackage{amsmath}
\usepackage{multirow}
\usepackage{tabularx}
\usepackage{enumitem}
\setlist{nosep}
\setlist[itemize]{leftmargin=*}

\geometry{margin=1in}

\begin{document}

\section*{Potential Indicators}

\subsection*{U.S. Army Branch Indicators}

\newcommand{\rr}{\raggedright}
\newcommand{\tn}{\tabularnewline}

\begin{tabularx}{6.5in}{|l|X|X|X|}

\hline
 & Armor & Logistics & Additional Notes \\
\hline
Education &
\begin{itemize}
    \item Commission Source
    \item Elective Training
\end{itemize}
&
\begin{itemize}
    \item Commission Source
    \item Elective Training
    \item Graduate Education
    \item Skill Identifier (SI)
    \item Secondary AOC
\end{itemize}
&
\rr
Listed examples are often competitive and selective that are optional or represent alternative paths for PME \newline
Commission source has significant effect on survival curves of U.S. army officers (Doganca 2006)
\tn
\hline
Experience &
\rr
\begin{itemize}
    \item Key Development (KD)
    \item Broadening Assignments
    \item Joint Service Credit
\end{itemize}
 &
\rr
\begin{itemize}
    \item Key Development (KD)
    \item Broadening Assignments
\end{itemize}
&
\rr
Performance in KD assignment heavily influences potential for promotion \newline
BSB/CSSB S3 position provided to exceptional captains \newline
mGPA is strong determinant as similar to jobs performed as officers (Spain 2020)
\tn
\hline
Assessments &
\rr
\begin{itemize}
\item Officer Evaluation Report (OER) 
\item Battalion Commander Assessment 
\item Commander Assessment Program 
\item Personality Assessment 
\end{itemize}
&
\rr
\begin{itemize}
    \item Battalion Commander Assessment
    \item Commander Assessment Program
    \item BOLC Assessment
    \item CCC Assessments
    \item Officer Evaluation Report (OER)
\end{itemize}
&
JDAL position or JQS points to measure JIIM (Joint Interagency Intergovernmental Multinational Environment) or diversity of experience \\
\hline
Progression &
\rr
\begin{itemize}
    \item Captain (5 years)
    \item Major (10 years)
    \item Lieutenant Colonel (16 years)
    \item Colonel (22 years)
\end{itemize}

&
\rr
\begin{itemize}
    \item Captain (4 years)
    \item Major (10.5 years)
    \item Lieutenant Colonel (16 years)
    \item Colonel (23 years)
\end{itemize}

&
\rr
Promotion is more difficult beyond O-3 (Captain) as target promotion from O-3 to O-4 is \(80\%\) to \(70\%, 50\%, 10\%\) with increasing service duration required (Doganca 2006) 
\tn
\hline
\end{tabularx} \\

\begin{tabularx}{6.5in}{|l|X|X|X|}
\hline
 & \multicolumn{1}{|c|}{Aviation} & \multicolumn{1}{c|}{Cyber} & \multicolumn{1}{c|}{Additional Notes} \\
\hline
Education &
\rr
\begin{itemize}
    \item Senior Service College (SSC)
    \item Pre-Combat Checks (PCC)
    \item Elective Training
\end{itemize}
&
\rr
\begin{itemize}
    \item Senior Service College (SSC)
    \item Elective Training
    \item Commission Courses
\end{itemize}
&
\rr
The options are often competitive and elective training is highly recommended.
\tn
\hline
Experience &
\rr
\begin{itemize}
    \item Key Developments (KD)
    \item Duty Assignments
\end{itemize}
&
\rr
\begin{itemize}
    \item Key Developments (KD)
    \item Duty Assignments
\end{itemize}
&
\rr
Emphasizing harder assignments and work done in the assignments over time doing them \\
For Aviation the choice of KD is a lot smaller than the choices given to Cyber
\tn
\hline
Assessments &
 & 
 \rr
\begin{itemize}
    \item Cyber Aptitude and Talent Assessment (CATA)
    \item Cyber Test (CT)
    \item Cyber Assessment and Selection Program (CASP)
\end{itemize}
&
\rr
Didn't find this information in either of the docs that were sent but did some research on my own and found this information (Morris 2020) \\
Was unable to find anything specific for Aviation
\tn
\hline
Progression &
\rr
\begin{itemize}
    \item Captain (4 years) 
    \item Major (10 years) 
    \item Lieutenant Colonel (16 years) 
    \item Colonel (20 years)
\end{itemize}
&
\rr
\begin{itemize}
    \item Captain (4 years) 
    \item Major (11 years) 
    \item Lieutenant Colonel (17 years) 
    \item Colonel (22 years)
\end{itemize}
&
\tn
\hline
\end{tabularx}


\subsection*{Additional Indicators}


\begin{tabularx}{6.5in}{|l|X|}
\hline
 & Additional Notes \\
\hline
\rr
Armed Services Vocational Aptitude Battery (ASVAB) & Major cognitive abilities measures are useful in predicting job performance (Ree 1994) \\
\hline
Age & Higher age has correlation with survival curve of U.S. army officer (Doganca 2006) \\
\hline
Race & African Americans have correlation with survival curve of U.S. army officer (Doganca 2006) \\
\hline
Gender & Females has correlation with survival curve of U.S. army officer (Doganca 2006) \\
\hline
Marital Status & Spousal dependents have correlation with survival curve of U.S. army officer (Doganca 2006) \\
\hline
\end{tabularx}

\subsection*{Non-Cognitive Attributes}

\begin{tabularx}{6.5in}{|l|X|}
\hline
 & Additional Notes \\
\hline
Hardiness-control & Significant correlation with military performance both at West Point and as an officer as measured by Reception Day battery of tests (Bartone 2013) \\
\hline
Hardiness-commitment & Significant correlation with military performance both at West Point and as an officer (Bartone 2013) \\
\hline
Hardiness-challenge & Do not perform well in conventional structured military and leadership tasks (Bartone 2013) \\
\hline
Motivation & Motivation was strongly associated with job performance (Doganca 2006) \\
\hline
Complex problem solving & Complex problem solving was consistent predictor of continuance and advancement (Zaccaro 2012) \\
\hline
Creative thinking & Creative thinking was consistent predictor of continuance and advancement (Zaccaro 2012) \\
\hline
Responsibility & Responsibility was consistent predictor of continuance and advancement (Zaccaro 2012) \\
\hline
\end{tabularx}
\end{document}