\documentclass[../main.tex]{subfiles}

\usepackage{amsmath}
\usepackage{amsfonts}
\usepackage{amssymb}
\usepackage{mathtools}
\usepackage{booktabs}
\usepackage{array}
\usepackage{longtable}
\usepackage{graphicx}
\usepackage{hyperref}
\usepackage{natbib}

\begin{document}

Job classification forms the foundation for analyzing and understanding career trajectories across organizations. This comprehensive framework enables researchers and practitioners to systematically categorize positions based on their relative value, required competencies, and organizational impact, providing valuable insights into career pathways and progression patterns.

\subsubsection{Approach Description \& Goal}

Job classification is a systematic process of categorizing jobs into different ranks based on their relative worth to an organization. It involves evaluating positions according to standardized criteria rather than the individuals occupying them \cite{aihr2024}. This approach serves multiple purposes: establishing fair and equitable compensation structures, creating clear organizational hierarchies, defining career progression pathways, enabling succession planning, and supporting consistent personnel decisions \cite{aihr2024, bamboohr2024}. When applied to career trajectory analysis, job classification provides a structured framework for understanding how individuals move between positions over time, identifying common pathways to success, and recognizing critical transition points in career development. This methodical approach allows organizations to develop evidence-based talent management strategies and helps individuals make informed career decisions based on established progression patterns.

\subsubsection{Critical Variables}

Job classification systems typically incorporate several key variable categories that serve as inputs to the evaluation process:

\subsubsubsection{Knowledge and Skills}
This dimension encompasses the education, experience, specialized training, technical capabilities, and expertise required to perform a job effectively \cite{peoplebox2024, synergogy2024}. The Hay System, a popular classification method, refers to this as "Know-How," defined as the "sum total of every kind of knowledge and skill, however acquired, needed for acceptable job performance" \cite{waterloo2024}. This includes practical procedures, specialized techniques, and professional knowledge.

\subsubsubsection{Problem-Solving Complexity}
This category evaluates the nature and complexity of challenges typically encountered in a position and the level of analytical, creative, and strategic thinking required to address them \cite{peoplebox2024, synergogy2024}. The Hay System measures this as "the amount and nature of thinking required in the job in the form of analyzing, reasoning, evaluating, creating, using judgment, forming hypotheses, drawing inferences, and arriving at conclusions" \cite{waterloo2024}.

\subsubsubsection{Accountability and Responsibility}
This dimension assesses the impact of the position on organizational outcomes, including resource management authority, decision-making scope, and the consequences of errors \cite{deel2024, peoplebox2024}. Variables include budget responsibility, span of control, signature authority, and policy influence \cite{wm2024}.

\subsubsubsection{Organizational Context}
This category considers the position's place within the broader organizational structure, including reporting relationships, peer positions, stakeholder interactions, and governance responsibilities \cite{wm2024}. It examines how the role interfaces with internal and external constituencies.

\subsubsubsection{Working Conditions}
This dimension evaluates the physical environment, stress factors, hazards, and other contextual elements that affect job performance and requirements \cite{synergogy2024}. This may include facility responsibility, equipment operation, and unique environmental challenges.

\subsubsection{Key Overviews}

\subsubsubsection{Occupational Classifications: A Machine Learning Approach (Ikudo, Lane, Staudt, and Weinberg, 2018)}
This National Bureau of Economic Research working paper examines the potential for using machine learning approaches to automatically classify job titles into standardized occupational categories. The authors develop a hierarchical occupation classification system based on individuals' relationships to universities (faculty, students, staff) and their functions, then test various algorithms to automatically classify job titles. Their findings indicate that random forest classifiers perform best, achieving accuracy rates around 90\% for common job titles. The researchers conclude that machine learning approaches show promise for efficiently classifying the many job titles that have relatively few people in them, potentially yielding substantial cost savings while maintaining classification accuracy. However, they emphasize that entirely algorithmic approaches remain insufficient, and human oversight continues to be necessary \cite{ikudo2018}.

\subsubsubsection{Job Classification: A Review on Data, Features, and Methods (2021)}
This comprehensive review examines job classification as both a process to categorize positions and a system to recommend jobs to candidates based on specific criteria. The paper analyzes various data types used in classification, including nominal and numerical data, and explores key features that drive effective job classification, such as academic performance, technical skills, and demographic factors. The authors review multiple classification methods, ranging from traditional approaches to advanced machine learning techniques like Naive Bayes classifiers. The study highlights that CGPA (Cumulative Grade Point Average) emerges as one of the most frequently used and significant features in job classification and recommendation systems, demonstrating the continued importance of academic performance in professional classification schemes despite the growing emphasis on practical skills and experience \cite{jeesr2021}.

\subsubsubsection{The Hay Group Guide Chart-Profile Method of Job Evaluation}
This foundational document outlines the Hay System, one of the most widely used job classification methods in North America and Europe. The system evaluates positions using three key factors: Know-how (skills and expertise), Problem-solving (complexity of thinking required), and Accountability (impact on organizational outcomes). The guide explains the unique numerical scale employed by the method, which follows a geometric rather than arithmetic pattern to properly represent proportional differences in job size. The approach is based on Weber's Law, which states that perceptions of differences between objects are relative to their magnitude rather than absolute. This "step difference" principle, set at 15\% in the Hay System, ensures consistent relativity in judgments regardless of where they fall on the classification scale, providing a scientifically grounded basis for establishing the relative worth of different positions within an organization \cite{haygroup2024}.

\subsubsubsection{Job Level Classification: The Complete Guide (2025)}
This comprehensive guide provides a practical framework for implementing job level classification systems in modern organizations. It outlines typical classification levels from entry-level to executive positions and explains how job families group related roles that share similar functions, responsibilities, and required skills. The guide describes the development process for a classification system, beginning with thorough job analysis and progressing through defining job families, establishing competency models, and creating career progression pathways. It emphasizes the importance of considering multiple criteria when classifying positions, including industry context, functional responsibilities, required competencies, organizational level, and compensation structures. The document serves as a practical roadmap for organizations seeking to implement or refine their job classification systems to support strategic talent management objectives \cite{deel2024}.

\subsubsection{Mathematical Approach}

Job classification employs several mathematical approaches, with point-factor methods being among the most common. These systems assign numerical values to job attributes according to predetermined scales, enabling systematic comparison across positions.

\subsubsubsection{Point-Factor Method}
The fundamental formula for the Point-Factor Method is:

\[
\text{Total Job Points} = \sum \text{points for (Skills + Responsibilities + Effort + Working Conditions)}
\]

Each factor is further broken down into subfactors with specific point scales. For example:

\begin{align}
\text{Skills} &= \text{Education points} + \text{Experience points} + \text{Technical expertise points} \\
\text{Responsibilities} &= \text{Supervision points} + \text{Financial impact points} + \text{Decision-making points} \\
\text{Effort} &= \text{Mental demand points} + \text{Physical demand points} \\
\text{Working Conditions} &= \text{Environment points} + \text{Hazard points}
\end{align}

A job with higher total points ranks higher in the classification hierarchy and typically commands higher compensation \cite{synergogy2024}.

\subsubsubsection{Hay Method Geometric Scale}
The Hay Method uses a geometric progression rather than an arithmetic one, with points values following a percentage progression (e.g., 100, 115, 132, 152). Each step represents a 15\% increase, based on Weber's Law:

\[
P_2 = P_1 \times (1 + S)
\]

Where:
\begin{itemize}
\item \(P_1\) is the initial point value
\item \(P_2\) is the next point value
\item \(S\) is the step difference (0.15 or 15\%)
\end{itemize}

This ensures that proportional differences between job levels remain consistent regardless of their absolute size. For example, an increase from 100 to 115 points represents the same proportional change as an increase from 1000 to 1150 points \cite{haygroup2024}.

\subsubsubsection{Machine Learning Classification}
Modern job classification increasingly employs machine learning algorithms, particularly for automatic classification of job titles. The general approach follows:

\begin{enumerate}
\item Feature extraction from job titles/descriptions (using techniques like TF-IDF)
\item Model training using labeled data (jobs with known classifications)
\item Classification of new job titles using the trained model
\end{enumerate}

Random forest classifiers have shown particular promise, with the mathematical framework involving:
\begin{itemize}
\item Creation of multiple decision trees using bootstrap sampling
\item Feature selection at each node using a random subset of features
\item Classification by majority vote across all trees in the forest
\end{itemize}

The accuracy of such models can be expressed as:

\[
\text{Accuracy} = \frac{\text{True Positives} + \text{True Negatives}}{\text{Total Classifications}}
\]

Studies have shown accuracy rates around 70-90\% for automated job classification systems \cite{ikudo2018, jeesr2021}.

\subsubsection{Example Applications}

\subsubsubsection{JobHop: A Large-Scale Dataset of Career Trajectories (2022)}
This study introduces a comprehensive dataset of real-world occupation transitions that enables detailed analysis of career trajectories and labor market dynamics. The researchers leveraged the standardized ESCO (European Skills, Competences, Qualifications and Occupations) taxonomy to analyze career mobility patterns, job stability across sectors, and the influence of education and career breaks on occupational transitions. The study explores multiple dimensions of the Flemish labor market, including the impact of career interruptions on subsequent job opportunities, patterns of mobility within and across sectors, and the quantifiable influence of university degrees on career pathways. The structured classification approach allows for international comparisons and integration with other labor market data sources, demonstrating how job classification frameworks can serve as a foundation for sophisticated career trajectory analysis. The researchers highlight the dataset's potential applications for career path prediction, skill gap analysis, workforce planning, and data-driven career guidance tools \cite{jobhop2022}.

\subsubsubsection{Occupational Classifications: A Machine Learning Application to Job Titles (Ikudo et al., 2018)}
This empirical study applies machine learning techniques to automatically classify job titles from university human resources records into standardized occupational categories. The researchers developed a hierarchical classification system based on individuals' relationships to universities and their functions, then tested various algorithms including random forests, support vector machines, and neural networks. Their analysis found that random forest classifiers achieved the highest accuracy rates (approximately 90\%) for common job titles. The authors demonstrate that machine learning approaches can significantly reduce the costs associated with manual classification while maintaining acceptable accuracy levels for most positions. However, they note that completely automated approaches remain insufficient for certain complex or unique job titles, suggesting a hybrid model that combines algorithmic classification with human review for challenging cases. The study provides concrete evidence of how computational approaches can enhance traditional job classification methods in large-scale administrative datasets \cite{ikudo2018}.

\subsubsubsection{A Machine Learning Approach to Job Seeker Profile Classification Based on Social Media Data}
This innovative study applies job classification techniques to social media data to automatically categorize job seekers based on their online presence. The researchers employed a Naive Bayes classifier algorithm in combination with W-IDF (Weighted-Inverse Document Frequency) weighting to analyze Twitter data and classify users' personality traits and professional orientations. The classification system categorized individuals according to their expressed interests, communication patterns, and content sharing behaviors, creating profiles that could be matched to suitable job categories. The study achieved an average accuracy of over 90\% with certain division ratios, demonstrating the potential for applying job classification frameworks to non-traditional data sources. This application shows how classification techniques can be extended beyond formal job titles to encompass broader aspects of career development, including personality traits and behavioral tendencies that influence career trajectories \cite{jeesr2021}.

\subsubsubsection{Job Recommender System Using Naïve Bayes Classification}
This applied research project developed a job recommendation system that classifies and suggests suitable positions based on candidates' profiles and qualifications. The researchers employed a Naive Bayes classifier to analyze multiple features including academic performance, educational background, professional experience, publications, projects, and grants. The system achieved high accuracy rates (91.33-92.74\% depending on division ratio) in matching candidates to appropriate positions based on their profiles. The study demonstrates how classification algorithms can be applied to career guidance by identifying patterns in successful career paths and recommending similar trajectories to individuals with corresponding qualifications and attributes. This application illustrates how job classification frameworks can transition from descriptive tools to predictive models that actively shape career development decisions \cite{jeesr2021}.

\subsubsection{Critiques}

Despite its widespread adoption, job classification as an analytical approach faces several significant limitations and criticisms:

\subsubsubsection{Subjectivity and Bias}
Despite attempts to create objective systems, job classification inherently involves subjective judgments that may introduce bias. The interpretation of job descriptions, the weighting of factors, and the evaluation of position requirements all contain elements of subjectivity that can undermine the system's objectivity \cite{haygroup2024}. This subjectivity may perpetuate existing biases in how different types of work are valued, potentially disadvantaging certain job categories or worker demographics.

\subsubsubsection{Resource Intensity}
Traditional job classification methods require substantial investments of time, expertise, and organizational resources. Manual classification processes are labor-intensive and expensive, particularly for large organizations with numerous positions \cite{ikudo2018}. Even when using automated approaches, significant upfront investment is needed to develop and validate classification frameworks, train evaluators, and maintain the system over time.

\subsubsubsection{Static Nature vs. Dynamic Workplace}
Job classification systems often struggle to keep pace with rapidly evolving job roles in dynamic work environments. The standardized frameworks may not adequately capture emerging skills, hybrid roles, or innovative working arrangements that characterize modern workplaces \cite{deel2024}. This rigidity can result in outdated classifications that fail to reflect the actual work being performed, limiting the approach's effectiveness for analyzing contemporary career trajectories.

\subsubsubsection{Algorithmic Limitations}
While machine learning approaches offer promising efficiency gains, they still face significant limitations. As noted by Ikudo et al., even advanced algorithms achieve accuracy rates that necessitate human oversight, particularly for uncommon or specialized job titles \cite{ikudo2018}. The "black box" nature of some algorithms also raises questions about transparency and fairness in classification decisions, potentially introducing new forms of bias while attempting to eliminate traditional ones.

\subsubsubsection{Oversimplification of Complex Work}
Critics argue that job classification systems inevitably reduce the complexity and nuance of actual work to standardized categories and numerical values. This reductionist approach may fail to capture important qualitative aspects of jobs that influence career development but resist easy quantification \cite{jeesr2021}. The focus on formal job attributes may also overlook informal roles, tacit knowledge, and relationship dynamics that significantly impact career trajectories.

\subsubsection{Software}

\subsubsubsection{ONETr (R Package)}
ONETr is an R package that facilitates interaction with the O*NET API, providing researchers with programmatic access to the O*NET database—America's primary source of occupational information. The package enables users to search occupational data based on keywords or O*NET-SOC codes and parse the XML output into structured list objects for analysis. Users must register for an account with O*NET Web Services to obtain the necessary API credentials. ONETr includes functions to extract specific occupational data components such as skills, abilities, work activities, and job zones, making it particularly valuable for researchers analyzing career pathways and occupational requirements. The package supports both exploratory analysis of occupational characteristics and systematic comparison across job categories, providing a foundation for evidence-based career trajectory research \cite{onetr2024}.

\subsubsubsection{labourR (R Package)}
labourR is a specialized R package that performs occupational coding for multilingual free-form text using the ESCO hierarchical classification model. Initially developed to analyze work experience histories from Curricula Vitae, the package generates term frequency-inverse document frequency statistics for terms in the ESCO occupations corpus and scores input queries based on matched terms. Classification is performed through a plurality vote in the corresponding hierarchical level of the ESCO ontologies with the highest score. The package includes the complete ESCO corpus and ESCO-to-ISCO mappings, supports multilingual text input, and provides fully vectorized, memory-efficient computations. labourR is particularly valuable for researchers analyzing career trajectories across international contexts or working with non-standardized job title data from diverse sources \cite{labourr2024}.

\subsubsubsection{O*NET Web Services}
O*NET Web Services provides programmatic access to the O*NET database, containing standardized and occupation-specific descriptors for hundreds of occupations. This comprehensive platform includes detailed information on the knowledge, skills, abilities, work activities, and work contexts associated with each occupation in the O*NET-SOC taxonomy. The system is regularly updated through surveys of workers across occupations, ensuring currency and relevance. O*NET Web Services enables researchers to access this rich occupational data programmatically, supporting sophisticated analyses of career requirements, progression patterns, and occupational transitions. The service requires registration and API credentials but provides free access to this authoritative occupational information, making it an essential resource for career trajectory research \cite{onet2024, onetonline2024}.

\subsubsubsection{Deel Engage}
Deel Engage includes career leveling functionality as part of its broader HR management platform, specifically designed to implement and maintain job classification systems. The software supports the creation of job leveling matrices that evaluate and assign levels to different roles within an organization based on customizable criteria such as required competencies, responsibilities, and impact. The platform enables organizations to define clear job titles, develop consistent salary structures, and establish visible career progression paths for employees. Deel Engage integrates job classification with broader talent management functions, allowing organizations to connect classification frameworks to performance management, compensation, and development planning. The system supports both traditional classification approaches and more flexible models adapted to evolving workplace structures \cite{deel2024}.

\subsubsection{Example Study Design: Job Classification Analysis of U.S. Army Officer Career Trajectories}

\subsubsubsection{Key Variables}
This study will utilize a multi-dimensional job classification framework tailored to military contexts, incorporating the following key variables:

\paragraph{Functional Competencies} Branch-specific technical skills and knowledge (e.g., armor tactical proficiency, logistics planning capabilities, aviation operational expertise, cyber defense competencies) that define core performance requirements within each division.

\paragraph{Leadership Dimensions} Command responsibilities, scope of authority, personnel management requirements, and strategic influence measured across organizational levels from platoon to division.

\paragraph{Problem-Solving Complexity} Tactical, operational, and strategic decision-making requirements, including time pressure, information ambiguity, and consequence severity typical of different positions.

\paragraph{Organizational Impact} Measurable outcomes attributable to the position, including resource management scope, mission criticality, and institutional influence.

\paragraph{Career Velocity Indicators} Time-in-grade metrics, promotion rates relative to peers, selection for prestigious assignments, and advanced educational achievements.

\paragraph{Non-Cognitive Attributes} Leadership adaptability, stress resilience, cultural intelligence, and other psychological factors that influence performance across different military contexts.

\subsubsubsection{Sample \& Data Collection}
The study will employ a stratified longitudinal sampling approach that balances representation across the four branch divisions (Armor, Logistics, Aviation, and Cyber) while ensuring sufficient career span coverage:

\paragraph{Sample Structure} 400 officers (100 per branch) with complete career records spanning at least 15 years of service, stratified by commissioning source (USMA, ROTC, OCS) and entry year cohorts (2005-2010).

\paragraph{Data Sources} Integrated personnel records including Officer Evaluation Reports (OERs), promotion board results, assignment histories, training records, and educational achievements from Army Human Resources Command databases.

\paragraph{Supplementary Data Collection} Semi-structured interviews with 40 senior officers (10 per branch) to validate classification frameworks and provide context for quantitative findings.

\paragraph{Temporal Coverage} Historical data spanning 2005-2025 to capture complete career trajectories, including critical transition points between company and field-grade ranks.

\paragraph{Classification Reference Data} Expert panel assessments of position classifications for standardization across branches, including Hay Method evaluations of knowledge requirements, problem-solving complexity, and accountability levels.

\subsubsubsection{Analysis Approach}
The analysis will implement a multi-method job classification strategy that combines traditional evaluation methods with advanced analytical techniques:

\paragraph{Position Classification Development} Create a standardized Army-specific job classification framework using modified Hay Method criteria, with branch-specific weighting of factors relevant to each division's unique requirements.

\paragraph{Career Pattern Identification} Apply sequence analysis techniques to identify common career trajectories within and across branches, mapping the progression through classified position types.

\paragraph{Transition Analysis} Employ Markov models to calculate transition probabilities between job classifications, identifying critical junctures and career acceleration points.

\paragraph{Success Predictor Modeling} Develop random forest classification models to identify early career indicators that predict successful progression to senior ranks, with separate models for each branch division.

\paragraph{Cross-Branch Comparison} Conduct comparative analysis of classification structures and career velocity across branches, controlling for cohort effects and external factors.

\paragraph{Validation Testing} Apply the classification framework to a hold-out sample of 100 officers to validate prediction accuracy and refine the model.

\subsubsubsection{Potential Findings}
This job classification analysis is expected to yield several significant insights into Army officer career trajectories:

\paragraph{Differentiated Success Pathways} The analysis will likely reveal distinct optimal career sequences for each branch division, with significant variation in the timing and nature of key developmental positions that predict successful progression to senior ranks.

\paragraph{Critical Classification Transitions} We expect to identify specific job classification levels where career velocity diverges most significantly between high-performing and average officers, potentially highlighting previously unrecognized career decision points.

\paragraph{Branch-Specific Competency Weightings} The classification framework will likely demonstrate that the relative importance of technical expertise versus leadership competencies varies substantially across branches, with Cyber placing higher value on technical specialization compared to more leadership-weighted branches like Armor.

\paragraph{Cross-Pollination Effects} The analysis may reveal the career impact of cross-branch assignments, potentially showing that officers who temporarily serve in positions classified within other branches gain valuable perspective that accelerates their subsequent advancement.

\paragraph{Non-Cognitive Attribute Influence} We anticipate finding that certain non-cognitive attributes correlate strongly with successful navigation of specific job classification transitions, with different attributes being more predictive in different branch contexts.

\subsubsubsection{Potential Implications}
The findings from this job classification study would have several significant implications for Army talent management and officer development:

\paragraph{Tailored Development Programs} Results could inform the creation of branch-specific career development roadmaps that highlight optimal timing and sequencing of key developmental positions, enabling more strategic career planning for both officers and their mentors.

\paragraph{Classification-Based Talent Management} The Army could implement a more sophisticated talent management approach that matches officers to positions based not only on technical qualifications but also on demonstrated success patterns in navigating similar classification transitions.

\paragraph{Recruitment Refinement} Findings about the predictive value of early career indicators could refine recruitment and commissioning standards, potentially with branch-specific emphasis on attributes most predictive of success in each domain.

\paragraph{Training Investment Optimization} Understanding which competencies drive successful transitions between classification levels would allow more targeted investment in developing those specific capabilities at the appropriate career stages.

\paragraph{Cross-Branch Development Strategy} Insights about the value of cross-branch experience could inform a more deliberate approach to developmental assignments outside an officer's primary branch, potentially creating a more flexible and adaptable officer corps.

\paragraph{Promotion System Refinement} Classification-based understanding of career trajectories could lead to reforms in the promotion system to better recognize and reward developmental diversity while maintaining branch-specific expertise.

\bibliographystyle{apalike}
\bibliography{references}

\begin{thebibliography}{99}

\bibitem{aihr2024}
AIHR. (2024). Job Classification: A Practitioner's Guide. Retrieved from https://www.aihr.com/blog/job-classification/

\bibitem{bamboohr2024}
BambooHR. (2024). Understanding Job Classification. Retrieved from https://www.bamboohr.com/resources/hr-glossary/job-classification

\bibitem{cpshr2024}
CPSHR. (2024). Job Classification: Aligning Roles and Compensation. Retrieved from https://www.cpshr.us/blog-article/job-classification/

\bibitem{deel2024}
Deel. (2024). The Complete Guide to Job Level Classification: Examples, Criteria. Retrieved from https://www.deel.com/blog/job-level-classification/

\bibitem{haygroup2024}
Hay Group. (2024). The Hay Group Guide Chart-Profile Method of Job Evaluation. Retrieved from https://professionals.lincolnshire.gov.uk/downloads/file/576/hay-group-je-method

\bibitem{ikudo2018}
Ikudo, A., Lane, J., Staudt, J., \& Weinberg, B. A. (2018). Occupational Classifications: A Machine Learning Approach. National Bureau of Economic Research Working Paper No. 24951.

\bibitem{indeed2024}
Indeed. (2024). How Does Job Classification Work? (With Definition and Types). Retrieved from https://www.indeed.com/career-advice/career-development/how-does-job-classification-work

\bibitem{jeesr2021}
JEESR. (2021). Job classification: A Review on Data, Features, and Methods. Journal of Electrical and Electronic Systems Research, 21.

\bibitem{jobhop2022}
JobHop. (2022). A Large-Scale Dataset of Career Trajectories. arXiv preprint arXiv:2505.07653.

\bibitem{labourr2024}
labourR. (2024). Introduction to labourR. CRAN. Retrieved from https://cran.r-project.org/web/packages/labourR/vignettes/labourR\_blogpost.html

\bibitem{lincolnhr2009}
Lincoln HR. (2009). Internal Equity Workshop – Hay Group Work Measurement Process. Retrieved from https://lincolnhr.org/wp-content/uploads/2009/12/Internal-Equity-Workshop-Hay-Group.pdf

\bibitem{onet2024}
O*NET Resource Center. (2024). About O*NET. Retrieved from https://www.onetcenter.org/overview.html

\bibitem{onetonline2024}
O*NET OnLine. (2024). O*NET Overview Help. Retrieved from https://www.onetonline.org/help/onet/

\bibitem{onetr2024}
ONETr. (2024). A small R package for interacting with the O*NET API. GitHub. Retrieved from https://github.com/eknud/ONETr

\bibitem{pelago2024}
Pelago. (2024). Job Classification: HR Terms Explained. Retrieved from https://www.pelagohealth.com/resources/hr-glossary/job-classification/

\bibitem{peoplebox2024}
Peoplebox. (2024). Hay System 101 - Job Evaluation Method [2024 Updated]. Retrieved from https://www.peoplebox.ai/blog/hay-system/

\bibitem{synergogy2024}
Synergogy. (2024). Job Evaluation Using Point Factor. Retrieved from https://synergogy.com/job-evaluation-using-point-factor/

\bibitem{ucsb2024}
UCSB Human Resources. (2024). Determining Job Classification. Retrieved from https://www.hr.ucsb.edu/hr-units/compensation/determining-job-classification

\bibitem{waterloo2024}
University of Waterloo. (2024). Hay evaluation method. Human Resources. Retrieved from https://uwaterloo.ca/human-resources/support-managers/compensation-information-managers/hay-evaluation-method

\bibitem{wm2024}
William \& Mary. (2024). Classification Variables. Department Resources. Retrieved from https://www.wm.edu/offices/uhr/departments/hiring-officials-business-partners/classification-compensation/class-comp-guide/class-variables/

\end{thebibliography}

\end{document}
